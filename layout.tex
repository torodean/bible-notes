


%Rename the bibliography to References.
\renewcommand\bibname{References}


%%% Header and Footer Info
\pagestyle{fancy}
\fancyhead[LO]{\small {\textbf{Antonius' Bible Notes -- Version \version}}}
\fancyhead[RE]{\small {\textbf{Antonius' Bible Notes -- Version \version}}}
\fancyhead[C]{}
\fancyhead[RO]{\small \thepage}
\fancyhead[LE]{\small \thepage}
\fancyfoot[L]{}
\fancyfoot[C]{}
\fancyfoot[R]{}


%re-defines the plain page style
\fancypagestyle{plain}{%
	\fancyhf{}
	\rhead{\thepage}
	\renewcommand{\headrulewidth}{0pt}}


%Rename the bibliography to References.
\renewcommand\bibname{References}

% Makes a chapter with no title
\makeatletter
\newcommand{\unchapter}[1]{%
	\begingroup
	\let\@makechapterhead\@gobble % make \@makechapterhead do nothing
	\chapter{#1}
	\endgroup
}
\makeatother



% Customize chapter heading to remove "Chapter #" prefix
\titleformat{\chapter}[display]
{} % Font and size of chapter heading
{} % No label (removes "Chapter N")
{0pt} % No separation
{\normalfont\Huge\bfseries} % Chapter title/name formatting
\titlespacing*{\chapter}{0pt}{0pt}{0pt} % Adjust spacing: left, before, after


% Store the current book name for later use
\newcommand{\setbookname}[1]{\def\currentbookname{#1}}
\newcommand{\currentbook}{} % initially empty

% Define a counter for Bible chapters (sections)
\newcounter{biblechapter}
\renewcommand{\thebiblechapter}{\arabic{biblechapter}}

% Command for book title and description: \booktitle{name}{desc}
\newcommand{\booktitle}[2]{%
	\setcounter{biblechapter}{0}       % Reset chapter counter
	\renewcommand{\currentbook}{#1} % set current book
	\chapter*{#1}                % Chapter name
	\setbookname{#1}             % Set the book name for verse labeling.
	\addcontentsline{toc}{chapter}{#1}
	\rule{\textwidth}{0.5pt}\\  % Line below title
	\vspace{-0.4em} \\
	\small \textbf{Summary:} #2 \\  % Description in small font
	\rule{\textwidth}{0.5pt}    % Line below description
	\vspace{0.2cm}              % Space after chapter section
}


% Remove chapter numbers from TOC entries
\renewcommand{\cftchappresnum}{}
\renewcommand{\cftchapaftersnum}{}
\renewcommand{\cftchapnumwidth}{0pt}


% remove tabs for new parahraphs.
\setlength{\parindent}{0pt}


% Command to define a Bible chapter with title and description
\newcommand{\bookchapter}[1]{%
	\thispagestyle{fancy}   % Set the page style to fancy.
	\refstepcounter{biblechapter}%
	\setcounter{bverse}{0}% Reset verse counter
	\section*{Chapter \thebiblechapter: \large #1}%
	\addcontentsline{toc}{section}{Chapter \thebiblechapter: #1}%
}

% Counter for inline notes (resets with each verse)
\newcounter{bversenote}

% Boolean flag to track whether any notes have been used after a verse
\newif\ifnotesafterverse
\notesafterversefalse

% Define verse counter and reset it at each new section
\newcounter{bverse}[section]
\renewcommand{\thebverse}{\arabic{bverse}}

% Verse command
\newcommand{\bverse}[1]{%
	\ifnotesafterverse\vspace{1em}\notesafterversefalse\fi%
	\setcounter{bversenote}{0}%
	\refstepcounter{bverse}%
	\label{\currentbook\space\thebiblechapter:\thebverse}%
	\textsuperscript{\textbf{\thebverse}}#1\ \ignorespaces%
}

% A custom command for referencing scriptures.
\newcommand{\bref}[1]{\hyperref[#1]{#1}}

% Redefine \ref to display the full reference string (e.g., Genesis 3:12)
\renewcommand{\ref}[1]{%
	\edef\temp{\noexpand\csname #1\endcsname}%
	\temp % Display the full reference text
}

% Inline marker that accepts a custom symbol/char and stores it for the note
\newcommand{\vmark}[1]{%
	\refstepcounter{bversenote}%
	\textsuperscript{#1}%
}

% History note
\newcommand{\historicalnote}[2]{%
	\global\notesafterversetrue%
	\par\noindent%
	\hspace*{2em}\begin{minipage}{\dimexpr\textwidth - 2em\relax}%
		{\footnotesize\color{gray!60!black}#1 - \faLandmark:~#2}%
	\end{minipage}\par%
}

% Translation note
\newcommand{\translationnote}[2]{%
	\global\notesafterversetrue%
	\par\noindent%
	\hspace*{2em}\begin{minipage}{\dimexpr\textwidth - 2em\relax}%
		{\footnotesize\color{violet!80!black}#1 - \faLanguage:~#2}%
	\end{minipage}\par%
}

% Context note (literary, cultural, narrative)
\newcommand{\contextnote}[2]{%
	\global\notesafterversetrue%
	\par\noindent%
	\hspace*{2em}\begin{minipage}{\dimexpr\textwidth - 2em\relax}%
		{\footnotesize\color{green!50!black}#1 - \faCompass:~#2}%
	\end{minipage}\par%
}

% Scientific reference note
\newcommand{\sciencenote}[2]{%
	\global\notesafterversetrue%
	\par\noindent%
	\hspace*{2em}\begin{minipage}{\dimexpr\textwidth - 2em\relax}%
		{\footnotesize\color{red!60!black}#1 - \faFlask:~#2}%
	\end{minipage}\par%
}

% Questions (reader or theological questions)
\newcommand{\questionnote}[2]{%
	\global\notesafterversetrue%
	\par\noindent%
	\hspace*{2em}\begin{minipage}{\dimexpr\textwidth - 2em\relax}%
		{\footnotesize\color{orange!70!black}#1 - \faQuestionCircle:~#2}%
	\end{minipage}\par%
}

% General note (uncategorized)
\newcommand{\generalnote}[2]{%
	\global\notesafterversetrue%
	\par\noindent%
	\hspace*{2em}\begin{minipage}{\dimexpr\textwidth - 2em\relax}%
		{\footnotesize\color{blue!70!black}#1 - \faStickyNote:~#2}%
	\end{minipage}\par%
}

% General note (uncategorized)
\newcommand{\sermonnote}[2]{%
	\global\notesafterversetrue%
	\par\noindent%
	\hspace*{2em}\begin{minipage}{\dimexpr\textwidth - 2em\relax}%
		{\footnotesize\color{yellow!60!black}#1 - \faMicrophone:~#2}%
	\end{minipage}\par%
}

% Cross-reference note
\newcommand{\crossrefnote}[2]{%
	\global\notesafterversetrue%
	\par\noindent%
	\hspace*{2em}\begin{minipage}{\dimexpr\textwidth - 2em\relax}%
		{\footnotesize\color{teal!60!black}#1 - \faLink:~#2}%
	\end{minipage}\par%
}

% Geographical note
\newcommand{\geographynote}[2]{%
	\global\notesafterversetrue%
	\par\noindent%
	\hspace*{2em}\begin{minipage}{\dimexpr\textwidth - 2em\relax}%
		{\footnotesize\color{brown!70!black}#1 - \faMap:~#2}%
	\end{minipage}\par%
}

% Theological insight (doctrine) note
\newcommand{\doctrinenote}[2]{%
	\global\notesafterversetrue%
	\par\noindent%
	\hspace*{2em}\begin{minipage}{\dimexpr\textwidth - 2em\relax}%
		{\footnotesize\color{purple!70!black}#1 - \faBookOpen:~#2}%
	\end{minipage}\par%
}

% Literary note (style, metaphor, genre)
\newcommand{\literarynote}[2]{%
	\global\notesafterversetrue%
	\par\noindent%
	\hspace*{2em}\begin{minipage}{\dimexpr\textwidth - 2em\relax}%
		{\footnotesize\color{teal!70!black}#1 - \faFeather:~#2}%
	\end{minipage}\par%
}

% Reset counter at start of each verse
\pretocmd{\bverse}{\setcounter{bversenote}{0}}{}{}

% A custom tab for indenting paragraphs.
\newcommand{\tab}{\hspace*{2em}}

% A custom command to make quoted red text.
\newcommand{\jesus}[1]{\textcolor{red}{``#1''}}

% A section-like marker for various sections.
\newcommand{\bmarker}[1]{%
	\begin{flushright}
		\textbf{#1}%
				
		\vspace{-.8em}%		
		\rule{\linewidth}{0.4pt}%
	\end{flushright}
}
\newcommand{\bmarkerdown}[1]{%
	\bmarker{#1 \faArrowCircleDown}
}
\newcommand{\bmarkerup}[1]{%
	\bmarker{#1 \faArrowCircleUp}
}


% squish the space between itemize items.
\setlist[itemize]{itemsep=0pt, topsep=0pt}