\usepackage[utf8]{inputenc}  % Allows the use of UTF-8 encoding for special characters (e.g., accents, non-ASCII characters).
\usepackage[T1]{fontenc}     % Ensures proper handling of fonts and allows for better hyphenation with T1 font encoding.
\usepackage[top=1in, bottom=1in, left=0.7in, right=.7in]{geometry}  % Customizes page margins for a more tailored document layout.
\usepackage{titlesec}        % Provides control over section title formatting (e.g., for chapters, sections, etc.).
\usepackage{lipsum}          % Generates placeholder text (Lorem Ipsum), useful for testing layout and content.
\usepackage{xcolor}          % Enables the use of colors in the document (e.g., for text, backgrounds, or other elements).
\usepackage{etoolbox}        % Provides advanced programming tools for LaTeX, such as patching existing commands.
\usepackage{enumitem}        % Enhances the formatting of lists (itemize, enumerate, etc.) and provides more customization options.


\usepackage{tocloft}

%used for custom page headers and page numbering
\usepackage{fancyhdr}

%enables indexing
\usepackage{makeidx} 

%enables changing the bibliography name
\usepackage[nottoc,notlof,notlot]{tocbibind}

%makes the index size footnote
\usepackage[font=footnotesize, columns=3]{idxlayout}

\usepackage{booktabs}

\usepackage{fontawesome5}

%enables the ability of including pages from a pdf.
\usepackage{pdfpages}

%Allows for tables with cells that span multiple rows and columns
\usepackage{multirow}

\usepackage{hyperref}

\usepackage{multicol}
\usepackage{etoc}

\usepackage{xspace}