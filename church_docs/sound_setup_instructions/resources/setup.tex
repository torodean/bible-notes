\chapter{Setup}

\section{Setting Up The Stage}

\begin{enumerate}[noitemsep, topsep=0pt]
	\item Bring one of the spare tables from the back of the main room to the stage. Place the table in the center of the stage so that the long edge is parallel with the stairs of the stage.
	\item Put the black cloth cover on the table.
	\item Place the podium in the center of the table, with the black square facing the audience.
	\item Place the podium topper on the podium. Make sure the handle of the podium is not under the topper. Center the topper with the podium.
	\item Place the UCG plack on the stand so that it is centered and in front of the black square of the podium - visible by the audience. Ensure that the ``United Church of God'' text is on the top.
	\item Ensure that Any Pagan Symbols on the Stage (such as crosses, Christmas trees, etc) are removed and stored elsewhere (keep note of these as they will need replaced after services).
	\item Setup a microphone stand. 
		\begin{enumerate}[noitemsep, topsep=0pt]
			\item Place the stand so that two of it's feet are touching the edge of the stage.
			\item The main center pole of the stand should be perpendicular to the ground and touching the table.
			\item Adjust the boon of the microphone so that it has approximately 6 inches of un-extended length. 
			\item Clip the microphone to the stand.
			\item Plug the microphone into the XLR cable. Adjust the XLR cable to wrap around the stand once or twice, with the clips or velcro holding it in place so that it is not hanging down and blends into the stand poles.
			\begin{note}
				XLR Cables transmit Analog data for audio. Therefore, any kinks or twists in the cable can affect the quality of the audio. Take care when handling these cables.
			\end{note}
		\end{enumerate}
\end{enumerate}

\section{Setting Up The Camera}

\begin{enumerate}[noitemsep, topsep=0pt]
	\item Setup the tripod so that it is directly in front of the podium (about halfway between the speaker and the sound booth, though this distance will be corrected in a later step).
	\item Attach the battery to the camera and attach the camera to the tripod.
	\item Open the lens cover and video screen on the camera.
	\item Plug in the power cable to the back-right side of the camera. Wrap the power cable once or twice around the tripod so that is is not hanging down and no one can trip on it.
	\item Attach the short (~6in) HDMI cable to the camera. Plug the long red HDMI cable into the shorter HDMI cable. Wrap this cable once or twice around the tripod so it is out of the way and not hanging down. Use the small velcro strap to hold the red-HDMI cable to the tripod so that the small HDMI cable is not pulling on the mini-HDMI connector. This prevents it from bending over time.
	\item Unroll the red-HDMI cable so that it lays between the sound booth and the camera. It should lay flat (no tripping hazards) and be just long enough to lay a few inches over the wooden booth around the sound table (adjust the tripod distance accordingly to fit this distance).
\end{enumerate}


\section{Setting Up The Laptop}

\begin{enumerate}[noitemsep, topsep=0pt]
	\item Turn the Laptop On and sign on.
	\begin{note}
	The laptop login code is \_\_\_\_\_\_\_\_\_\_\_\_\_\_\_\_\_\_\_\_\_\_\_\_\_\_\_.
	\end{note}
	\item Detach the HDMI and Audio Cables from the JourneyIFC church's computer and plug them into the laptop. When plugging the audio cable in, the laptop may popup with a little window asking which type of connection this is. Select HEADPHONES (this is important).
	\item Plugin the gray USB cable for the Focusrite box into the USB-3 port (blue) on the left side of the laptop (directly next to the audio port).
	\item Plugin the black USB cable for the sermonette timer into the remaining USB port on the left side of the laptop. 
	\item Plugin the Laptop power cable to the large power supply box under the sound table (there is a power outlet on the front right next to the power button). Let the power cable sit over top of the other cables so that there is no unnecessary tension on the other cables.
		\begin{enumerate}[noitemsep, topsep=0pt]
		\item Turn the large power box under the sound table on while there. There is a switch on the top right and bottom right of the box. Both need to be ON. They will be illuminated with a red light when on.
		\end{enumerate}
	\item Plugin the blue capture card USB cable to the USB port on the right side of the laptop.
\end{enumerate}


\section{Setting Up The Projector}

\begin{enumerate}[noitemsep, topsep=0pt]
	\item Using the Projector remote, stand in front of the projector. Point the remote at the projector and push the red power button. You should hear a beep. The projector takes a few minutes to startup but you should start to be able to see the screen turn on after a short period (~20 seconds).
\end{enumerate}

\section{Setting Up The Capture Card}

\begin{enumerate}[noitemsep, topsep=0pt]
	\item Plug the blue USB cable into the USB port on the capture card.
	\item Plug the red HDMI cable into the INPUT HDMI port on the capture card.
\end{enumerate}

\section{Setting Up The Timer}

\begin{enumerate}[noitemsep, topsep=0pt]
	\item Plugin the USB-C cable into the timer.
	\item Hold the power button of the timer to turn it on.
	\item Adjust the brightness so that is is as bright as it can be. Pressing (don't hold) the power button will change the brightness.
	\item Ensure the timer is set to count UP and not down (there should be a small arrow which points up). Use the mode button to change this.	
\end{enumerate}


\section{Setting Up Zoom}

\begin{enumerate}[noitemsep, topsep=0pt]
	\item On the laptop, click the Zoom icon to open zoom. The icon should be on the bottom menu bar.
	\item Click the \textit{start a meeting} button. Once the meeting starts, you no longer need the initial Zoom startup window which appeared and you can close out of it (not the window with the meeting/camera).
	\item If someone has joined the meeting before you have opened it, zoom may show their video and show them as un-muted. Make sure they are muted and their video is OFF by clicking on the 3 dots at the top right of their video frame.
	\item Click on the \textit{Host tools} button on the bottom bar of the zoom window (you might need to click the \textit{More} option if you do not see this). Uncheck everything EXCEPT \textit{Chat} and \textit{Rename Themselves} under the \textit{Allow all participants to} heading. This will ensure there are no interruptions to the zoom feed during services.	
\end{enumerate}


\section{Setting Up The Hymns}

\begin{enumerate}[noitemsep, topsep=0pt]
	\item Find the song leader and get the list of chosen hymns for services. Write down this list on a note card.
	\item Open the \textit{slide page numbers} PDF on the desktop of the laptop. Find the slide numbers for all of the chosen hymns and write them on the note card next to the hymn numbers.
	\item Open VLC Media player (this is a traffic cone icon which is on the bar at the bottom of the screen).
	\item Open the \textit{Hymn MP3s} folder on the desktop.
	\item Find all of the chosen hymns in this folder, and drag/drop each one into the VLC window. You can drag and drop the various hymns in the VLC window to re-order. Make sure they are in the order they will need played.
\end{enumerate}


\section{Setting Up Special Music}

\begin{enumerate}[noitemsep, topsep=0pt]
	\item Open Firefox (using the icon on the bottom bar of the desktop). This should open a page with the church laptop's email and already be logged in.
	\item Open the most recent email which contains special music. There are typically two emails, one containing a doc with the title of special music and one containing the actual audio file (you want the latter). Note: If the special music is a video file, it may download silently and appear as though it is not downloading. 
	\item Download the special music.
	\item If the special music is just an audio file:
		\begin{enumerate}[noitemsep, topsep=0pt]
			\item Move the special music from the downloads folder to the \textit{special music} folder on the desktop of the laptop. Then, with the \textit{special music} folder open, drag the appropriate special music file from that folder into the VLC window.
			\item Move the special music in VLC so that it is the fifth item in the list (it is always played after four hymns).
		\end{enumerate}
	\item If the special music is a video:
		\begin{enumerate}[noitemsep, topsep=0pt]
			\item Open a new VLC window.
			\item Drag the special music video into the new VLC window.
			\item Play the video for a few seconds so that it opens up into a video screen instead of a list.
			\item Scroll the video back to the start.
			\item Drag the video to the second screen (projector screen).
			\item Double click the screen to make it full-screen.
		\end{enumerate}
\end{enumerate}


\section{Setting Up The Hymn Slides}

\begin{enumerate}[noitemsep, topsep=0pt]
	\item On the desktop of the computer, open the hymn slides with the \textit{Hymn Slides} shortcut. A window with google slides will open. Be careful not to press any random/extra keys while this window is selected because it may change the slides.
	\item Drag this window the the second monitor (the projector screen) by dragging it to the right.
	\item Make this window full-screen by selecting the \textit{Presentation Mode} button in Google Slides.
\end{enumerate}