\documentclass[10pt]{article}

%%% These are some packages that are useful
\usepackage{amsmath,amssymb, amscd,amsbsy, amsthm, enumerate}
\usepackage[export]{adjustbox}
\usepackage{lastpage}
\usepackage[top=1in, bottom=1in, left=1in, right=1in]{geometry}
\usepackage[unicode]{hyperref}
\usepackage{tikz, pgfplots, xcolor, fancyhdr}
\usepackage{multicol}
\usepackage{lipsum}

%%% Page formatting
%\setlength{\headsep}{30pt}
\setlength{\textheight}{9in}
\newcommand{\tab}{\hspace{1cm}}
%\setlength{\parindent}{25pt}

\title{Doom And Gloom}
\author{Antonius Torode}

%%% Header and Footer Info
\pagestyle{fancy}
\fancyhead[L]{{\large Template - \textbf{Change 03}}}
\fancyhead[C]{\today}
\fancyhead[R]{Name: Antonius Torode}


\fancyhf{} % sets both header and footer to nothing
\renewcommand{\headrulewidth}{0pt}
% your new footer definitions here

\fancyfoot[L]{}
\fancyfoot[C]{}
\fancyfoot[R]{\thepage\ of \pageref{LastPage}}

% Used to define spacing and format of References
\let\OLDthebibliography\thebibliography
\renewcommand\thebibliography[1]{
	\OLDthebibliography{#1}
	\setlength{\parskip}{0pt}
	\setlength{\itemsep}{0pt plus 0.3ex}
}

%%% Document Starts now
\begin{document}

\maketitle
\thispagestyle{fancy}

\begin{multicols}{2}

Throughout my messages, I've often mentioned the concept of perspective and how it affects what we see, think, and understand. I'm not going to explicitly discuss \textit{perspective} in this message, but it does form a large basis for the foundation of what I will focus on - so keep it in mind throughout. Instead, I'm going to discuss the concept of \textit{doom and gloom}. Starting with the definition, doom and gloom can be defined as 

\begin{quotation}
	``a feeling or attitude that things are only getting worse." \cite{Dictionary}
\end{quotation}

There are two distinct prevailing thoughts I've regularly heard about this topic. The first is that we have far too much focus on \textit{doom and gloom} - whether in out messages or our conversations. In contrast, the second is that we \textit{don't} focus on this too much, and we could potentially focus on it more.

Beginning with the first, do we give too many \textit{doom and gloom} messages? Do we discuss \textit{doom and gloom} too often? Is this too much of our focus? There's a couple different ways to approach this question but the most important is that of \textit{perspective}. I've heard from many different people that we have far too much doom and gloom in our church - or that we have too many messages with the wrong focus. But I've also heard the opposite from many people - that our messages and conversations always tie in hope, so they aren't really \textit{doom and gloom}. I've even heard from some that if the speakers would only move the \textit{hope} part of their messages to \textit{before} the doom part - it would have been a good message. Hopefully you already see where this is going. Each of us are individuals, and we all interpret messages and conversations differently. How you interpret something is \textit{not} how someone else will. It's all a matter of perspective. So do we have too many \textit{doom and gloom} messages? There is no answer, because it's different for everyone. My suggestion then, if you see this as a problem, is to thus change your perspective - which is (in part) where the second thought comes in.

The second thought is then that we don't focus on this too much, and we could maybe focus on it more. Perhaps we have the perfect amount of \textit{doom and gloom} messages. Perhaps we don't have enough. As I mentioned previously, this is all a matter of perspective. For some we have too many, for some not enough, and for some we have just the right amount. Therefore, I cannot say that this is true (that we focus too much on ) - but I can show that it is not false.


First,

\begin{quotation}
	``But if the watchman sees the sword coming and does not blow the trumpet, and the people are not warned, and the sword comes and takes any person from among them, he is taken away in his iniquity; but his blood I will require at the watchman’s hand." - Ezekiel 33:6
\end{quotation}






Second, a huge portion of the Bible is prophecy - some say about a third of the entire Bible. Some people equate prophecy with \textit{doom and gloom}. However, not all prophecy is doom or gloom. But also much of the non-prophetic parts of the Bible are very much full of both doom and gloom. As an example, almost the entirety of Ecclesiastes is \textit{gloomy} - ``all is vanity.'' The Bible is full of both doom and gloom.
One of the primary reasons we come to church is to learn about the Bible. Thus, we \textit{have} to discuss these topics.





\begin{thebibliography}{9}
	{\footnotesize
	\bibitem{Dictionary} Dictionary, ``gloom and doom," \url{https://www.merriam-webster.com/dictionary/gloom%20and%20doom}

	}
\end{thebibliography}

\end{multicols}


%%%%%%%%%%%%%%%%%%%%%%%%%%%%%%%%%%%%%%%%%%%%%%%%%%%%%%%%%%%%%%%%%%%%%%%%%%%%%%%%%%%%%%%%%%%
\end{document}





















