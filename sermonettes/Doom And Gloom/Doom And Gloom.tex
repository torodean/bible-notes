\documentclass[10pt]{article}

%%% These are some packages that are useful
\usepackage{amsmath,amssymb, amscd,amsbsy, amsthm, enumerate}
\usepackage[export]{adjustbox}
\usepackage{lastpage}
\usepackage[top=1in, bottom=1in, left=1in, right=1in]{geometry}
\usepackage[unicode]{hyperref}
\usepackage{tikz, pgfplots, xcolor, fancyhdr}
\usepackage{multicol}
\usepackage{lipsum}

%%% Page formatting
%\setlength{\headsep}{30pt}
\setlength{\textheight}{9in}
\newcommand{\tab}{\hspace{1cm}}
%\setlength{\parindent}{25pt}

\title{Doom And Gloom}
\author{Antonius Torode}

%%% Header and Footer Info
\pagestyle{fancy}
\fancyhead[L]{{\large Template - \textbf{Change 03}}}
\fancyhead[C]{\today}
\fancyhead[R]{Name: Antonius Torode}


\fancyhf{} % sets both header and footer to nothing
\renewcommand{\headrulewidth}{0pt}
% your new footer definitions here

\fancyfoot[L]{}
\fancyfoot[C]{}
\fancyfoot[R]{\thepage\ of \pageref{LastPage}}

% Used to define spacing and format of References
\let\OLDthebibliography\thebibliography
\renewcommand\thebibliography[1]{
	\OLDthebibliography{#1}
	\setlength{\parskip}{0pt}
	\setlength{\itemsep}{0pt plus 0.3ex}
}

%%% Document Starts now
\begin{document}

\maketitle
\thispagestyle{fancy}

\begin{multicols}{2}

Throughout my messages, I've often mentioned the concept of perspective and how it affects what we see, think, and understand. Perspective forms a large part of the foundation for what I will focus on - keep it in mind throughout this message. Consider the concept of \textit{doom and gloom}. Starting with the definition, \textit{doom and gloom} can be defined as 

\begin{quotation}
	``a feeling or attitude that things are only getting worse." \cite{Dictionary}
\end{quotation}

There are two prevailing thoughts I've regularly heard about this topic. The first is that we focus far too much on \textit{doom and gloom} - whether in our messages, conversations or even thoughts. In contrast, the second thought is that we \textit{don't} focus on this too much, and we could potentially even give it more focus.

First, do we give too many \textit{doom and gloom} messages? Do we discuss it too often? Is this too much of our focus? There are several ways to approach this question, but the most important is that the answer is entirely based on individual \textit{perspective}. I've heard from many different people that we have far too much \textit{doom and gloom} in our church - or that we have too many messages with the wrong focus. But I've also heard the opposite from many people - that our messages and conversations always tie in hope, so they aren't really \textit{doom and gloom}. I've even heard from some that if a speaker would have only moved the \textit{hope} part of their messages to \textit{before} the doom part - it would have been a good message. But I've also heard for those exact same messages, that they were very good. Hopefully you already see the pattern. Each of us is an individual, and we all interpret messages and conversations differently. How you interpret something is \textit{not} how someone else will. It's all a matter of perspective. So do we have too many \textit{doom and gloom} messages? There is no answer, because it's different for everyone. My suggestion then, if you see this as a problem, is to thus change your perspective. Ask yourself - why does this topic bother me? This then directly leads into the second thought.

Do we not focus on \textit{doom and gloom} enough? Could we maybe focus on it more? Perhaps we have the perfect amount of \textit{doom and gloom} messages. Perhaps we don't have enough. As I mentioned previously, this is all a matter of perspective. For some we have too many, for some not enough, and for some we have just the right amount. So does that mean we don’t need more focus on this topic? Not necessarily. I won't say that this is true - but I will show that it is not false.

\begin{quotation}
	``But if the watchman sees the sword coming and does not blow the trumpet, and the people are not warned, and the sword comes and takes any person from among them, he is taken away in his iniquity; but his blood I will require at the \textit{watchman’s hand}." - Ezekiel 33:6
\end{quotation}

The world is constantly changing. Within recent years, we saw just how fast something (referring to COVID-19) can sweep into our lives and change almost every aspect of how we live. There were many of us who saw things coming before they happened - the lock downs, the COVID passports, and even the propaganda. Most of those people were ridiculed. Even though things have mostly gone back to `normal', what if they hadn't? We know from the Bible where things in this world \textit{have to} lead. So if we see the sword coming, it is literally our \textit{duty} to warn each other.

Second, a huge portion of the Bible is prophecy - some say about a third of the entire Bible. Some equate prophecy with \textit{doom and gloom}. However, not all prophecy fits into that classification. But also much of the non-prophetic parts of the Bible do. As an example, almost the entirety of Ecclesiastes is \textit{gloomy} - ``all is vanity.'' The Bible is full of both \textit{doom and gloom}. One of the primary reasons we come to church is to learn about the Bible. Thus, it is essential to address these topics.

It is therefore pivotal that we do discuss topics that some of us may see as \textit{doom and gloom}. It is a necessity to having a complete and thorough understanding of the Bible and the times we live in. Because of the different perspective each and every one of us has, we have to also remain diligent in presenting the information in a way that will be receptive. And if we don't hear something we like, maybe we should just try shifting our perspective to see the value that might be there, rather than just shunning it completely. Not all messages might be for you; sometimes someone else might be needing to hear it. So perhaps, instead of asking whether we have too much \textit{doom and gloom}, we should ask how we can better understand and apply the messages we receive. Even messages we don't necessarily like can be essential for our growth.

\begin{quotation}
	``All Scripture is given by inspiration of God, and is profitable for doctrine, for reproof, for correction, for [a]instruction in righteousness, that the man of God may be complete, thoroughly equipped for every good work.'' - 2 Timothy 3:16-17
\end{quotation}

\begin{thebibliography}{9}
	{\footnotesize
	\bibitem{Dictionary} Dictionary, ``gloom and doom," \url{https://www.merriam-webster.com/dictionary/gloom%20and%20doom}

	}
\end{thebibliography}

\end{multicols}


%%%%%%%%%%%%%%%%%%%%%%%%%%%%%%%%%%%%%%%%%%%%%%%%%%%%%%%%%%%%%%%%%%%%%%%%%%%%%%%%%%%%%%%%%%%
\end{document}





















