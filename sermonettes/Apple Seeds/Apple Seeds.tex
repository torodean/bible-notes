\documentclass[9.5pt]{article}

%%% These are some packages that are useful
\usepackage{amsmath,amssymb, amscd,amsbsy, amsthm, enumerate}
\usepackage[export]{adjustbox}
\usepackage{lastpage}
\usepackage[top=1in, bottom=1in, left=1in, right=1in]{geometry}
\usepackage[unicode]{hyperref}
\usepackage{tikz, pgfplots, xcolor, fancyhdr}
\usepackage{multicol}
\usepackage{lipsum}

%%% Page formatting
%\setlength{\headsep}{30pt}
\setlength{\textheight}{9in}
\newcommand{\tab}{\hspace{1cm}}
%\setlength{\parindent}{25pt}

\title{Apple Seeds}
\author{Antonius Torode}

%%% Header and Footer Info
\pagestyle{fancy}
\fancyhead[L]{{\large Template - \textbf{Change 03}}}
\fancyhead[C]{\today}
\fancyhead[R]{Name: Antonius Torode}


\fancyhf{} % sets both header and footer to nothing
\renewcommand{\headrulewidth}{0pt}
% your new footer definitions here

\fancyfoot[L]{}
\fancyfoot[C]{}
\fancyfoot[R]{\thepage\ of \pageref{LastPage}}

% Used to define spacing and format of References
\let\OLDthebibliography\thebibliography
\renewcommand\thebibliography[1]{
	\OLDthebibliography{#1}
	\setlength{\parskip}{0pt}
	\setlength{\itemsep}{0pt plus 0.3ex}
}

%%% Document Starts now
\begin{document}

\maketitle
\thispagestyle{fancy}

\begin{multicols}{2}

While watching a podcast, I saw a T-shirt that said ``Trust God, Not Government”. This is a message that I think all of us understand, but I know all of us do not adhere to. In fact, if we change one word of this, it summarizes a well known scripture.

\begin{quotation}
``It is better to trust in the Lord than to put confidence in man." - Psalm 118:8
\end{quotation}

In other words, trust God, not man. This statement is incredibly easy to understand, however, it is not always easy to live by. We are constantly bombarded with propaganda, information, facts, gossip, advice, and much more. Many of these things we hear sound plausible or make perfect sense to us. We hear things and we adjust our life accordingly. But do we always critique what we hear against God's word? No, we probably do not. I'll begin with a simple example and then take a deep dive into a more subtle one to demonstrate just how important it is that we consider God's word in even the smallest things we hear. 

Consider Butter. Just the other day, I brought in some Kerry-gold butter, from grass-fed Irish cows, to go with some all-organic homemade sourdough banana bread that I made for the congregation. One person asked me something similar to ``but isn't butter bad for you? Because of the saturated fats?”. About a week later, my little brother had sent me a text stating ``you said butter was a good fat, but everywhere I hear that saturated fats are bad. Soooo, I'm confused." That is certainly the case according to Harvard \cite{Butter vs. Margarine}, and The University of Minnesota \cite{Margarines > Butter}, The CDC \cite{Cholesterol Myths} and many more. But if that's the case, why did God describe his promised land as ``a land flowing with milk and honey”\footnote{Exodus 3:8, Numbers 14:8, Deuteronomy 26:9, Ezekiel 20:6, Jeremiah 11:5, Exodus 13:5, Exodus 33:3, Numbers 13:27, Leviticus 20:24, etc...}. If you didn't know, butter is simply made from churning milk (Proverbs 30:33). If one were to dive into the science on butter, you'd see that it is packed with nutrition \cite{butter nutrition}, it reduces inflammation \cite{Butter and Inflammation}, it protects against diabetes\cite{butter superfood, butter and diabetes}, and much more. According to La Pediatria Medica e Chirurgica, an Italian medical journal, it is ``very important for human health'' \cite{Butter from neolithic to current}.

\begin{quotation}
``Butter and honey He shall eat, that He may know to refuse the evil and choose the good.'' - Isaiah 7:15 
\end{quotation}

Honey is ``sweet to the soul and health to the bones" (Proverbs 16:24), and butter is right along side it.

Now, consider Cyanide. Cyanide is a fast-acting poisonous substance containing both Carbon and Nitrogen. This substance can be lethal \cite{The Facts About Cyanides}. Clearly not a good thing for us to consume. The New York State Health Department gives a good summary of where Cyanide is used: 

\begin{quotation}
``Historically, hydrogen cyanide has been used as a chemical weapon. Cyanide and cyanide-containing compounds are used in pesticides and fumigants, plastics, electroplating, photodeveloping and mining. Dye and drug companies also use cyanides. Some industrial processes, such as iron and steel production, chemical industries and wastewater treatment can create cyanides. During water chlorination, cyanogen chloride may be produced at low levels.” \cite{The Facts About Cyanides}
\end{quotation}

This substance is produced by man for many uses - including chemical warfare. Clearly this is not something good for human health - so why am I talking about it? 

According to the Centre for Food safety of Hong Kong, ``Cyanide-containing substances occur naturally in over 2,000 plant species” \cite{Cyanides and Food Safety} This includes the ``seeds or stones of apples, apricots, pears, plums, prunes, cherries, peaches, etc” \cite{Cyanides and Food Safety}. These are not man-made though. God made the fruits. God designed the plants of the earth. We cultivate and breed them, but God designed them. So questions arise - Why do they contain Cyanide? Are they not made for eating? Did God make a mistake?

I think most people have heard that you should not eat apple seeds because they contain Cyanide. I did a quick internet search to see what is suggested. Just two examples follow.

\begin{itemize}
\item According to Medical News Today - ``Apples can be a healthy snack or ingredient. However, it is not advisable to eat the seeds, as they contain small amounts of a chemical that produces cyanide, which is highly toxic." \cite{apple seeds}

\item According to the Times of India - ``Apple is considered as one the healthiest fruits but this same nutritious fruit can turn fatal. Yes, you heard it right. apple seeds are capable of poisoning and causing death." … ``you should immediately spit them out to avoid any potential issues.” \cite{India Times Apple Seeds}
\end{itemize}

When hearing something like this, we typically don't think twice about it. The apple part is for food, but the apple seeds are clearly not. It makes sense because cyanide is poisonous. We may, therefore, just take this information and continue on with our lives. This is not something we'd think to consult our Bible about. However, what if we did? What would we find? 

\begin{quotation}
``And God said, “See, I have given you every herb that yields seed which is on the face of all the earth, and every tree whose fruit yields seed; to you it shall be for food.'' - Genesis 1:29
\end{quotation}

It is clear from this that all fruits yielding seed were made for food. How this verse is worded in English, however, it's not entirely clear if this includes the seeds. So to include them might be a stretch. Let's continue on.

\begin{quotation}
``And out of the ground the Lord God made every tree grow that is pleasant to the sight and good for food...'' - Genesis 2:9
\end{quotation}

\begin{quotation}
``And the Lord God commanded the man, saying, `Of every tree of the garden you may freely eat...'' - Genesis 2:16
\end{quotation}

\begin{quotation}
``And the woman said to the serpent, `We may eat the fruit of the trees of the garden...'' - Genesis 3:2
\end{quotation}

Growing up, back in Michigan, we had both wild blueberries and red currents in our backyard. Wild fruits are \textit{nothing} like those grown for our modern grocery stores. Both the blueberries and red currents were \textit{packed} full of seeds. You could not eat more than a few without also eating some of the seeds. So again, did God make a mistake? Why would he make these fruits so hard to eat if we weren't supposed to eat the seeds as well? You can easily put the doubt to rest with the existence of Ezekiel bread.

\begin{quotation}
``Also take for yourself wheat, barley, beans, lentils, millet, and spelt; put them into one vessel, and make bread of them for yourself. During the number of days that you lie on your side, three hundred and ninety days, you shall eat it.'' - Ezekiel 4:9
\end{quotation}

Ezekiel bread requires seeds to make. So seeds are definitely okay for food. But... does that include apple seeds?

Just for curiosity, I asked ChatGPT\footnote{ChatGPT is an advanced language model developed by OpenAI that generates human-like text based on the input it receives.} ``Are there any fruits with seeds that are poisonous?”. It replied with “Yes, apples, cherries, apricots, peaches, plums”, which is almost identical to our list of Cyanide containing seeds from before. Because I don't trust chatGPT, the government, or man, I did some real digging into the scientific literature. First, these fruits do not directly contain Cyanide. They contain something called amygdalin. Amygdalin is broken down into prunasin by enzymes in digestion. Prunasin is broken down into mandelonitrile in the small intestine. Mandelonitrile then decomposes into benzaldehyde and hydrogen cyanide \cite{Amygdalin}. This hydrogen cyanide is the toxic component of this process, and by this point, it is already inside of you.

The body contains something called Thiosulfate-Cyanide Sulfurtransferase, also known as rhodanese. This rhodanese takes the cyanide and transforms it into thiocyanates \cite{Thiosulfate-Cyanide Sulfurtransferase, Rhodanese}. It is fascinating that the body is \textit{designed} to break down cyanide, but what is even more fascinating is the role of these thiocyanates. These thiocyanates are no longer toxic. In addition, they have antioxidant properties and have the ability to protect cells against oxidizing agents, repair proteins, protect the immune system and more \cite{Thiocyanate}! These compounds can actually be quite beneficial. So then where is the disconnect? Are these seeds good for food or are they bad as man says?

\begin{quotation}
``Have you found honey? Eat only as much as you need, Lest you be filled with it and vomit.'' - Proverbs 25:16
\end{quotation}

The key is that a small amount can be good for you, \textit{moderation} is key, and perhaps that's exactly why God made apple seeds so small. Apricot seeds, on the other hand, are quite large. On top of that, apricot seeds have one of the highest concentrations of amygdalin in the list I provided earlier. I found one article titled `6 foods that can \textit{kill} you' from the Institute of Culinary Education which states ``It's recommended to consume no more than 10 apricot seed kernels a day. But toxins aside, they are quite nutritious and rich in B-vitamins” \cite{apricot seeds}. I adore fruit, especially apricots. That said, I cannot even get close to averaging 10 apricots a day let alone 10 apricot seeds. Just imagine how many apple seeds you would have to eat every day to see any potential negative effects. I would argue that it's not even a concern.

It is safe to assume that God knew what he was doing when he designed these fruits, their seeds, our human bodies, and the interactions we'll have with the environment. So while man says avoid apple seeds because of Cyanide. I choose to believe God knows best. When we hear anything that man or the government has to say, we should always consider whether or not it makes sense from the perspective of God.


\begin{thebibliography}{9}
	{\footnotesize	
	
	\bibitem{Butter vs. Margarine} Harvard Health Publishing. Harvard Medical School. ``Butter vs. Margarine.''  January 29, 2020. \url{https://www.health.harvard.edu/staying-healthy/butter-vs-margarine}
		
	\bibitem{Margarines > Butter} University of Minnesota Research Brief. ``Margarines now nutritionally better than butter after hydrogenated oil ban."  December 13, 2021. https://twin-cities.umn.edu/news-events/margarines-now-nutritionally-better-butter-after-hydrogenated-oil-ban
	
	\bibitem{Cholesterol Myths} Center For Disease Control. ``Cholesterol Myths and Facts." May 15, 2024. \url{https://www.cdc.gov/cholesterol/about/myths.html}	
	
	\bibitem{Butter and Inflammation} Letícia A Penedo, et. al. ``Intake of butter naturally enriched with cis9,trans11 conjugated linoleic acid reduces systemic inflammatory mediators in healthy young adults.'' 2013 Dec;24(12):2144-51.
	doi: 10.1016/j.jnutbio.2013.08.006. \url{https://pubmed.ncbi.nlm.nih.gov/24231103/}
	
	\bibitem{butter nutrition} U.S. Department of Agriculture. FoodData Central. ``Butter, salted.'' April, 2019. \url{https://fdc.nal.usda.gov/fdc-app.html#/food-details/173410/nutrients}
	
	\bibitem{butter superfood} Jackie Patti. ``butter is a true superfood.'' \url{https://www.deductiveseasoning.com/2014/03/butter-is-true-superfood.html}
	
	\bibitem{butter and diabetes} Gao Z, Yin J, Zhang J, Ward RE, Martin RJ, Lefevre M, Cefalu WT, Ye J. Butyrate improves insulin sensitivity and increases energy expenditure in mice. Diabetes. 2009 Jul;58(7):1509-17. doi: 10.2337/db08-1637. Epub 2009 Apr 14. PMID: 19366864; PMCID: PMC2699871. \url{https://pubmed.ncbi.nlm.nih.gov/19366864/}
	
	\bibitem{Butter from neolithic to current} Caramia, G., Losi, G., Frega, N., Lercker, G., Cocchi, M., Gori, A., \& Cerretani, L. (2012). Milk and butter. From the Neolithic to the current nutritional aspects. La Pediatria Medica E Chirurgica, 34(6). https://doi.org/10.4081/pmc.2012.51 \url{https://www.pediatrmedchir.org/pmc/article/view/51}
	
	\bibitem{The Facts About Cyanides} New York State Department Of Health. ``The Facts About Cyanides.'' April 2006. \url{https://www.health.ny.gov/environmental/emergency/chemical_terrorism/cyanide_general.htm}
	
	\bibitem{Cyanides and Food Safety} Mr. Jonny CHU, Scientific Officer. Centre for Food Safety. ``Cyanides and Food Safety.'' September, 2015. \url{https://www.cfs.gov.hk/english/multimedia/multimedia_pub/multimedia_pub_fsf_110_01.html}
	
	\bibitem{apple seeds} Atli Arnarson Ph.D, ``What happens if you eat apple seeds?'' February 16, 2024. \url{https://www.medicalnewstoday.com/articles/318706}
	
	\bibitem{India Times Apple Seeds} Times Entertainment. ``Apple seeds can be poisonous! Here’s what happens when you eat them.'' July 2017. https://timesofindia.indiatimes.com/life-style/health-fitness/diet/apple-seeds-can-be-poisonous-heres-what-happens-when-you-eat-them/articleshow/59632823.cms
	
	\bibitem{Amygdalin} Qadir M, Fatima K (2017) Review on Pharmacological Activity of Amygdalin. Arch Can Res Vol.5:No.4:160. \url{https://pdfs.semanticscholar.org/cc0d/2f32b1fbf5e4ea6889857c56c317723a0e6b.pdf}
	
	\bibitem{Thiosulfate-Cyanide Sulfurtransferase} Buonvino S, Arciero I, Melino S. Thiosulfate-Cyanide Sulfurtransferase a Mitochondrial Essential Enzyme: From Cell Metabolism to the Biotechnological Applications. International Journal of Molecular Sciences. 2022; 23(15):8452. https://doi.org/10.3390/ijms23158452. \url{https://www.mdpi.com/1422-0067/23/15/8452}
	
	\bibitem{Rhodanese} John Westley. ``Thiosulfate: Cyanide sulfurtransferase (Rhodanese), Methods in Enzymology, Academic Press, Volume 77, 1981, Pages 285-291. https://doi.org/10.1016/S0076-6879(81)77039-3. \url{https://www.sciencedirect.com/science/article/abs/pii/S0076687981770393}
	
	\bibitem{Thiocyanate} Joshua D. Chandler. ``THIOCYANATE: A potentially useful therapeutic agent with host defense and antioxidant properties.'' Biochemical Pharmacology. Volume 84, Issue 11, 1 December 2012, Pages 1381-1387. https://doi.org/10.1016/j.bcp.2012.07.029. \url{https://www.sciencedirect.com/science/article/abs/pii/S0006295212005114}
	
	\bibitem{apricot seeds} Pamela Vachon. Institute of Culinary Education. ``Handle With Care: 6 Foods That Can Kill You.'' March 16, 2022. \url{https://www.ice.edu/blog/harmful-foods}
	}
\end{thebibliography}

\end{multicols}

%%%%%%%%%%%%%%%%%%%%%%%%%%%%%%%%%%%%%%%%%%%%%%%%%%%%%%%%%%%%%%%%%%%%%%%%%%%%%%%%%%%%%%%%%%%

\end{document}





















