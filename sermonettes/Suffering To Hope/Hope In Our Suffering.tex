\documentclass[9.5pt]{article}

%%% These are some packages that are useful
\usepackage{amsmath,amssymb, amscd,amsbsy, amsthm, enumerate}
\usepackage[export]{adjustbox}
\usepackage{lastpage}
\usepackage[top=1in, bottom=1in, left=1in, right=1in]{geometry}
\usepackage[unicode]{hyperref}
\usepackage{tikz, pgfplots, xcolor, fancyhdr}
\usepackage{multicol}
\usepackage{lipsum}

%%% Page formatting
%\setlength{\headsep}{30pt}
\setlength{\textheight}{9in}
\newcommand{\tab}{\hspace{1cm}}
%\setlength{\parindent}{25pt}

\title{Hope In Our Suffering}
\author{Antonius Torode}

%%% Header and Footer Info
\pagestyle{fancy}
\fancyhead[L]{{\large Template - \textbf{Change 03}}}
\fancyhead[C]{\today}
\fancyhead[R]{Name: Antonius Torode}


\fancyhf{} % sets both header and footer to nothing
\renewcommand{\headrulewidth}{0pt}
% your new footer definitions here

\fancyfoot[L]{}
\fancyfoot[C]{}
\fancyfoot[R]{\thepage\ of \pageref{LastPage}}

% Used to define spacing and format of References
\let\OLDthebibliography\thebibliography
\renewcommand\thebibliography[1]{
	\OLDthebibliography{#1}
	\setlength{\parskip}{0pt}
	\setlength{\itemsep}{0pt plus 0.3ex}
}

%%% Document Starts now
\begin{document}

\maketitle
\thispagestyle{fancy}

\begin{multicols}{2}

With the advent of some recent neuroscience studies, an interesting connection can be made to a popular biblical concept. Suffering is not often thought of in a positive connotation, but through study of a small section of the brain in relation to various forms of suffering, the benefits can be seen and understood. This relation puts a unique understanding to a popular and well known scripture.

\begin{quotation}
``3. Not only so, but we also glory in our sufferings, because we know that suffering produces perseverance; 4. perseverance, character; and character, hope.'' - Romans 5:3-4 (NIV)
\end{quotation}

From this, we see that suffering produces perseverance, perseverance produces character, and character produces Hope. Through examination of the Anterior MidCingulate Cortex (aMCC), we can better understand just how this chain comes to fruition within our lives.

Humans have a small section of the brain called the Cingulate Cortex. The Cingulate Cortex contains a smaller section called the Anterior Cingulate Cortex (ACC). The ACC plays a crucial role in initiation, motivation, and goal-directed behaviors \cite{ACC}. Each person has two ACC, one on each side of the brain. Within the ACC is an even smaller part of the brain called the Anterior MidCingulate Cortex (aMCC). The aMCC is a crucial and greatly interconnected part of the brain that is a prime candidate for new research.

\begin{quotation}
``Its position as a structural and functional hub allows the aMCC to integrate signals from diverse brain systems to predict energy requirements that are needed for attention allocation, encoding of new information, and physical movement, all in the service of goal attainment. \cite{aMCC}''
\end{quotation}

The aMCC has many interesting connections. There is a subgroup of humans referred to as `super agers'. These are people who are typically 60+ but with the cognitive capabilities of people in their 30's or 40's. One thing in common among super agers is that they all have a larger and more active aMCC. It is also closely tied to a person's willpower - their ability to either do or resist things they do not want to do. Conversely, a smaller aMCC is closely linked with people who have depression and struggle to find motivation. The aMCC is a vital component to nurture and grow for healthy and vital mental health. 

The word translated to \textit{sufferings}\footnote{Word: qliyij; pronounced: thlip'-sis; Strongs Number: G2347; Orig: from 2346; pressure (literally or figuratively):--afflicted(-tion), anguish, burdened, persecution, tribulation, trouble. G2346} (from Romans 5:3) means a pressing, pressure, metaphorical oppression, affliction, tribulation, distress, or strait. In essence, this is things that someone would not want to go through or a situation they would not want to be in. Some examples could include working out, dieting, or writing sermonettes (although this list is very subjective on an individual level). It turns out, that when you do things that you do not want to do, the aMCC grows! It's been shown to be larger in people who successfully diet, people who workout more often, and many more cases. Therefore suffering has a positive impact on the aMCC.

The aMCC has the impact it does because of the way it functions.

\begin{quotation}
``We present evidence that the aMCC, positioned at the intersection of multiple brain networks, is wired to integrate signals relating to allostasis with its sensory consequences, termed interoception, as well as with cognitive control processes, sensory, and motor functions. \cite{Allostasis}'' 
\end{quotation}

``Allostasis is defined as the process of maintaining homeostasis through the adaptive change of the organism's internal environment to meet perceived and anticipated demands \cite{Allostasis_defn}.'' This is the brains ability to produce and anticipate the necessary signals and processes for what the body demands. When your brain can properly anticipate the demand of what it will be put through, you will have the capability to withstand and do more. The ability to draw on willpower to accomplish or resist what you need to is directly correlated to the functions of the aMCC. Another name for this is perseverance. The term translated to \textit{perseverance}\footnote{Word: upomonh; Pronounce: hoop-om-on-ay'; Strongs Number: G5281; Orig: from 5278; cheerful (or hopeful) endurance, constancy:--enduring, patience, patient continuance (waiting). G5278} (from Romans 5:3-4) means steadfastness, constancy, endurance and sustaining. It is the ability to continue forward in the face of challenge. Another name for this is Tanacity.

\begin{quotation}
``Tenacity–persistence in the face of challenge–has received increasing attention, particularly because it contributes to better academic achievement, career opportunities and health outcomes.\cite{Tenacity}''
\end{quotation}

The aMCC directly communicates with the brain and body to modulate Tenacity \cite{HubermanLab}. It gives you the sense to resist, move, and act. In turn, these abilities give better performance in all aspects of life. The stronger your Tanacity, the better your health, career opportunities, achievements, and much more. All of these facets combine to define who you are and form (by definition) someone's character\footnote{Dictionary definition: ``The mental and moral qualities distinctive to an individual.''}. The term translated to \textit{character}\footnote{Word: dokimh; Pronounce: dok-ee-may'; Strongs Number: G1382; Orig: from the same as 1384; test (abstractly or concretely); by implication, trustiness:--experience(-riment), proof, trial. G1384} (from Romans 5:3-4) means a tried character. We all know this can come from suffering and hardship, but now you see how distresses actually modify your brain to facilitate these changes.

Along with all these other unique relationships to the aMCC, there is one final piece that emerges from these recent studies. I find this best summarized by a direct quote from Dr. Andrew Huberman. 

\begin{quotation}
``Scientists are starting to think of the aMCC not just as the seat of willpower, but perhaps as the seat of the will to live. \cite{Goggins}''
\end{quotation}

The will to live is hope\footnote{Dictionary definition: “A feeling of expectation and desire for a certain thing to happen.”}. The term translated to \textit{hope}\footnote{Word: elpij; Pronounce: el-pece'; Strongs Number: G1680; Orig: from a primary elpo (to anticipate, usually with pleasure); expectation (abstractly or concretely) or confidence:--faith, hope.} (from Romans 5:3-4) means an expectation of evil or good, or having hope. It typically relates to a desirable outcome. If we look forward to the future, we have a much stronger will to live. And this comes about through having a well developed aMCC.

Finally, in Romans 5:3, the phrase used is \textit{glory in our sufferings}. In the English translations I've seen, this phrase is always plural. This implies that it's not just a one time hardship we are to glory in. When looking at the aMCC studies, the aMCC can be built up through suffering. But it can also be broken down through comfort. We should glory in this because we can clearly see what changes hardships can bring. We glory in doing things we do not like to do because it builds our character and helps us develop perseverance. When we live in comfort and don't experience distress, the aMCC actually breaks down and weakens. We lose these benefits. This process is a continual life long process of growth and development. So when we have to live through hardships, that's a good thing. With this understanding of the aMCC, this scripture should have a whole new meaning.

\begin{quotation}
``3. Not only so, but we also glory in our sufferings, because we know that suffering produces perseverance; 4. perseverance, character; and character, hope.'' - Romans 5:3-4 (NIV)
\end{quotation}

\begin{thebibliography}{9}
	{\footnotesize
	
	\bibitem{ACC} Orrin Devinsky, et. al., ``Contributions of anterior cingulate cortex to behaviour.'' Brain, Volume 118, Issue 1, February 1995, Pages 279–306, https://doi.org/10.1093/brain/118.1.279
	
	\bibitem{aMCC} Touroutoglou A, Andreano J, Dickerson BC, Barrett LF. The tenacious brain: How the anterior mid-cingulate contributes to achieving goals. Cortex. 2020 Feb;123:12-29. doi: 10.1016/j.cortex.2019.09.011. Epub 2019 Oct 9. PMID: 31733343; PMCID: PMC7381101.
		
	\bibitem{HubermanLab} ``How to Increase Your Willpower \& Tenacity | Huberman Lab Podcast,'' https://www.youtube.com/watch?v=cwakOgHIT0E
	
	\bibitem{suffering1} Veronika Job, et. al., ``Beliefs about willpower determine the impact of glucose on self-control.'' Psychological and Cognitive Sciences, August 19, 2013, 110 (37) 14837-14842 https://doi.org/10.1073/pnas.1313475110
	
	\bibitem{Allostasis_defn} https://www.sciencedirect.com/topics/immunology-and-microbiology/allostasis
	
	\bibitem{Allostasis} Alexandra Touroutoglou, et. al., ``Chapter One - Motivation in the Service of Allostasis: The Role of Anterior Mid-Cingulate Cortex.'' Advances in Motivation Science, Volume 6, 2019, Pages 1-25. https://doi.org/10.1016/bs.adms.2018.09.002
	
	\bibitem{Tenacity} Alexandra Touroutoglou, et. al., `The tenacious brain: How the anterior mid-cingulate contributes to achieving goals.'' Cortex,	Volume 123, February 2020, Pages 12-29 https://doi.org/10.1016/j.cortex.2019.09.011
			
	\bibitem{Goggins} ``David Goggins: How to Build Immense Inner Strength,'' https://www.youtube.com/watch?v=nDLb8\_wgX50
	}
\end{thebibliography}

\end{multicols}

%%%%%%%%%%%%%%%%%%%%%%%%%%%%%%%%%%%%%%%%%%%%%%%%%%%%%%%%%%%%%%%%%%%%%%%%%%%%%%%%%%%%%%%%%%%

\end{document}





















