\documentclass[10pt]{article}

%%% These are some packages that are useful
\usepackage{amsmath,amssymb, amscd,amsbsy, amsthm, enumerate}
\usepackage[export]{adjustbox}
\usepackage{lastpage}
\usepackage[top=1in, bottom=1in, left=1in, right=1in]{geometry}
\usepackage[unicode]{hyperref}
\usepackage{tikz, pgfplots, xcolor, fancyhdr}
\usepackage{multicol,caption}
\usepackage{lipsum}
\usepackage[version=4]{mhchem}
\usepackage{float}

%%% Page formatting
%\setlength{\headsep}{30pt}
\setlength{\textheight}{9in}
\newcommand{\tab}{\hspace{1cm}}
%\setlength{\parindent}{25pt}

\title{The Old gods}
\author{Antonius Torode}

%%% Header and Footer Info
\pagestyle{fancy}
\fancyhead[L]{{\large Template - \textbf{Change 03}}}
\fancyhead[C]{\today}
\fancyhead[R]{Name: Antonius Torode}


\fancyhf{} % sets both header and footer to nothing
\renewcommand{\headrulewidth}{0pt}
% your new footer definitions here

\fancyfoot[L]{}
\fancyfoot[C]{}
\fancyfoot[R]{\thepage\ of \pageref{LastPage}}

% Used to define spacing and format of References
\let\OLDthebibliography\thebibliography
\renewcommand\thebibliography[1]{
	\OLDthebibliography{#1}
	\setlength{\parskip}{0pt}
	\setlength{\itemsep}{0pt plus 0.3ex}
}

\newenvironment{Figure}
  {\par\medskip\noindent\minipage{\linewidth}}
  {\endminipage\par\medskip}


%%% Document Starts now
\begin{document}

\maketitle
\thispagestyle{fancy}


\begin{abstract}
TODO - \lipsum[0-1]
\end{abstract}

\begin{multicols}{2}

\begin{quotation}
Observe Israel after the flesh: Are not those who eat of the sacrifices partakers of the altar? 19 What am I saying then? That an idol is anything, or what is offered to idols is anything? Rather, that the things which the Gentiles sacrifice they sacrifice to demons and not to God, and I do not want you to have fellowship with demons. You cannot drink the cup of the Lord and the cup of demons; you cannot partake of the Lord’s table and of the table of demons. Or do we provoke the Lord to jealousy? Are we stronger than He? - 1 Corinthians 10:18-22
\end{quotation}

\begin{quotation}
They provoked Him to jealousy with foreign gods;
With abominations they provoked Him to anger.
They sacrificed to demons, not to God,
To gods they did not know,
To new gods, new arrivals
That your fathers did not fear. - Deuteronomy 32:16-17
\end{quotation}

\begin{quotation}
No one can serve two masters; for either he will hate the one and love the other, or else he will be loyal to the one and despise the other. You cannot serve God and mammon. - Mathew 6:24
\end{quotation}

AI says ""Annuit cœptis" translates to "He [God] favors our undertakings," while "Novus ordo seclorum" means "A new order of the ages." "

\begin{itemize}
\item BRAINSTORMING
\item How do we worship God? We simple live by his word. We keep his commandments and live like him. We don't have to shout on the rooftops. Likewise, how do we worship demons? We simply live in their ways. What we do in silence shows who we worship. So when I hear people say "I see a lot of good people around, some of the nicest people I know are the Gay or transgender, or others of a sinful lifestyles"... Did Satan come to Eve as a kind friend? Or a ruthless beast? He came as a calm, collective, trusting friend. To serve demons does not mean you become a beast.
\item
\end{itemize}

\begin{itemize}
\item QUESTIONS TO ANSWER
\item Why do all cultures throughout history tend towards the same things? The same sins? The same abominations, etc?
\item We have the capabilities of being a great people, a thriving people - why do we not? We have the resources and technology. Yet we do not. Star Trek is the only series which kind of dives into this idea. We have the money - all the money spent on WW2 could have put a hospital and school in every town on the planet.
\end{itemize}


\begin{thebibliography}{9}
	{\footnotesize
		
	\bibitem{tmp} tmp
	
	}
\end{thebibliography}

\end{multicols}


%%%%%%%%%%%%%%%%%%%%%%%%%%%%%%%%%%%%%%%%%%%%%%%%%%%%%%%%%%%%%%%%%%%%%%%%%%%%%%%%%%%%%%%%%%%
\end{document}





















}{den}