\documentclass[10pt]{article}

%%% These are some packages that are useful
\usepackage{amsmath,amssymb, amscd,amsbsy, amsthm, enumerate}
\usepackage[export]{adjustbox}
\usepackage{lastpage}
\usepackage[top=1in, bottom=1in, left=1in, right=1in]{geometry}
\usepackage[unicode]{hyperref}
\usepackage{tikz, pgfplots, xcolor, fancyhdr}
\usepackage{multicol}
\usepackage{lipsum}

%%% Page formatting
%\setlength{\headsep}{30pt}
\setlength{\textheight}{9in}
\newcommand{\tab}{\hspace{1cm}}
%\setlength{\parindent}{25pt}

\title{The Basis Of The Food Laws}
\author{Antonius Torode}

%%% Header and Footer Info
\pagestyle{fancy}
\fancyhead[L]{{\large Template - \textbf{Change 03}}}
\fancyhead[C]{\today}
\fancyhead[R]{Name: Antonius Torode}


\fancyhf{} % sets both header and footer to nothing
\renewcommand{\headrulewidth}{0pt}
% your new footer definitions here

\fancyfoot[L]{}
\fancyfoot[C]{}
\fancyfoot[R]{\thepage\ of \pageref{LastPage}}

% Used to define spacing and format of References
\let\OLDthebibliography\thebibliography
\renewcommand\thebibliography[1]{
	\OLDthebibliography{#1}
	\setlength{\parskip}{0pt}
	\setlength{\itemsep}{0pt plus 0.3ex}
}

%%% Document Starts now
\begin{document}

\maketitle
\thispagestyle{fancy}

\begin{multicols}{2}

I was reading through the Journal of Agricultural and Environmental Ethics, when I came across something referred to as 'the next best thing to humanity' \cite{Alvaro} Last month, the USDA approved the sale of this new product. I won't tell you what it is just yet.




When it comes to food, I always read the label. I do this for many reasons. One is to simply know what it is. In today's society, we see fancy labels and deceptive marketing. It's not always clear what a product may actually be. For example, consider a chicken sausage. You may find this chicken sausage at the store and think its just that - chicken. But in reality, it could have pork casing, tartrazine\footnote{Tartrazine, also known as FD\&C Yellow \#5, is an approved artificial food color that has been widely used in foods and pharmaceuticals for many years.}, preservatives, or perhaps even 20\% camel meat. Another reason I read labels is to know if it is something I think \textit{should} be consumed. The bible has very clear instructions on what animals are clean and which are unclean. But what about petroleum? What about coal? These may seem obvious, but what about those preservatives or the tartrazine that may be in the chicken sausage? Are those good to eat? My previous sermonette was about testing all things for yourself \cite{test all things}, which will very much apply here. I'm not going to tell you here what shouldn't be eaten. I will point you at something to get you thinking about it though and let you decide for yourself, given the tools.

Throughout the Bible, there is a recurring theme of what some refer to as the `spirit of the law' (which some classify as an idiomatic antithesis). The `letter of the law' is often a phrase referring to the laws of the Bible as they are written - a literal interpretation. Although a literal interpretation is often very helpful in understanding and learning the teachings of the Bible, there is generally much deeper meanings and lessons to be gained from the seemingly simple laws. Jesus often elaborated into the intent and `spirit of the law' to those around him. I present two examples found below.

\begin{quotation}
	``You have heard that it was said to those of old, `You shall not murder, and whoever murders will be in danger of the judgment.' But I say to you that whoever is angry with his brother without a cause shall be in danger of the judgment." - Mathew 5:21-22
\end{quotation}

\begin{quotation}
	``You have heard that it was said to those of old, `You shall not commit adultery.’ But I say to you that whoever looks at a woman to lust for her has already committed adultery with her in his heart." - Mathew 5:27-28
\end{quotation}

These literal laws have an immediate interpretation that is easy to understand but each also contain potential for deeper understandings, which is the spirit of the laws. It is not enough to just live by the letter of the law - as Jesus himself stated. 

Another repeating theme of the Bible is that God's people are to keep themselves Holy and separate from others. We are taught that our bodies belong to God and we are to treat them as such by taking care of them the best we are able. 

\begin{quotation}
	"Do you not know that you are the temple of God and that the Spirit of God dwells in you? If anyone defiles the temple of God, God will destroy him. For the temple of God is holy, which temple you are." - 1 Corinthians 3:16-17
\end{quotation}

One of the fundamental teachings of the Bible and the United Church of God is commonly referred to as `God's Food Laws'. One such way that we keep ourselves Holy is by keeping the food laws described in the Bible. It is a simple enough law to understand that even the children of our congregations learn it and understand it. The food laws were given to ancient Israel so that they could separate themselves and keep themselves from `unclean' things. Here is a short excerpt of the food laws.

\begin{quotation}
	"These are the animals which you may eat: the ox, the sheep, the goat, the deer, the gazelle, the roe deer, the wild goat, the mountain goat, the antelope, and the mountain sheep. And you may eat every animal with cloven hooves, having the hoof split into two parts, and that chews the cud, among the animals. Nevertheless, of those that chew the cud or have cloven hooves, you shall not eat, such as these: the camel, the hare, and the rock hyrax; for they chew the cud but do not have cloven hooves; they are unclean for you. Also the swine is unclean for you, because it has cloven hooves, yet does not chew the cud; you shall not eat their flesh or touch their dead carcasses." - Deuteronomy 14:4-8
\end{quotation}

This is just part of God's food laws. Notice how it isn't even just not eating them, it's extended to not touching dead carcasses as well. No doubt there's a bit of cleanliness tied into being holy. Now, remember our chicken sausage. We can apply this law to clearly see that that the pork casing and the camel meat are not edible. But what about the tartrazine or the preservatives? Is there another scripture we can use when considering these things? There is. In fact, there's a scripture that I would argue summarizes the entirety of the food laws. The very first part of Deuteronomy, before any of the clean and unclean meats are listed, is to me the key to understanding the `spirit of the law' in this context. In order to remain holy, do we only consider what meats we eat?

\begin{quotation}
	``For you are a holy people to the Lord your God, and the Lord has chosen you to be a people for Himself, a special treasure above all the peoples who are on the face of the earth. \textit{You shall not eat any detestable thing.}" - Deuteronomy 14:2-3
\end{quotation}

This last part refers to `detestable' things. In the Greek, 'detestable thing' is Darez\footnote{Strong H8441}, which can be translated as ``a disgusting thing, abomination, abominable'' or ``in ethical sense (of wickedness etc)''\cite{Strongs}. If something can be considered detestable, we are to avoid it. So then how do we determine what is detestable? One way is to apply some simple human reasoning and to search for answers in order to test and understand things for ourselves. Back to our chicken sausage example - let's assume it no longer contains the pork casing or camel meat. We know those are unclean based on the `letter of the law'. That would mean it's only chicken, tartrazine, and preservatives. Let's assume the preservative is just sea salt. Sea salt is of course okay to eat\footnote{Job 6:6, Leviticus 2:13, Mark 9:50, Ezra 6:9, Colossians 4:6}. Then we are just left with the question of whether tartrazine is a detestable thing or not. I decided to look up tartrazine in ScienceDirect (an academic journal for publishing research). Some notable statements I found were as follows \cite{tartrazine}.

\begin{enumerate}
	\item ``Tartrazine is one of the FDA-approved azo dyes for use as food additives.''
	
	\item ``Ingestion of tartrazine is associated with adverse reactions (asthma and chronic hives) in a sensitive subpopulation of consumers (Lockey, 1959).''
	
	\item ``Studies have shown that intake of tartarzine can cause a series of biochemical markers changes at both higher doses and low doses, which are significantly harmful to asthma patients and children at higher doses (Amin, Hameid, \& Elsttar, 2010).''
	
	\item ``the European Union currently requires foods that contain ... Yellow 5 and 6, to be labeled as `may have an adverse effect on activity and attention in children.' ''
	
	\item ``The risks of allergic reactions are 4–14\% of those with asthma or allergies or both...''
	
	\item ``Synthetic or artificial colors are derived from petroleum or coal...''
	
	\item And so on...
\end{enumerate}

I previously said I'm not going to tell you what shouldn't be eaten. However, personally, I would consider something fitting these descriptions as detestable. 

Let us now go one step further and assume that this isn't in our chicken sausage anymore. Now our chicken sausage is just chicken and sea salt. Suppose you find this chicken sausage at the store. The label looks enticing and you're drawn to the mouthwatering packaging. You pick it up, read the ingredients \textit{`chicken and sea salt'}, and conclude it's good to eat. It's that simple. Right? Well, I started this message by mentioning that last month, the USDA approved the sale of a new product. I didn't tell you what it was. This new product uses bioreactors\footnote{Bioreactors are vessels or tanks in which whole cells or cell-free enzymes transform raw materials into biochemical products and/or less undesirable by-products \cite{bioreactors}.} for cellular harvest in tissue engineering-based cellular agriculture\cite{LabGrownMeatReview}. In more simple terms, cultured meat (also called in vitro, artificial or lab-grown meat was just approved for sale in the U.S. God designed certain animals to be eaten. If we study similarities between those animals we see common functions of digestion, toxin filtering, nutrition, and more. If we study the effects of eating unclean animals, we can easily identify potential dangers associated with them. When it comes to artificial meat, each and every one of us will have to make the determination whether this could be a detestable thing or not. It's not clear whether these products will have ingredient labels that just state `chicken', or expanded to include things like `bovine foetal blood cultures'\cite{LabGrownMeatReview}. 


It is clear that the food laws are beneficial to our physical bodies. And despite how resilient the human body is, certain things, when consumed, can cause harm - whether apparent or not. It is our duty to remain holy and care for ourselves so that we can remain a healthy and holy people. For us to do this, we must remain diligent, we must put things to the test, and we must ensure that what we consume is not detestable or an abomination to the Lord.



\begin{thebibliography}{9}
	{\footnotesize
	\bibitem{test all things} Torode, A. Test All Things. 2023.
	
	\bibitem{Alvaro} Alvaro, C. Lab-Grown Meat and Veganism: A Virtue-Oriented Perspective. J Agric Environ Ethics 32, 127–141 (2019). https://doi.org/10.1007/s10806-019-09759-2
	
	\bibitem{Strongs} https://godrules.net/library/kjvstrongs/kjvstrongsdeu14.htm
	
	\bibitem{tartrazine} https://www.sciencedirect.com/topics/agricultural-and-biological-sciences/tartrazine
	
	\bibitem{bioreactors} https://www.sciencedirect.com/topics/immunology-and-microbiology/bioreactor
	
	\bibitem{LabGrownMeatReview} Dr Bedanta Roy, et. al., A review on lab-grown meat: Advantages and disadvantages. Review Article. Quest International Journal of Medical and Health Sciences. July 2021.
	}
\end{thebibliography}

\end{multicols}


%%%%%%%%%%%%%%%%%%%%%%%%%%%%%%%%%%%%%%%%%%%%%%%%%%%%%%%%%%%%%%%%%%%%%%%%%%%%%%%%%%%%%%%%%%%
\end{document}





















