\documentclass[10pt]{article}

%%% These are some packages that are useful
\usepackage{amsmath,amssymb, amscd,amsbsy, amsthm, enumerate}
\usepackage[export]{adjustbox}
\usepackage{lastpage}
\usepackage[top=1in, bottom=1in, left=1in, right=1in]{geometry}
\usepackage[unicode]{hyperref}
\usepackage{tikz, pgfplots, xcolor, fancyhdr}
\usepackage{multicol}
\usepackage{lipsum}

%%% Page formatting
%\setlength{\headsep}{30pt}
\setlength{\textheight}{9in}
\newcommand{\tab}{\hspace{1cm}}
%\setlength{\parindent}{25pt}

\title{Altruism}
\author{Antonius Torode}

%%% Header and Footer Info
\pagestyle{fancy}
\fancyhead[L]{{\large Template - \textbf{Change 03}}}
\fancyhead[C]{\today}
\fancyhead[R]{Name: Antonius Torode}


\fancyhf{} % sets both header and footer to nothing
\renewcommand{\headrulewidth}{0pt}
% your new footer definitions here

\fancyfoot[L]{}
\fancyfoot[C]{}
\fancyfoot[R]{\thepage\ of \pageref{LastPage}}

% Used to define spacing and format of References
\let\OLDthebibliography\thebibliography
\renewcommand\thebibliography[1]{
	\OLDthebibliography{#1}
	\setlength{\parskip}{0pt}
	\setlength{\itemsep}{0pt plus 0.3ex}
}

%%% Document Starts now
\begin{document}

\maketitle
\thispagestyle{fancy}

\begin{multicols}{2}

The more stratified a society, the fewer people we have as peers (by definition). Living in one of the most individualistic societies as we do here in America, with capitalism allowing us to climb higher and higher on a social and economic ladder, we end up with and endless number of differing social positions and hierarchies. With the increased use of social media, we can pick and choose the numerous groups we want to belong to and affiliate with. It's a system that promotes unlimited individual stratification and gives limited symmetrical and reciprocal relationships which leads to nothing but a lack of altruism. We, as humans, have a fundamental human need for companionship.

To form companionship, one must learn to trust someone. Trust is a basis for how we interact with others, and a large part of how we treat them. If we trust someone, we are much more likely to act in ways that will benefit them over ourselves. However, if we do not trust someone, we are much more likely to act in ways which benefit ourselves over them. 

By applying something referred to as game theory, we can analyze how different types of behaviors effect the outcomes of relationships and society around us. One such example of applying game theory is called the Prisoner's Dilemma. The Prisoner's Dilemma is this

\begin{quotation}
"Suppose each of two prisoners A and B, who are not allowed to communicate with each other, is offered to be set free if he implicates the other. If neither implicates the other, both will receive the usual sentence. However, if the prisoners implicate each other, then both are presumed guilty and granted harsh sentences. \cite{Prisoners Dilemma}"
\end{quotation}

This puzzle demonstrates a conflict between individual and group rationality. On one hand, if the prisoners implicate the other, they may be set free, which is in their best interest. But on the other hand, if both prisoners implicate each other, there may be harsher sentences for both of them. This demonstrates that a group of people who make decisions based on rational self-interest may all end up worse than a group where members act against their own self-interest. On the surface, this doesn't necessarily seem like it would make sense to apply to society. But we can find a biblical example of this. One of the most fundamental scriptures in the bible that I think we all have heard will perfectly demonstrates this principal.

\begin{quotation}
"Therefore, whatever you want men to do to you, do also to them, for this is the Law and the Prophets." - Mathew 7:12
\end{quotation}

This is commonly referred to as the golden rule and it often directly contradicts doing what is right just for your own self-interest. Sometimes doing to others what you'd want them to do to you is not in your best interest, such as in the prisoner's dilemma. So why should we?

I recently came across someone named Nicky Case who created a short story driven game based off the book "The Evolution of Cooperation". This game creates a situation where two fictional characters interact in a scenario similar to the prisoner's dilemma. The characters will get different rewards based on whether they cooperate or cheat the others. If they cooperate, they both get a reward. If both try to cheat each other, they get nothing. And if only one cheats, that player gets a larger reward. The game has characters of varying play styles, such as one that will always cheat, and another that will always cooperate. One that will play detective, and one that follows the golden rule, and so on. 

It then creates large tournaments where many characters of each play style can play against each other to see the various outcomes. For example, if a character always chooses to cheat, and they are against a character who will always cooperate, the cheater will unfortunately always come out ahead. The rounds of the game represent single interactions with other individuals. On small scales, with very few interactions, it actually seems that the cheaters do come out ahead and win. This mirrors our society, for we have less and less interactions and reciprochal relationships as we become more stratified. However, with a large number of interactions, and against a large number of other characters, the characters that always come out on top, having the largest rewards, are those that play by the golden rule. This mirrors a group of people who have many social interactions, such as we are told to do with our fellow human beings. And so there is actually a very strong mathematical truth for the golden rule that works for the benefit of the group, rather than the self-interest of the individual.

However, there's another important factor to this that this game looks at. And that, is mis-communication. Because as humans, and in the context of the golden rule, mistakes can be made. And you may do something that isn't what you want others to do to you. You may say something you don't mean. By adding in this factor into the equation, you see that broken trust can cause great harm to the overall outcome of the group. And communication is one of the most important factors to build trust. A while back, I gave a sermonette on precision of speak. I spoke about the importance of being precise in what you say as to not be mis-understood. But building trust is so much more than just being precise.

\begin{quotation}
"But let your ‘Yes’ be ‘Yes,’ and your ‘No,’ ‘No.’ For whatever is more than these is from the evil one." - Mathew 5:37
\end{quotation}

The most important part of building trust is speaking in a way that is both precise and accurate. Accurate meaning that exactly what you say is what you mean. And by maintaining good communication, the trust between people over time can build, and this principal of the golden rule will flourish and those who follow it will have the best outcome for the entire group. When we build trust with each other, we can help build each other up and improve each and every one of us to be the best versions of ourselves. Something that's not easy without trust. Unfortunately, as I started this message, our society is not one based on the ideas and principals for building trust. We are in a stratified system that demotes altruism. So we must recognize this, adapt, and behave accordingly.

From an individual's perspective, "A candle loses nothing by lighting another candle"\footnote{Quote by James Keller}. However, this principle of doing unto others as you want done to you is so much more than just helping yourself or one other person.

\begin{quotation}
"Love does no harm to a neighbor; therefore love is the fulfillment of the law." -Romans 13:10
\end{quotation}

When we think about God's ultimate plan, He is not created rules and principals to make your life perfect. He's not outlining how to make any one of us better than the others. But on the contrary, he's giving us the foundational principals so that we can all lift each other up and build off one another. He's creating a family that is based on many things, one of which being trust. Our culture can be thought of as a game, and through game theory it can be shown that in the short run, the game defines the players. But in the long run, it's us players who define the game. Our society is not one that's evolving towards conditions for creating trusting and close knit relationships. We have to create those conditions ourselves using these principals. And if we apply game theory, we learn that the golden rule will greatly benefit society as a whole in the long run.

\begin{thebibliography}{9}
	{\footnotesize
	\bibitem{Prisoners Dilemma}  Weisstein, Eric W. "Prisoner's Dilemma." From MathWorld--A Wolfram Web Resource. https://mathworld.wolfram.com/PrisonersDilemma.html 
	
	\bibitem{evolution of trust} https://ncase.me/trust/?fbclid=IwAR2rZKZanoFUBii-LRWj72ROdg71rL3n4ADNMSbRVsxKheZjR2dwcy6siig
	}
\end{thebibliography}

\end{multicols}


%%%%%%%%%%%%%%%%%%%%%%%%%%%%%%%%%%%%%%%%%%%%%%%%%%%%%%%%%%%%%%%%%%%%%%%%%%%%%%%%%%%%%%%%%%%
\end{document}





















