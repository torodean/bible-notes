\documentclass[10pt]{article}

%%% These are some packages that are useful
\usepackage{amsmath,amssymb, amscd,amsbsy, amsthm, enumerate}
\usepackage[export]{adjustbox}
\usepackage{lastpage}
\usepackage[top=1in, bottom=1in, left=1in, right=1in]{geometry}
\usepackage[unicode]{hyperref}
\usepackage{tikz, pgfplots, xcolor, fancyhdr}
\usepackage{multicol}
\usepackage{lipsum}

%%% Page formatting
%\setlength{\headsep}{30pt}
\setlength{\textheight}{9in}
\newcommand{\tab}{\hspace{1cm}}
%\setlength{\parindent}{25pt}

\title{Test All Things}
\author{Antonius Torode}

%%% Header and Footer Info
\pagestyle{fancy}
\fancyhead[L]{{\large Template - \textbf{Change 03}}}
\fancyhead[C]{\today}
\fancyhead[R]{Name: Antonius Torode}


\fancyhf{} % sets both header and footer to nothing
\renewcommand{\headrulewidth}{0pt}
% your new footer definitions here

\fancyfoot[L]{}
\fancyfoot[C]{}
\fancyfoot[R]{\thepage\ of \pageref{LastPage}}

% Used to define spacing and format of References
\let\OLDthebibliography\thebibliography
\renewcommand\thebibliography[1]{
	\OLDthebibliography{#1}
	\setlength{\parskip}{0pt}
	\setlength{\itemsep}{0pt plus 0.3ex}
}

%%% Document Starts now
\begin{document}

\maketitle
\thispagestyle{fancy}

\begin{multicols}{2}

The bible is a large book full of all sorts of stories, information, and wisdom. But how do we approach it and not get overwhelmed? How do we prove to ourselves that the words in here are true and beneficial? Actually, how do we really prove anything? We are constantly bombarded by information, whether it be through headlines, podcasts, the news, social media or elsewhere. What should our approach be to the information all around us so that we can properly navigate this world and figure out what to believe?

The definition of science is

\begin{quotation}
``the systematic study of the structure and behavior of the physical and natural world through observation, experimentation, and the testing of theories against the evidence obtained.''
\end{quotation}

Each and every one of us is a scientist. God created the universe which encompasses science and mathematics. And we are made in His image. Whether we are conscious about it or not, we all observe the world around us. We constantly take in information, experiment, and make conclusions about how things work. As little children we are fascinated by the new. When we see a stovetop we want to touch it to see if it's really hot or if mom is just saying that. Unfortunately, there are far too many adults who lost their passion for good and proper science. There are many who just believe what they hear and don't properly put things to the test. We've all probably heard the saying `just trust the science'. But that's not what being a scientist is about.

\begin{quotation}
``The simple believes every word, But the prudent considers well his steps.'' - Proverbs 14:15 (NKJV)
\end{quotation}

We are not to be simple. The information we establish to be true can be beneficial or detrimental to our future and the future of others. For this reason, we need to be prudent, and consider the things we believe to be true. To prove in an instant that the Bible is true is an impossible task. But have you not put this to the test over years? We live by the words of the Bible, constantly putting it to the test and have you not seen the fruit it bears and have tested that it is indeed good? I'm going to use an example of Alzheimer's research to demonstrate just how important this principal is.

Alzheimer's disease is a progressive neurological disorder that affects the brain, causing problems with memory, thinking, and behavior. It is the most common form of dementia and affects millions of people worldwide. As the disease progresses, individuals with Alzheimer's may have difficulty with daily tasks, experience changes in mood or behavior, and may eventually lose the ability to communicate or recognize loved ones. Scientists have been trying to understand this disease for many decades. In 1991 a paper important to Alzheimer's was published in the journal Science\cite{John}. The authors proposed what is known as the ``amyloid hypothesis," which posited that the accumulation of amyloid-beta proteins in the brain is a key factor in the development of Alzheimer's disease. Years went by with no confirmation of these ideas. It wasn't until 2006, 15 years later, that this important link was confirmed. A paper was published titled ``A specific amyloid-beta protein assembly in the brain impairs memory"\cite{Sylvain}. This paper made the claim that an amyloid-beta protein ``may contribute to cognitive deficits associated with Alzheimer's disease.'' This was considered a breakthrough in the field.

Because of this paper, drug companies were founded, research was started, and money was poured into putting the amyloid hypothesis into practical medicine. Of course this level of pioneering requires much testing, trial, error, and prudence. Over the next few years, drugs started to come out and enter clinical trials. In 2012, Pfizer's experimental drug bapineuzumab, which targeted amyloid-beta entered trials. Unfortunately it failed to show significant benefits in two large clinical trials. But in the spirit of science, you learn from our mistakes, and course correct. In 2016, Biogen's aducanumab showed promise in reducing amyloid plaques, but did not hold up in its clinical benefits. In 2019, aducanumab was halted after a futility analysis suggested it was unlikely to be effective. Over and over drugs were released and tested. Each one failing, but each one being a learning opportunity from the last. 

\begin{quotation}
``Test\footnote{Dokimazw (dok-im-ad'-zo) - To test, examine, prove, scrutinize (to see whether a thing is genuine or not), as metals; To recognize as genuine after examination, to approve, deem worthy.} all things; hold fast what is good.'' - I Thessalonians 5:21 (NKJV)
\end{quotation}

This is the scientific approach. Don't `just trust the science', but test the science. Just like in these clinical trials over the years, we need to put the information we have and the experiences we observe to the test. We can't just accept things as true. That would make us simple minded. Imagine your life as this Alzheimer's research. We all grow and change and continually improve. Without testing what we know, we'd never be able to achieve that. What would happen if we take that first druge targetting amyloid proteins and accepted it as if it worked? It probably would have caused some harm. And it's not a quick process. Sometimes it takes years to gather the necessary information needed. It's a slow and continual process, but it's one that is required and well worth it. Sometimes we can learn something is true or false quickly, but sometimes we could think something is true for years before learning it's false.

In July 2022, within the past year, an article titled ``Blots on a field?" was published in Science\cite{Charles}, a well respected scientific publication. This article makes a profound discovery about the Alzheimer's research over these past few decades. Something somewhat simple that was overlooked over these years of research. What scientist failed to do all these years was something they should have known to do. They did not put the paper from 2006 to the test. They did not test all things. And in fact, they just believed what was said. As it turns out, that 2006 paper was completely fabricated. And over the course of over 15 years following this - billions of dollars were spent, and some say mostly wasted. One neuroscientist named Andrew D. Huberman even says that the toll was over a trillion dollars, because of this one fake paper, and the failure to put it to the test.

Now imagine for a moment. Imagine in your life you were to not test something so important. Imagine you continued to live believing something that could have such a large effect - that you never put to the test. Imagine someone tells you one little thing wrong about the Bible and without testing it, you throw all the years of experience you've had out the window and turn away... The damage could be irreparable. Just as scientists must rigorously test and verify their findings before accepting them as true, so too must we test and examine the ideas, beliefs, and values we encounter in our daily lives. It is only by testing all things and subjecting them to careful scrutiny that we can separate truth from error and avoid being led astray by false or misleading information.

The example of the Alzheimer's research being exposed as fraudulent serves as a powerful reminder of the consequences of failing to test all things. The wasted time, resources, and effort that went into researching a false premise could have been better spent pursuing other avenues of investigation that might have led to more meaningful breakthroughs.

By testing all things and subjecting them to careful scrutiny, we can avoid making similar mistakes in our own lives and ensure that our beliefs and actions are grounded in truth and wisdom.

\begin{quotation}
	``Examine yourselves as to whether you are in the faith. Test yourselves.'' - II Corinthians 13:5 (NKJV)
\end{quotation}

\begin{thebibliography}{9}
	{\footnotesize
	\bibitem{John} ``Alzheimer's Disease: The Amyloid Cascade Hypothesis'', John A. Hardy and Gerald A. Higgins, Science,	10 Apr 1992, Vol 256, Issue 5054, pp. 184-185 DOI: 10.1126/science.1566067
			
	\bibitem{Sylvain} ``A specific amyloid-beta protein assembly in the brain impairs memory'', Sylvain Lesné, et. al., Nature. 2006 Mar 16;440(7082):352-7. DOI: 10.1038/nature04533
	
	\bibitem{Charles} ``Blots on a field'', Science, Vol 377, Issue 6604. Charles Piller. DOI: 10.1126/science.ade0209 
	
	\bibitem{K Beyreuther} ``Amyloid precursor protein (APP) and beta A4 amyloid in the etiology of Alzheimer's disease: precursor-product relationships in the derangement of neuronal function'', K. Beyreuther and C. L. Masters, PMID: 1669714 DOI: 10.1111/j.1750-3639.1991.tb00667.x 
	
	}
\end{thebibliography}

\end{multicols}


%%%%%%%%%%%%%%%%%%%%%%%%%%%%%%%%%%%%%%%%%%%%%%%%%%%%%%%%%%%%%%%%%%%%%%%%%%%%%%%%%%%%%%%%%%%
\end{document}





















