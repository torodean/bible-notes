\documentclass[10pt]{article}

%%% These are some packages that are useful
\usepackage{amsmath,amssymb, amscd,amsbsy, amsthm, enumerate}
\usepackage[export]{adjustbox}
\usepackage{lastpage}
\usepackage[top=1in, bottom=1in, left=1in, right=1in]{geometry}
\usepackage[unicode]{hyperref}
\usepackage{tikz, pgfplots, xcolor, fancyhdr}
\usepackage{multicol,caption}
\usepackage{lipsum}
\usepackage[version=4]{mhchem}
\usepackage{float}
\usepackage{enumitem}

%%% Page formatting
%\setlength{\headsep}{30pt}
\setlength{\textheight}{9in}
\newcommand{\tab}{\hspace{1cm}}
%\setlength{\parindent}{25pt}

\title{Know Thy Enemy: The Unveiling (Part 1)}
\author{Antonius Torode}

%%% Header and Footer Info
\pagestyle{fancy}
\fancyhead[L]{{\large Template - \textbf{Change 03}}}
\fancyhead[C]{\today}
\fancyhead[R]{Name: Antonius Torode}


\fancyhf{} % sets both header and footer to nothing
\renewcommand{\headrulewidth}{0pt}
% your new footer definitions here

\fancyfoot[L]{}
\fancyfoot[C]{}
\fancyfoot[R]{\thepage\ of \pageref{LastPage}}

% Used to define spacing and format of References
\let\OLDthebibliography\thebibliography
\renewcommand\thebibliography[1]{
	\OLDthebibliography{#1}
	\setlength{\parskip}{0pt}
	\setlength{\itemsep}{0pt plus 0.3ex}
}

\newenvironment{Figure}
  {\par\medskip\noindent\minipage{\linewidth}}
  {\endminipage\par\medskip}


%%% Document Starts now
\begin{document}

\maketitle
\thispagestyle{fancy}


%\begin{abstract}
%TODO
%\end{abstract}

\begin{multicols}{2}

How many of us have been at war? Have you (the reader) been in a war before?

\begin{quotation}
For though we walk in the flesh, we do not war according to the flesh. For the weapons of our warfare are not carnal but mighty in God... - 2 Corinthians 10:3-4
\end{quotation}

\begin{quotation}
Put on the whole armor of God, that you may be able to stand against the wiles of the devil. For we do not wrestle against flesh and blood, but against principalities, against powers, against the rulers of the darkness of this age, against spiritual hosts of wickedness in the heavenly places. - Ephesians 6:11-12
\end{quotation}

\begin{quotation}
above all, taking the shield of faith with which you will be able to quench all the fiery arrows of the wicked one.- Ephesians 6:16
\end{quotation}

We are all at war. We are in a spiritual war. This isn't a war against our physical bodies - although that's not excluded. This is a war after our very souls. This is one of the most fundamental concepts of the Bible. We are at war with the rulers of darkness of this age; against the \textit{spiritual} hosts  of wickedness. This is a war greater than any other throughout history. This is a war waged against all of humanity ever since the beginning of time. 

\begin{itemize}[noitemsep]
\item Do you know your enemy? 
\item Should you know your enemy? 
\item How can you know the enemy?
\end{itemize}

These are three questions I will focus on and begin to answer in this message.

\subsection*{Do you know your enemy?}

\begin{quotation}
``All warfare is based on deception. Hence, when we are able to attack, we must seem unable; when using our forces, we must appear inactive; when we are near, we must make the enemy believe we are far away; when far away, we must make him believe we are near.'' - Sun Tzu: The Art of War
\end{quotation}

I could fittingly add to this `when we are at war, it must seem like we aren't'. If we do not realize we are even at war, then the enemy has us just where he wants us. For how can we know an enemy if we are ignorant he is even among us. Any Bible scholar knows of the Devil and his demons. He is mentioned numerous times across scriptures throughout the entire book, from Genesis to Revelation.

The scriptures say that he is not an idle being. He is actively walking about and seeking who to devour\footnote{1 Peter 5:8 - ``Be sober, be vigilant; because your adversary the devil walks about like a roaring lion, seeking whom he may devour.''}. He is constantly working, plotting, and deceiving - and he successfully deceives the whole world\footnote{Revelation 12:9 - ``So the great dragon was cast out, that serpent of old, called the Devil and Satan, who deceives the whole world.''}. Given that, do you see his hand in the world around you? Can you discern his works, plans, or goals?

As one small example, can you imagine a path from where society is today to a world where no one (neither rich or poor) can buy or sell without a mark, or a name, or a number of the beast on their hand or forehead. Can you imagine multiple different paths for the same outcome? When you hear people warn of such things, do you pass them off as a conspiracy or toss them aside as rubbish? Since he deceives the whole world, can you see where his deception is mixed into the things around you? Does it matter whether you do or not?







\subsection*{Should you know your enemy?}

\begin{quotation}
``If you know the enemy and know yourself, you need not fear the result of a hundred battles. If you know yourself but not the enemy, for every victory gained you will also suffer a defeat.'' - Sun Tzu: The Art of War
\end{quotation}

When someone is unaware of the enemies methods, you will not be able to see his influence all around you. You will win battles against him, but you will be unaware he's winning others at the same time. In politics, people often say they voted for the `lesser of two evils' - they'll even go as far as saying that's the right thing to do. But isn't `the lesser of two' evils still an evil? Humans can actively admit to propping up an evil, and claim they are doing good. 

The scriptures say clearly that Satan masquerades as an Angel of light\footnote{2 Corinthians 11:14 - ``And no wonder! For Satan himself transforms himself into an angel of light.''}. Therefore, the works of the devil will look (on the surface) as if they are from God. They will be pleasing to the eye, to the carnal mind, and the body. If you do not understand how the Devil warps the truth, you may fall into his snare thinking you are following truth and doing good. If you do not know he is targeting you or even those around you, how will you know where his arrows will land? 

\begin{quotation}
Therefore submit to God. \textit{Resist} the devil and he will flee from you. - James 4:7
\end{quotation}

Can you ignore how the enemy thinks, just submit to God, and call it safe? Well, yes - but... Can you submit to God? Can you perfectly submit to God, without sinning, twenty-four hours a day, seven days a week? `If' you can, what happens when the devil flees from you? Will he flee to a neighbor, or a brother, to friend in the distant? And when if he flees to a brother, how can you point out the arrows around them if you cannot see them yourself? If your children are under attack, can you show them how to mount a defense if you do not know what weapons are attacking? Is the defense for an arrow the same as the defense for cannon? Will a bomb shelter save you from a computer virus? Metaphorical examples of course, but the point remains; understanding your adversary is not optional.









\subsection*{How can you know your enemy?}

\begin{quotation}
And have no fellowship with the unfruitful works of darkness, but rather expose them. For it is shameful even to speak of those things which are done by them in secret. - Ephesians 5:11-12
\end{quotation}

Expose them. That means talk about, point out, and shine a light on the works of darkness. Our society currently tends to do the opposite. We often keep these things to ourselves out of fear or being labeled a conspiracy theorist or a crazy person. Some people can look around and clearly see the enemies hands at play - but some people never do. It's up to us to expose these things, so that those who don't see them can begin to. If we do not expose them, they will remain a secret. If they remain a secret, then the enemy remains hidden. And finally, if the enemy remains hidden, then we cannot know him.
















\begin{thebibliography}{9}
	{\footnotesize
		
	\bibitem{SunTzu} \url{https://en.wikiquote.org/wiki/Sun_Tzu}
	
	\bibitem{SunTzu2} \url{https://suntzusaid.com/book/3/18/}
	}
\end{thebibliography}

\end{multicols}


%%%%%%%%%%%%%%%%%%%%%%%%%%%%%%%%%%%%%%%%%%%%%%%%%%%%%%%%%%%%%%%%%%%%%%%%%%%%%%%%%%%%%%%%%%%
\end{document}





















}{den}