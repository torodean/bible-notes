\documentclass[10pt]{article}

%%% These are some packages that are useful
\usepackage{amsmath,amssymb, amscd,amsbsy, amsthm, enumerate}
\usepackage[export]{adjustbox}
\usepackage{lastpage}
\usepackage[top=1in, bottom=1in, left=1in, right=1in]{geometry}
\usepackage[unicode]{hyperref}
\usepackage{tikz, pgfplots, xcolor, fancyhdr}
\usepackage{multicol,caption}
\usepackage{lipsum}
\usepackage[version=4]{mhchem}
\usepackage{float}
\usepackage{enumitem}

%%% Page formatting
%\setlength{\headsep}{30pt}
\setlength{\textheight}{9in}
\newcommand{\tab}{\hspace{1cm}}
%\setlength{\parindent}{25pt}

\title{Know Thy Enemy: The Unveiling (Part 1)}
\author{Antonius Torode}

%%% Header and Footer Info
\pagestyle{fancy}
\fancyhead[L]{{\large Template - \textbf{Change 03}}}
\fancyhead[C]{\today}
\fancyhead[R]{Name: Antonius Torode}


\fancyhf{} % sets both header and footer to nothing
\renewcommand{\headrulewidth}{0pt}
% your new footer definitions here

\fancyfoot[L]{}
\fancyfoot[C]{}
\fancyfoot[R]{\thepage\ of \pageref{LastPage}}

% Used to define spacing and format of References
\let\OLDthebibliography\thebibliography
\renewcommand\thebibliography[1]{
	\OLDthebibliography{#1}
	\setlength{\parskip}{0pt}
	\setlength{\itemsep}{0pt plus 0.3ex}
}

\newenvironment{Figure}
  {\par\medskip\noindent\minipage{\linewidth}}
  {\endminipage\par\medskip}


%%% Document Starts now
\begin{document}

\maketitle
\thispagestyle{fancy}


\begin{abstract}
TODO - \lipsum[0-1]
\end{abstract}

\begin{multicols}{2}




--------------------------























How many of us have been at war? Have you (the reader) been in a war before?

\begin{quotation}
For though we walk in the flesh, we do not war according to the flesh. For the weapons of our warfare are not carnal but mighty in God... - 2 Corinthians 10:3-4
\end{quotation}

\begin{quotation}
Put on the whole armor of God, that you may be able to stand against the wiles of the devil. For we do not wrestle against flesh and blood, but against principalities, against powers, against the rulers of the darkness of this age, against spiritual hosts of wickedness in the heavenly places. - Ephesians 11-12
\end{quotation}

We are all at war. We are in a spiritual war. This isn't a war against our physical bodies - although that's not excluded. This is a war after our very souls. This is one of the most fundamental concepts of the Bible. We are at war with the rulers of darkness of this age; against the \textit{spiritual} hosts  of wickedness. This is a war greater than any other throughout history. This is a war waged against all of humanity ever since the beginning of time. 

\begin{itemize}[noitemsep]
\item Do you know your enemy? 
\item Should you know your enemy? 
\item How can you know the enemy?
\end{itemize}

\subsection*{Do you know...}

\begin{quotation}
``All warfare is based on deception. Hence, when we are able to attack, we must seem unable; when using our forces, we must appear inactive; when we are near, we must make the enemy believe we are far away; when far away, we must make him believe we are near.'' - Sun Tzu: The Art of War
\end{quotation}

I could fittingly add to this `when we are at war, it must seem like we aren't'. If we do not realize we are even at war, then we do not know our enemy. For how can we know an enemy if we are ignorant he is even among us.

\subsection*{Should you know...}

\begin{quotation}
``If you know the enemy and know yourself, you need not fear the result of a hundred battles. If you know yourself but not the enemy, for every victory gained you will also suffer a defeat.'' - Sun Tzu: The Art of War
\end{quotation}

If you do not know your enemy, then you will not know 	

\begin{quotation}
Therefore submit to God. \textit{Resist} the devil and he will flee from you. - James 4:7
\end{quotation}

\subsection*{How can you know...}

\begin{quotation}
And have no fellowship with the unfruitful works of darkness, but rather expose them. For it is shameful even to speak of those things which are done by them in secret. - Ephesians 5:11-12
\end{quotation}


















\begin{thebibliography}{9}
	{\footnotesize
		
	\bibitem{SunTzu} \url{https://en.wikiquote.org/wiki/Sun_Tzu}
	
	\bibitem{SunTzu2} \url{https://suntzusaid.com/book/3/18/}
	}
\end{thebibliography}

\end{multicols}


%%%%%%%%%%%%%%%%%%%%%%%%%%%%%%%%%%%%%%%%%%%%%%%%%%%%%%%%%%%%%%%%%%%%%%%%%%%%%%%%%%%%%%%%%%%
\end{document}





















}{den}