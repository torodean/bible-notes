\documentclass[10pt]{article}

%%% These are some packages that are useful
\usepackage{amsmath,amssymb, amscd,amsbsy, amsthm, enumerate}
\usepackage[export]{adjustbox}
\usepackage{lastpage}
\usepackage[top=1in, bottom=1in, left=1in, right=1in]{geometry}
\usepackage[unicode]{hyperref}
\usepackage{tikz, pgfplots, xcolor, fancyhdr}
\usepackage{multicol}
\usepackage{lipsum}

%%% Page formatting
%\setlength{\headsep}{30pt}
\setlength{\textheight}{9in}
\newcommand{\tab}{\hspace{1cm}}
%\setlength{\parindent}{25pt}

\title{Mean What You Say, Say What You Mean}
\author{Antonius Torode}

%%% Header and Footer Info
\pagestyle{fancy}
\fancyhead[L]{{\large Template - \textbf{Change 03}}}
\fancyhead[C]{\today}
\fancyhead[R]{Name: Antonius Torode}


\fancyhf{} % sets both header and footer to nothing
\renewcommand{\headrulewidth}{0pt}
% your new footer definitions here

\fancyfoot[L]{}
\fancyfoot[C]{}
\fancyfoot[R]{\thepage\ of \pageref{LastPage}}

% Used to define spacing and format of References
\let\OLDthebibliography\thebibliography
\renewcommand\thebibliography[1]{
	\OLDthebibliography{#1}
	\setlength{\parskip}{0pt}
	\setlength{\itemsep}{0pt plus 0.3ex}
}

%%% Document Starts now
\begin{document}

\maketitle
\thispagestyle{fancy}

\begin{multicols}{2}
	
We all use words to communicate with each other. We listen to podcasts, we hear sermons, we watch videos, read books, articles, messages, etc. All of these things are full of words. These words we use all have definitions which can be very precise. Sometimes these words have multiple definitions and we have to distinguish which one is appropriate based on the context of a situation. Sometimes even the inflection of how a word is spoken or written can change its meaning. Despite this, it's generally fairly easy to distinguish the meaning of words that are used. I don't think I've said anything in this message so far that is not easy to understand.

Reading and writing skills are some of the first things we learn on the journey of education within our current culture. Even so, misunderstandings are abundant. You can say something that is extremely accurate and yet be entirely misunderstood. At the same time, you could read or hear something that is entirely accurate and completely misunderstand it. Our personal experiences and understandings shape how we interpret the words around us. We can read something one day and get an entirely different meaning out of it than we do reading that same thing years later.

It is our duty in every day life to both use our words to be understood, and understand the words we come across. I'll address only the first half of that coin in this message. I cannot count the number of times in my life that I've spoken words and someone understood them in a way that I did not mean them. No matter how accurate you are, there will always be someone to interpret words you say incorrectly. It is therefore imperative that you try your best to do your due diligence with the words you use. Here are two concepts that you should be following in daily life to make sure that you are misunderstood less often.

The first concept, which may sound familiar, is to `say what you mean'.

\begin{quotation}
``If you don't tidy your room, then you won't get any ice cream." \cite{Mathematician} 
\end{quotation}

This is a quote from one of my mathematical textbooks. The book is rightfully titled \textit{How To Think Like A Mathematician}. Even though this quote has nothing to do with mathematics on the surface, it encompasses the single most important aspect of using words in mathematics. This quote demonstrates the importance of being precise with your words. On the surface, many who hear this would interpret it as one getting ice cream after they tidy their room. However, this statement says \textit{nothing} about what will actually happen if someone tidies their room. It only speaks about what will happen if they \textit{do not} tidy their room.

Some may see this as a petty distinction. Others may even think it's unfair, misleading or intentionally deceptive. On the contrary, it is not. Those who have studied mathematics or some scientific field understand just how important this concept is. Words can have very precise meanings. Everything you say will be interpreted differently by different people. You therefore have a duty to be precise with your words and to `say what you mean'. With anything else, how can you expect to be understood?

\begin{quotation}
``Even things without life, whether flute or harp, when they make a sound, unless they make a distinction in the sounds, how will it be known what is piped or played? For if the trumpet makes an uncertain sound, who will prepare for battle? So likewise you, unless you utter by the tongue words easy to understand, how will it be known what is spoken? For you will be speaking into the air.'' - 1 Corinthians 14:7-9 
\end{quotation}

If you say something that is not precise, using uncertain words, how can you expect to be understood? We all think differently. If your words require personal experiences or thoughts to accurately interpret, then they are not easy to understand. Alternately, If you say something precisely, then there can be no uncertainty as to what it means. But of course, we are human. Sometimes we say things that are very accurate, but it's not what we meant.

The second concept to keep in mind is thus to `mean what you say'.

\begin{quotation}
``But let your `Yes’ be `Yes,’ and your `No,’ `No.’ For whatever is more than these is from the evil one.'' - Mathew 5:37
\end{quotation}

In other words, be true to your word. Don't be a hypocrite. When Jesus walked the earth, He fulfilled everything that he promised. He never said one thing and did another. If He had not been true to his word through His actions, then who could take His words seriously. We must do the same. 

One of the best ways to understand someone's words is through their actions. The old adage says `actions speak louder than words'. That is \textit{only} the case because actions give you a foundation of whether someone is true to their word. If someone says they will be somewhere, and then never shows up - maybe they don't mean what they say. If someone says they enjoy playing a baseball, but never wants to play it - perhaps they are not true to their word. If someone says they obey the biblical food laws, but you see them eating pork - you could interpret their `yes' as not being a `yes'.

But here's a twist. I would argue that this adage is incorrect. `Actions speak louder than words' is a precise blanket statement for \textit{all} actions. Actions only speak \textit{louder} than words when our words are not \textit{loud}\footnote{Within this adage, \textit{loudness} is a metaphor for weight or gravity that words or actions hold, not volume.} enough. When the expectation is that your actions will match your words, then your words hold just as much (if not more) weight than actions. For example, if I ask my wife to pick me up from work one day, I would believe she was going to. The action of her picking me up simply reiterates the words she spoke. But if, for some reason, she didn't show up, would I start doubting her words? No, I would assume something out of the ordinary happened and try to figure it out. The words in that scenario would speak louder than the actions. In contrast, if I say something like `I'll unload the dishwasher' and her thought is `I'll believe it when I see it,' there is little expectation that my words will be fulfilled, making the action itself the true measure of credibility.

When God tells us something clear through His words, we do not have to wait for His actions to take place before understanding it. It should be the same for us. When we speak, it should be known that our words are true and we mean what we say. Our `Yes’ should be `Yes,’ and our `No’ should be `No.’ If it is anything other than that, it comes from the evil one. So say what you mean, and mean what you say.



\begin{thebibliography}{9}
	{\footnotesize
	\bibitem{Mathematician} ``How to Think Like a Mathematician: A Companion to Undergraduate Mathematics". Kevin Houston. Cambridge University Press. 2009.

	}
\end{thebibliography}

\end{multicols}


%%%%%%%%%%%%%%%%%%%%%%%%%%%%%%%%%%%%%%%%%%%%%%%%%%%%%%%%%%%%%%%%%%%%%%%%%%%%%%%%%%%%%%%%%%%
\end{document}





















