\documentclass[10pt]{article}

%%% These are some packages that are useful
\usepackage{amsmath,amssymb, amscd,amsbsy, amsthm, enumerate}
\usepackage[export]{adjustbox}
\usepackage{lastpage}
\usepackage[top=1in, bottom=1in, left=1in, right=1in]{geometry}
\usepackage[unicode]{hyperref}
\usepackage{tikz, pgfplots, xcolor, fancyhdr}
\usepackage{multicol}
\usepackage{lipsum}

%%% Page formatting
%\setlength{\headsep}{30pt}
\setlength{\textheight}{9in}
\newcommand{\tab}{\hspace{1cm}}
%\setlength{\parindent}{25pt}

\title{Seeking Understanding}
\author{Antonius Torode}
\date{March 15, 2025}

%%% Header and Footer Info
\pagestyle{fancy}
\fancyhead[L]{{\large Template - \textbf{Change 03}}}
\fancyhead[C]{March 15, 2025}
\fancyhead[R]{Name: Antonius Torode}


\fancyhf{} % sets both header and footer to nothing
\renewcommand{\headrulewidth}{0pt}
% your new footer definitions here

\fancyfoot[L]{}
\fancyfoot[C]{}
\fancyfoot[R]{\thepage\ of \pageref{LastPage}}

% Used to define spacing and format of References
\let\OLDthebibliography\thebibliography
\renewcommand\thebibliography[1]{
	\OLDthebibliography{#1}
	\setlength{\parskip}{0pt}
	\setlength{\itemsep}{0pt plus 0.3ex}
}

%%% Document Starts now
\begin{document}

\maketitle
\thispagestyle{fancy}

\begin{multicols}{2}

In some of my previous messages, I've spoke about the importance of being precise in speech \cite{Mean What You Say, Precision of Speech}. These messages can be summarized and expanded by the example of Jesus speaking in parables.

\begin{quotation}
``And the disciples came and said to Him, `Why do You speak to them in parables?' He answered and said to them, `Because it has been given to you to know the mysteries of the kingdom of heaven, but to them it has not been given.'" - Mathew 13:10-11

...

``Therefore I speak to them in parables, because seeing they do not see, and hearing they do not hear, nor do they understand." - Mathew 13:13
\end{quotation}
 
Jesus was intentionally not being \textit{precise} in his speech here because it was not meant for those listening to grasp the mysteries he was speaking about. He acknowledges that they will hear the words clearly and yet not understand them. This is a unique case in the sense that despite it seeming like Jesus is not speaking precisely - he is in fact being very precise with his words. Paradoxically, when his disciples were taught these parables, they were able to understand them. When we read them, we are often able to understand them. This is a conundrum which demonstrates a mutually related concept to precise speech - which is precise understanding. 

Speech can be very precise, and yet still be misunderstood. At the same time, it can be inaccurate and be understood completely. Sometimes, the words we use require cultural, personal, or contextual information to be understood. Precise understanding is a concept with two important sides - \textit{to understand}, and \textit{to be understood}.

First, to understand others requires a few important factors. The first factor is that you have to listen.

\begin{quotation}
``To answer before listening - that is folly and shame.'' - Proverbs 18:13
\end{quotation}

The disciples honed in on what Jesus said. They listened carefully to what he had to say. If they had not, how would they have understood it? Secondly, asking questions and getting clarification is key to understanding. The disciples asked questions! When they did not understand what he was saying, they asked him to clarify. Everyone speaks coming from a different background. The old adage says you have to put yourself in someone's shoes in order to see their perspective. However, you can't just put their shoes on, then stand there to see their frame of reference. You have to put their shoes on and explore the world in them. One cannot expect to do this without getting that information of experience directly from them - through asking questions. Shoes get you in their space, but questions get you into their head.

The problem with this metaphor is that we all have different feet. We can put ourselves in someones shoes, but we will never be able to experience the feelings that they have while in their own shoes. This then brings me to the second side of precise understanding - being understood. I spent some time recently in some online debates. I don't often do this, but a physicist from my old university posted something online that had caught my eye. He said

\begin{quotation}
``We should all be outraged...''
\end{quotation}

He then followed this statement with some political statement that he presumably thought everyone should be outraged about. I decided to engage. I countered and said, 

\begin{quotation}
``If you're going to spend your precious years in outrage, there are better things to be outraged about."
\end{quotation}

We had a bit of a follow-up discussion - or perhaps more of a sparring match. Unfortunately, almost immediately, he turned to insulting me, making straw-man arguments (arguing for and against things I never said), putting words in my mouth, and more. I kept trying to make my argument clear - that outrage is not beneficial or the solution. Throughout the conversation, I continually tried to understand why he thinks outrage is beneficial, or even the solution. I kept asking questions specifically related to outrage but his replies were all about Trump, evil billionaires, and other political theater. He rambled on and on, while I had to ask him four different times, in four different ways, why is outrage the answer. His final response was

\begin{quotation}
``My concern isn’t personal outrage...''
\end{quotation}

And there it was. I tried listening, and asking plenty of questions. But I soon realized he didn't actually know what he believes. He's contradicting himself, he's saying very clear and precise statements - then saying the opposite. He's arguing for the sake of arguing. He was not really trying to understand the points I was making or even the arguments he was making.

\begin{quotation}
``Fools find no pleasure in understanding but delight in airing their own opinions.'' - Proverbs 18:2
\end{quotation}

Although he admitted my entire premise was correct, we didn't really come to any agreements in this discussion. But the conversation enhanced an important lesson for me. Precise speech is our job, but being understood takes two. I tried to put on his shoes, I tried to get into his frame of mind, but he would not take a step forward with me. Jesus spoke in parables which were clear to those who listened and nonsense to those who didn't. The disciples asked `why' and they gained understanding because they wanted to know. We can listen, question, and speak truth, but if the other side's just shouting in their place, understanding will fail. Understanding others is not on us alone. It's a journey forward and both parties have to take steps forward.




\begin{thebibliography}{9}
	{\footnotesize
	\bibitem{Mean What You Say} Torode, A. ``Mean What You Say, Say What You Mean.'' January 2025. \url{https://torodean.github.io/sermonettes.html}
	
	\bibitem{Precision of Speech} Torode, A. ``Precision of Speech.'' May 2022. \url{https://torodean.github.io/sermonettes.html}

	}
\end{thebibliography}

\end{multicols}


%%%%%%%%%%%%%%%%%%%%%%%%%%%%%%%%%%%%%%%%%%%%%%%%%%%%%%%%%%%%%%%%%%%%%%%%%%%%%%%%%%%%%%%%%%%
\end{document}





















