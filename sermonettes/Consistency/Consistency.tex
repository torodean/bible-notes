\documentclass[11pt]{article}

%%% These are some packages that are useful
\usepackage{amsmath,amssymb, amscd,amsbsy, amsthm, enumerate}
\usepackage[export]{adjustbox}
\usepackage{lastpage}
\usepackage[top=1in, bottom=1in, left=1in, right=1in]{geometry}
\usepackage[unicode]{hyperref}
\usepackage{tikz, pgfplots, xcolor, fancyhdr}
\usepackage{multicol}
\usepackage{lipsum}

%%% Page formatting
%\setlength{\headsep}{30pt}
\setlength{\textheight}{9in}
\newcommand{\tab}{\hspace{1cm}}
%\setlength{\parindent}{25pt}

\title{The Power Of Consistency}
\author{Antonius Torode}

%%% Header and Footer Info
\pagestyle{fancy}
\fancyhead[L]{{\large Template - \textbf{Change 03}}}
\fancyhead[C]{\today}
\fancyhead[R]{Name: Antonius Torode}


\fancyhf{} % sets both header and footer to nothing
\renewcommand{\headrulewidth}{0pt}
% your new footer definitions here

\fancyfoot[L]{}
\fancyfoot[C]{}
\fancyfoot[R]{\thepage\ of \pageref{LastPage}}

% Used to define spacing and format of References
\let\OLDthebibliography\thebibliography
\renewcommand\thebibliography[1]{
	\OLDthebibliography{#1}
	\setlength{\parskip}{0pt}
	\setlength{\itemsep}{0pt plus 0.3ex}
}

%%% Document Starts now
\begin{document}

\maketitle
\thispagestyle{fancy}

\begin{multicols}{2}
Imagine for a moment. You wake up one day, you look in the mirror, and you go ``ugh, I need to work out''. So you get up, and you go for a long jog. It's 98 degrees and humid out, you're dripping sweat, breathing heavy, and parched for water. You reach a point where you're too tired to continue. About to pass out and starting to get dizzy, you turn back. You're walking home and after passing through some shade and drinking some water, you cool down a bit, catch your breath, and then you've caught your second wind. To your left is the gym. You've been paying for the membership but never regularly use it. Today is the day. You go inside, immediately noticing the air conditioning your energy restores itself instantly. You puff up your chest and hit the weights. An hour later, arms shaking, abs aching, you feel very accomplished. You head home, you start a nice cold shower. You're about to get in, you look over at the mirror, and you flex your heart out. After a brutal workout, you strike a pose and what do you see? \textit{Nothing}.

And so you think to yourself, ``I have to do better tomorrow.'' The next day comes around, and even though you're sore and aching, you go at it again. You lift twice as hard, you run twice as far. You push through with everything you have. And when you get home and strike that pose again - you see nothing.

You think back and recall all those body builders, the strong movie actors, the Olympic athletes and you remember what they do. You make small tweaks here and there. You work on your diet, you make sure to get enough sleep. You envision yourself and scult the perfect you in your mind. You take that vision and you focus on it. You truly believe what you are doing is the right thing, and so you stick to it. You might eat a candy bar here and there, you may skip some days, but you stick to it. Then each and every day you strike that pose and you continue to see nothing. Until one day, you think back to a few months ago and when you're striking that pose, you don't see anything different than yesterday, but you realize, ``I see something.''

Consider Galatians 6:9 which states ``and let us not grow weary of doing good, for in due season we will reap, if we do not give up.'' The concept of consistency is one of the foundational biblical principles taught throughout its pages\footnote{1 Corinthians 15:58, Galatians 6:9, Titus 2:7-8, Hebrews 13:8, Hebrews 10:23, James 1:4}. We can easily do something good and see no result come from it. We could easily grow weary of doing good things if we don't see results from such. But if we consistently do good things, in due season we will see results, if we do not give up.

You don’t get in shape because you went to the gym or ran a mile. You don't become strong because you did a push up. You don't get in shape for going on a long hike. But rather, it’s the daily practice of the little monotonous things that you consistently do that lead to the results. You get in shape because you put in effort each day. You get better at push ups by doing 10 a day, until you can do 20, and then 30, and so on. You may slip up. You may skip leg day, or eat some ice cream a few times. But over time, if you keep with it, you will see results.

Being a Christian, and being in the body of Christ as we strive to do is the exact same thing. There is no one event we can do to become a true member of God's family. There is no single event we can do to say, wow I'm now a member of this church. But rather it's an accumulation of little things we do. Little actions we do and consistency by living God's way as a good human being. All these little actions by themselves could be seen as useless. It's just like doing those push ups. Doing it just once is pointless. But being consistent with it will lead to a season of harvest.

Gods family is the same way. There is no one event that makes someone part of Gods family\footnote{Ephesians 2:19, 1 John 3:1-24, John 13:34-35, 1 Thessalonians 5:11, 1 Timothy 5:8}. Not even baptism can fully bring us into Gods family. For what happens if you don't change after baptism? What happens if you repent only once and never again? You have to do it consistently. Because living a Godly life is about constant growth. Constant betterment of ones self. You cannot just honor your father and mother once, you have to do if every chance you get! And you know what? The system allows for failure. Not because we are meant to fail. But because we are imperfect human beings. We struggle to do what we're suppose to. And if we fail, if we mess up, if we sin, we can repent, we can keep striving towards being a better person, because it's not about a single event, it's about the journey of consistency.

God designed the family structure as an example to what he is creating. We are not meant to be just friends or acquaintances with each other here in the congregation, but brothers and sisters. We're told to develop a deep agape love for one another. Me and my siblings will bicker, and we'll fight, but the love does not go away. We'll bother each other, make fun of each other. But we won't hurt each other - intentionally. And if someone does something against one of us, we're sticking together, as brothers and sisters. And this relationship is not built by one event. It's built by a lifetime of events. And absolutely have I messed up and treated my siblings bad. I've yelled, I've gotten angry. And I've regretted all those times. But it wasn't about those single events, it was about the consistency. 

How do we become brothers and sisters here in the church? How do you take a bunch of people from different backgrounds and different lives and give them that deep connection? Common goals, beliefs, values, love, and consistency. Continuing on in Galations 6:10 - ``Therefore, as we have opportunity, let us do good to all, especially to those who are of the household of faith". We cannot just interact with these people around us. As members of God's church we are becoming brothers and sisters\footnote{Romans 12:10-12 says ``Love one another with brotherly affection. Outdo one another in showing honor. Do not be slothful in zeal, be fervent in spirit, serve the Lord. Rejoice in hope, be patient in tribulation, be constant in prayer.''} with each and every one of us around. We cannot just say hi to a fellow member once and call it good. We have to be consistent with our behavior. We have to love and care for each and every one of us so we can be able to call them family. Brothers and sisters are built through trials and experiences together. That's what we need to be striving to have within our Godly family. 

Just as your body feels great after a good workout and healthy meal, as compared to slow and sluggish after lounging on junk foods, you can tell one action is good for you and one is not. When interacting with each other we need to do good as stated in Galatians. And we can't just do it once, but repeatedly. Consistency is key. \footnote{In I Timothy 5:8, ``But if anyone does not provide for his relatives, and especially for members of his household, he has denied the faith and is worse than an unbeliever,'' we see we have to provide for our relatives.} As human beings we have a fundamental need to be loved and for social interactions. We were built this way and so we have to provide for each and every one of us with consistent love and care. Consistently showing each and every one of us that we think of each other as a family. So that we can be brothers and sisters - and all benefit from it. So, Let us not grow weary of doing good, for in due season we will reap, if we do not give up.

\end{multicols}


%%%%%%%%%%%%%%%%%%%%%%%%%%%%%%%%%%%%%%%%%%%%%%%%%%%%%%%%%%%%%%%%%%%%%%%%%%%%%%%%%%%%%%%%%%%
\end{document}





















