\documentclass[11pt]{article}

%%% These are some packages that are useful
\usepackage{amsmath,amssymb, amscd,amsbsy, amsthm, enumerate}
\usepackage[export]{adjustbox}
\usepackage{lastpage}
\usepackage[top=1in, bottom=1in, left=1in, right=1in]{geometry}
\usepackage[unicode]{hyperref}
\usepackage{tikz, pgfplots, xcolor, fancyhdr}
\usepackage{multicol}
\usepackage{lipsum}

%%% Page formatting
%\setlength{\headsep}{30pt}
\setlength{\textheight}{9in}
\newcommand{\tab}{\hspace{1cm}}
%\setlength{\parindent}{25pt}

\title{It All Goes Back in The Box}
\author{Antonius Torode}

%%% Header and Footer Info
\pagestyle{fancy}
\fancyhead[L]{{\large Template - \textbf{Change 03}}}
\fancyhead[C]{\today}
\fancyhead[R]{Name: Antonius Torode}


\fancyhf{} % sets both header and footer to nothing
\renewcommand{\headrulewidth}{0pt}
% your new footer definitions here

\fancyfoot[L]{ABC}
\fancyfoot[C]{}
\fancyfoot[R]{\thepage\ of \pageref{LastPage}}

% Used to define spacing and format of References
\let\OLDthebibliography\thebibliography
\renewcommand\thebibliography[1]{
	\OLDthebibliography{#1}
	\setlength{\parskip}{0pt}
	\setlength{\itemsep}{0pt plus 0.3ex}
}

%%% Document Starts now
\begin{document}

\maketitle
\thispagestyle{fancy}

\begin{multicols}{2}
Imagine a world. Imagine a world where one percent of the planet's population owns forty seven percent of the planet's wealth\cite{Suisse}. Imagine a world where twenty five thousand people die every single day because they can't afford food \cite{UN}. Imagine a world where polluting the environment usually means higher profits \cite{pollution}. Imagine a world where one company can own more wealth than almost every nation on earth.

It’s not just imagination. This is the world we live in, A world of greed and corruption. A world of pain and suffering. A world of thrills and spending. A world of luxury and desires. Now we may not notice all these things, because they don’t all directly affect us. However, they are all around us every single day. What do all these things have in common? They are all byproducts of our monetary system. Now one could say money has always existed and so what am I talking about. But what I mean by money is not gold, silver, or trading commodities. But rather, a fictional currency that us humans created and decided to give value to, a fiat currency\footnote{Hold up a dollar bill and say "this."}. Money is the lifeblood of all our establishments. The most needed thing on the planet that can't physically be used for much anything practical. You can’t build a house with it. You can’t eat it. You can’t drink it. Some of it magically has more value based on some pretty designs\footnote{Hold up a 5/10/20 dollar bill.}. And most of it doesn't even actually exist outside of our imaginations and computer systems.

I’d like to take a glimpse at the modern money system we have and hopefully leave you with a new perspective to think about as you see how this is being used by Satan. I want to demonstrate how and why it causes suffering. To do so, I’m going to use an analogy. This analogy comes from a book but I've modified it slightly to use here.

Life can be compared to the game of monopoly. My older sister is a great person. She taught me how to play the game monopoly. She knew the name of the game was to acquire and accumulate wealth. She was ruthless, I couldn't beat her. One year I decided I wanted to be the best. I played it over and over until I learned to win that game. I realized the only way to win was through a total commitment to possessions, money and wealth. Consider I Timothy 6:10\footnote{Hold your place here in Timothy.} – ``the love of money\footnote{filarguria: Can also be translated to avarice (extreme greed for wealth or material gain).} is the root of all kinds of evil.” One day I sat down with my sister to play. I was ready to bend the rules to win if I had to. I destroyed her financially and psychologically. I watched her give her last dollar and quit in utter defeat. It felt great!

The love of money and the love of material gain leads to a craving. One who loves money is never satisfied until they have it all. They want more. Like in Monopoly. Now how does this analogy affect us today? Why can we not love money? We need money for rent, we need money for food, we need money to tithe. What’s so bad about it? If we have money, we can give it to the poor. If we have money, we can feed the hungry. Does it not seem like we should love money? Let’s take a brief look at history and how the love of money affects us.

In 1689, John Locke published ``Two Treatises of Government” \cite{TwoTreatises}. In these documents he argued that all men are created equal in the state of nature by God\footnote{Proverbs 22:2, Romans 2:11, Leviticus 19:33-34, Acts 10:34-35, James 2:1-4, Galatians 3:26-29}. In this document, he outlined how private property should be distributed among people. He argued that three provisos need to be considered when dealing with private property. First, you must leave enough (and as good) for others. There was an inherent consideration for other's. A love for neighbors\footnote{Mark 12:30-31}. Second, you must mix your labor with the property. This implied you were using it and making it your own. Lastly, you can only take so much as you can use before it spoils, for "nothing was made by God for a man to spoil or destroy." Back then, money was considered something with value you could trade, generally gold or silver. Gold and silver cannot rot, nor can any other precious metal or gem. Trading it for food can transfer the food around before it rots. Therefore, society can be dedicated to the protection of property. John Locke based his principles on that of protection and eliminating waste and destruction because he believed that God was the one who provides our resources.

However, after detailing this system that would support everyone and eliminate waste, he surprises everyone and states ``the 'one thing' that blocks this is the invention of money, and men's tacit agreement to put a value on it.” With modern money, there's no longer consideration for if there's enough left over for others. Money buys labor, and money buys property whether it goes to waste or not. With money there is no consideration of the well being of others. The goal is instead to acquire money and assets; to become the master of the board.

In 1776, Adam Smith published ``An Inquiry into the Nature and Causes of the Wealth of Nations,” which is one of the worlds fundamental works in classical economics \cite{WealthofNations}. This talks about supply and demand coming into equilibrium which is done by the ‘invisible hand of the market.’ This takes the argument of private property from God giving natural rights for everyone to instead being the system itself as God\footnote{As Dr. John McMurtry puts it; a professor of philosophy who analyzed this text.}. In this document, Mr. Smith states "among inferior ranks of people, the scantiness of subsistence can set limits to the further multiplication of the human species; and it can do so in no other way than by destroying a great part of the children." This is essentially saying that inferior people (which would be those in or born into poverty) will have their children destroyed because they can't support themselves. In other words, it is their own fault, and this is all caused by the invisible hand of the market, which is seen as Gods doing. We can see all of the blame is shifted to God as it so often is. 

These men who created the system knew exactly the ramifications of the system. They also knew how it could make them rich, which was the real purpose of implementing this system. For why else would a system based on preservation and balance be replaced? These men were drowning in greed and they figured out a great way to exploit others. An enlightening quote on the topic, from a documentary by Peter Joseph is this; ``physical slavery requires you to feed and house people, but economic slavery requires people to feed and house themselves.” We are slaves to this system. We have to make money to live. We are living in this world controlled by money and possessions. But we cannot love money. We cannot let the love of material possessions become part of us. Though we live in the world, we cannot be part of it\footnote{I John 2:15, John 15:19}. We do have to understand though, that built directly into the economic system we use today is destruction and suffering all due to the greed of the individuals that created it, and the love of money.

There’s a better way. The wisdom of God is much greater than any of us can comprehend. To me, it’s clear He saw all this coming. God knew what man would do to each other and how we would twist our economics. In fact, He left us a warning to not be the ones doing it ourselves. Look at I Timothy 6:9 – ``But those who desire to be rich fall into temptation and a snare, and into many foolish and harmful lusts which drown men in destruction and perdition.” This says, desiring to be rich, perhaps for example by creating fictional currency with no real value, will drown men in destruction and perdition and create wrong morals and values. Not only can this apply to the self, but could this have a duel meaning applying to others as well? The Greek word for ``men” here is anthropous\footnote{anqrwpoj: a human being, whether male or female; generically, to include all human individuals.}, which in other verses is translated as `others'\footnote{Mathew 5:16, Mathew 6:14-15, etc.}. I think we can see from the statistics I started with how true this is that the love for money can drown others in destruction.

When I beat my sister in Monopoly, I wasn't only drowning myself in destruction by falling to greed, but this was also me drowning her in destruction. After that game, she had one more lesson to teach me. After I beat her, and took everything she had, she looked at me, and she said ``Now it all goes back in the box.” All those houses and cars. All those utilities and railroads. All those titles and clothes. It all goes back in the box. None of it is really yours. The truth is, after we acquire and consume and horde, we’re going to lose it all. It all just belongs to God. And so, you have to think, when you’ve made the ultimate purchase, when you got that big promotion, when you've climbed the ladder of success, then what?

Consider I Timothy 6:6-8 – ``Now godliness with contentment is great gain. For we brought nothing into this world, and it is certain we can carry nothing out. And having food and clothing, with these we shall be content.” What does this say? It all goes back in the box\footnote{The Idea for this comes from the book ``When the Game is Over: It All Goes Back in the Box" by John Ortberg}. And so, what really matters?

Isn't it better to imagine a world where there is no suffering from poverty? Imagine a world where no one is without food. Imagine a world where nothing spoils because we don’t let it. This could be the world if everyone chose to avoid the love of money, material possessions, and greed for power. Make the choice. Choose to not let yourselves fall prey to lusting after money but instead imagine the things that truly matter. My brethren, imagine The Kingdom of God. No, don’t imagine it, but make it a reality.

\begin{thebibliography}{9}
	{\footnotesize
	\bibitem{Suisse} (Credit Suisse Research Institute Global Wealth Report - Oct 2018
		
	\bibitem{Unicef} https://www.un.org/en/chronicle/article/losing-25000-hunger-every-day
	
	\bibitem{pollution} Joseph H. Bragdon, Is Pollution Profitable? – April 1972
	
	\bibitem{Paul Harvey} Harvey P., ``If I were the Devil.'' 1965
	
	\bibitem{TwoTreatises} http://www.yorku.ca/comninel/courses/3025pdf/Locke.pdf
	
	\bibitem{WealthofNations} https://www.ibiblio.org/ml/libri/s/SmithA\_WealthNations\_p.pdf
	}
\end{thebibliography}
\end{multicols}


%%%%%%%%%%%%%%%%%%%%%%%%%%%%%%%%%%%%%%%%%%%%%%%%%%%%%%%%%%%%%%%%%%%%%%%%%%%%%%%%%%%%%%%%%%%
\end{document}





















