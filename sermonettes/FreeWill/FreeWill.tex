\documentclass[11pt]{article}

%%% These are some packages that are useful
\usepackage{amsmath,amssymb, amscd,amsbsy, amsthm, enumerate}
\usepackage[export]{adjustbox}
\usepackage{lastpage}
\usepackage[top=1in, bottom=1in, left=1in, right=1in]{geometry}
\usepackage[unicode]{hyperref}
\usepackage{tikz, pgfplots, xcolor, fancyhdr}
\usepackage{multicol}
\usepackage{lipsum}

%%% Page formatting
%\setlength{\headsep}{30pt}
\setlength{\textheight}{9in}
\newcommand{\tab}{\hspace{1cm}}
%\setlength{\parindent}{25pt}

\title{Free Will}
\author{Antonius Torode}
\date{September 1, 2019}

%%% Header and Footer Info
\pagestyle{fancy}
\fancyhead[L]{{\large Template - \textbf{Change 03}}}
\fancyhead[C]{\today}
\fancyhead[R]{Name: Antonius Torode}


\fancyhf{} % sets both header and footer to nothing
\renewcommand{\headrulewidth}{0pt}
% your new footer definitions here

\fancyfoot[L]{}
\fancyfoot[C]{}
\fancyfoot[R]{\thepage\ of \pageref{LastPage}}

% Used to define spacing and format of References
\let\OLDthebibliography\thebibliography
\renewcommand\thebibliography[1]{
	\OLDthebibliography{#1}
	\setlength{\parskip}{0pt}
	\setlength{\itemsep}{0pt plus 0.3ex}
}

%%% Document Starts now
\begin{document}

\maketitle
\thispagestyle{fancy}

\begin{multicols}{2}
``People are inherently good.'' This is a quote that someone I know told me years ago. To put it in context, they had just left the church, and for some reason they were convinced that Godly defined morals were not needed in our world because people know how to be good and know how to live good lives without the Bible. Naturally we had an interesting little discussion and inevitably I gave up after this person started making some absurd statements such as ``suicide bombers aren't actually real,'' among other things. Some people who think people are inherently good also believe that people who do bad things are the way they are because of genetics. ``Oh, that behavior is genetic,'' has become a widespread notion that many people believe is valid. But are certain behaviors genetic? Or is it something else that determines how we behave? The genetic argument implies that there's an unchangeable rooted biological cause to bad behavior and there is thus no point in trying to change it \cite{It's genetic}. If behaviors were truly genetic, I suppose repentance wouldn't even be always possible and thus we would be able to throw out the Bible as truth. You can't repent if your genetics make you that way.

However, what does science actually suggest about this genetic argument? A wonderful example is demonstrated through a fancy technique which is used on mice to knock out specific genes from that mouse and its descendants \cite{yang}. There's one gene that codes for a protein that effects memory and learning. At Washington University, scientists knocked out this gene in mice and now they had mice that didn't learn as well. Thus, a genetic basis for intelligence \cite{dumb mice}. At the time this of course got talked about by the media left and right spreading the genetic argument. Although, there was another part to this study that was overlooked by most. When these genetically impaired mice were put in a more enriching environment than before, they completely overcame the deficit \cite{It's genetic}! The genes made no difference when the environmental impacts supported learning and memory, which contradicts the genetic argument. In reality, it's not genetics that significantly effect us and our behavior, but it's our environment \cite{It's genetic}. 

This statement is even supported biblically! You don't have to turn to these but some examples include 1 Corinthians 15:33 which tells us ``Do not be misled: Bad company corrupts good character.'' Similarly Proverbs 13:20 states ``Whoever walks with the wise becomes wise, but the companion of fools will suffer harm.'' These and other scriptures \footnote{Proverbs 22:24-2, 1 Corinthians 5:11, 2 Thessalonian 3:6, Proverbs 12:26, Titus 3:10, Proverbs 27:17, Deuteronomy 11:19, etc} support the idea that the environment we put ourselves in effects our beliefs and thoughts. In fact, this effect is so strong that it begins as soon as we have an environment, as a fetus. A great example can be seen through the Dutch Hunger winter. In 1944, Nazi's who were occupying Holland diverted all the food to Germany for three months. Tens of thousands of people starve to death, but if you were a second or third trimester fetus during this time, your body programmed itself to a world that is very limited on nutrients and not plentiful. A half of a century later, these children were shown to be more likely to have high blood pressure, obesity, or metabolic syndrome \cite{It's genetic part 2}. This environmental effect is why Advertising is such a large part of our lives. It has such a large effect that often companies spend more on advertisements than they do their actual products. It's also why people can be conditioned to do things like strictly follow military commands even to the point of becoming a suicide bomber!

In March of 1968, one of the largest forms of horror from the Vietnam war happened which is now known as the My Lai Massacre \cite{My Lai Massacre, Robert Sapolsky}. A brigade of American soldiers, "Charlie" company first battalion 20$^\textrm{th}$ infantry, was tasked with assisting in ``search and destroy'' operations on the Batangan Peninsula in Vietnam. The soldiers were tasked with destroying a place known as ``Pinkville'' along with all of the inhabitants. Intelligence suggested that Pinkville was full of traps and enemy soldiers. However, Pinkville turned out to be nothing more than an undefended village full of civilians. 

The American soldiers traveled to Pinkville and killed between 350 to 500 men, women, and children. The American soldiers not only killed the villagers, but gang raped the women beforehand, tore children away from their mothers and shot them, mutilated bodies, and many more nightmarish activities. Almost no one escaped this village. Those who ran were shot down by encircling troops. ``Inherently good people'' would not be capable of doing these types of actions. These actions are of those who have been conditioned to do such. Conditioned to follow their orders and act in kind. The military environment had such a large impact on these soldiers that they were able to perform such atrocities.

Why am I telling you about such a horrible event? Because I want to demonstrate that even though our environment can seemingly control us, it does not. If we look at Genesis 2:16-17, ``And the Lord God commanded the man, `You are free to eat from any tree in the garden; 17. but you must not eat from the tree of the knowledge of good and evil, for when you eat from it you will certainly die.''' We can see that from the beginning of time, God has given us a choice. A choice to do what our environment suggests, which would be to eat from the tree, or a choice to go against such. In Deuteronomy 30:19\footnote{Relavent scriptures include Deuteronomy 30:15-18;	15. See, I set before you today life and prosperity, death and destruction. 16. For I command you today to love the Lord your God, to walk in obedience to him, and to keep his commands, decrees and laws; then you will live and increase, and the Lord your God will bless you in the land you are entering to possess. 17. But if your heart turns away and you are not obedient, and if you are drawn away to bow down to other gods and worship them, 18 I declare to you this day that you will certainly be destroyed. You will not live long in the land you are crossing the Jordan to enter and possess.}, it says ``This day I call the heavens and the earth as witnesses against you that I have set before you life and death, blessings and curses. Now choose life, so that you and your children may live.'' God sets before us blessings and curses and he gives us the choice to choose between them. The free moral agency we have is a unique quality that often makes our choices contradict what our environment would lead us to do. So which do we choose?

The My Lai Massacre was stopped by one man. A man named Hugh Thompson. This man was piloting a helicopter and when he flew over Pinkville, he saw his fellow American soldiers among gunfire. He landed and got out only to find the incomprehensible sight of what these American soldiers were doing to the villagers. He got back into his helicopter and in the course of minutes undid every bit of training he had as to who's an us and who is a them. He landed his helicopter between the last group of surviving villagers and his American troops, turned his machine gun towards his own people and threatened to mow them all down if they did not stop \cite{Robert Sapolsky}! This man ignored the influences of his environment and made a free choice to do something against all of his training - to go against his environmental influences.

What this tells us is that no matter how strong the influences are around us, no matter how enticing our surroundings are, no matter how influential the advertisement is. Regardless of the programming we get fed every day by the world around us. We still have the free will to make our own choices, between blessings and curses, between life and death, we must choose life. We understand this concept of having free moral agency to be essential to understanding the plan of God, to understanding why there is suffering in the world, and much more. But how do we choose life? How do we choose against the influences of our environment? If we look at one final scripture, John 16:33, which states ``these things I have spoken unto you, that in me ye might have peace. In the world ye shall have tribulation: but be of good cheer; I have overcome the world.'' Christ is telling us that in the world and environment around us, we will have tribulations and struggles. However, He was able to overcome the world and through Him we also can do so. We have to use the example that Christ set to make our choices and in so doing discern between the many environmental influences. If we are aware of how the world around us effects us, we can consciously choose to enrich the environment around us. We can choose life, rather what a negative environment can lead to, which is death.



\begin{thebibliography}{9}
	{\footnotesize
	\bibitem{It's genetic} Robert Sapolsky. ``It's genetic'' Interview from Zeitgeist: Moving Forward. 2011. Movie. Part 1. https://www.youtube.com/watch?v=pQTMO3bQgpU	
	
	\bibitem{yang} Y. Yang, T. Yamada, K. K. Hill, M. Hemberg, N. C. Reddy, H. Y. Cho, A. N. Guthrie, A. Oldenborg, S. A. Heiney, S. Ohmae, J. F. Medina, T. E. Holy, A. Bonni. Chromatin remodeling inactivates activity genes and regulates neural coding. Science, 2016; 353 (6296): 300 DOI: 10.1126/science.aad4225
	
	\bibitem{dumb mice} https://medicine.wustl.edu/news/ability-turn-off-genes-brain-crucial-learning-memory/
		
	\bibitem{It's genetic part 2} Robert Sapolsky. ``It's genetic'' Interview from Zeitgeist: Moving Forward. 2011. Movie. Part 2. https://www.youtube.com/watch?v=G87aUGV5Mok
		
	\bibitem{My Lai Massacre} ``The My Lai Massacre | History". History Channel. Oct 13, 2017. https://www.youtube.com/watch?v=OnvTyMptOt8
	
	\bibitem{Robert Sapolsky} Sapolsky R. "Behave: The Biology of Humans at Our Best and Worst." Jul 18, 2017. Youtube.  https://youtu.be/k5rUwupQSQY?t=2050
	}
\end{thebibliography}

\end{multicols}


%%%%%%%%%%%%%%%%%%%%%%%%%%%%%%%%%%%%%%%%%%%%%%%%%%%%%%%%%%%%%%%%%%%%%%%%%%%%%%%%%%%%%%%%%%%
\end{document}





















