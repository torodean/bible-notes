\documentclass[10pt]{article}

%%% These are some packages that are useful
\usepackage{amsmath,amssymb, amscd,amsbsy, amsthm, enumerate}
\usepackage[export]{adjustbox}
\usepackage{lastpage}
\usepackage[top=1in, bottom=1in, left=1in, right=1in]{geometry}
\usepackage[unicode]{hyperref}
\usepackage{tikz, pgfplots, xcolor, fancyhdr}
\usepackage{multicol,caption}
\usepackage{lipsum}
\usepackage[version=4]{mhchem}
\usepackage{float}

%%% Page formatting
%\setlength{\headsep}{30pt}
\setlength{\textheight}{9in}
\newcommand{\tab}{\hspace{1cm}}
%\setlength{\parindent}{25pt}

\title{Mindset, Value, and Perspective (MVP)}
\author{Antonius Torode}

%%% Header and Footer Info
\pagestyle{fancy}
\fancyhead[L]{{\large Template - \textbf{Change 03}}}
\fancyhead[C]{\today}
\fancyhead[R]{Name: Antonius Torode}


\fancyhf{} % sets both header and footer to nothing
\renewcommand{\headrulewidth}{0pt}
% your new footer definitions here

\fancyfoot[L]{}
\fancyfoot[C]{}
\fancyfoot[R]{\thepage\ of \pageref{LastPage}}

% Used to define spacing and format of References
\let\OLDthebibliography\thebibliography
\renewcommand\thebibliography[1]{
	\OLDthebibliography{#1}
	\setlength{\parskip}{0pt}
	\setlength{\itemsep}{0pt plus 0.3ex}
}

\newenvironment{Figure}
  {\par\medskip\noindent\minipage{\linewidth}}
  {\endminipage\par\medskip}


%%% Document Starts now
\begin{document}

\maketitle
\thispagestyle{fancy}


%\begin{abstract}
%TODO - \lipsum[0-1]
%\end{abstract}

\begin{multicols}{2}

In this message, I'm going to very briefly touch on three related concepts and demonstrate how important and beneficial they are when put into practice in every day life; Mindset, Value, and Perspective (MVP for those who remember acronyms well). Our perspective determines how we interpret events. Our mindset governs how we respond to it. And together, those shape the value that comes out of our experiences. I'm going to tell a story in which these all apply, and I want you (the reader) to abstract and apply these to your own life.

\subsection*{Perspective}

During this past Feast\footnote{The Feast of Tabernacles was just about a week out at the time of writing this message.}, there was one day which left a memory a bit different than the others. Me and my Wife were staying with my little brother and my dad - it was the first time she had met them in person (since we live so far away). She was of course a tad concerned that maybe she wouldn't be able to stand them for that long; after all, many of us know how families can be. I don't think that was a problem, but nonetheless we made sure to schedule a date-night during the Feast where we could get away as just as us two.

I'm not very big on doing a lot of planning when it comes to short term things, so for the most part we were just going to wing it. I'll start with the end of the day and work backwards. This was a Friday going into the beginning of the Sabbath day. By the end of the day, it wasn't that late yet, but we ended up coming home pretty early. We both decided it had been a long day/Feast and we should just head back and relax and not do anything else. Just before heading home, we had went out to my wife's favorite restaurant that she had been talking about almost nonstop before we got to the area. We almost didn't even end up going and we ended up eating rather quickly while there. I had skipped dessert - even though I had previously decided that was the Feast day I was going to splurge and get one. The food was good, but overall we weren't there long. I even spent a while on my phone looking something up while there.

Just before the restaurant, we went to Home Depot. What better place to spend a date night than Home Depot? I bought myself some new tools and a few supplies and we spent a while out in the parking lot trying them out. Before Home Depot, we stopped at a car wash. I fairly recently got a new (used but still new to me) truck, and I've never washed it before. We ended up leaving because they only took change and I didn't actually have any on me. Shortly before that, we were back at our temporary dwelling (AirBnB) prepping for our date night that evening. My wife took a quick nap after church and I spent some time chatting with my little brother. An obvious question proceeds this story - What kind of a date night is that?!

\begin{quotation} % Perspective
``My brethren, count it all joy when you fall into various trials, knowing that the testing of your faith produces patience.'' - James 1:2-3
\end{quotation}

I intentionally left out one key detail of that story to demonstrate the importance of perspective. It should sound like that was a pretty \textit{dull} and \textit{peculiar} date-night. As we left to head out on our date night, we had originally planned to head straight to the restaurant as our first stop. It was a little over a half-hour drive. Around one of the long 35 mile-per-hour Tennessee curves, things went from normal to chaos in the blink of an eye. Glass was everywhere, my newly installed rear-view towing mirror was dangling by a wire, and the first thing out of my wife's mouth was ``Did you hit him or he hit you?'' The answer surprised her, ``It was a deer.'' A deer had launched itself directly into my driver side window - it's face only inches from mine for a split-second.



\subsection*{Mindset}

When this happened, my immediate reaction was to make sure everyone was okay. Then, my next reaction was to prioritize. My heart-rate was high, and I was a little shaky. I had to formulate a plan. My priorities were the following: clean up the glass, patch the window, and then get us to the restaurant. We were heading back home across the country in just a couple days. Now the story should make a bit more sense. We had to go to the car wash to vacuum out the glass - turns out we had no change. We then went to Home Depot to buy a portable vacuum and supplies to patch the window. Then we continued onto the restaurant. While there, I did some initial research to determine the cost of the repairs. My quick math came to something like \$5,000 if I took the car somewhere to get it fixed. I intentionally don't have comprehensive insurance - it's usually a scam.

\begin{quotation} % mindset
``For God has not given us a spirit of fear, but of power and of love and of a \textit{sound mind}.'' - 2 Timothy 1:7
\end{quotation}

A few years ago, I heard a podcasts from a retired U.S. Navy SEAL officer named Jocko Willink. He wrote a book title \textit{Extreme Ownership}. One idea he poses in the book a is that no matter what setback or problem arises, respond with ``Good.'' This \textit{is} the concept of keeping a sound mind. When you are sad, afraid, angry, or frustrated, you cannot think as clearly. This is usually easier said then done, but in this incident, I had this concept on my mind - to keep a \textit{sound mind} and find what is \textit{good}.

This evening was the start of the Sabbath. The next morning we had people coming over for breakfast, and more people coming over for dinner - our homemade sourdough pizzas. I could have let this incident ruin my entire sabbath day. Instead, I decided to leave it be until Sunday. I had already handled my immediate priorities of patching the window and getting the glass cleaned up. 

I could have also let this incident ruin my Sunday. However, I spent this day looking into my options. I took my truck apart to assess the damage. I fixed what I could, and I figured out what else I would need to fix it up. I was able to bend the door back into shape (mostly, but good enough for me) using my tire iron. I was able to fix the weather seal on the door. I had just enough pieces to re-build a side mirror (good enough to get us back home). My dad had suggested Safelight who wanted \$800 just to fix the window. I was able to find a glass company which would have a replacement window to pickup en route home, for only \$95 (after tax). Did I have the tools to replace a window? I happen to already have everything in my truck except one little hex bit which we picked up at Home Depot. The tow mirror fix is as simple as ordering a new one for \$150 - I'm the one who installed them in the first place. Overall, I was able to fix the door, the mirror, and the window myself for less money than a deductible would have been if I had insurance.

\subsection*{Value}

There are many valuable things I gained from this situation. I learned to install a powered window on my truck - I had never done that before. I put my financial decision to have no insurance to the test, and it payed off. I got a new tool. And while I struggled to think of a sermonette topic, one essentially leaped into my lap.

\begin{quotation} % value
``But the fruit of the Spirit is love, joy, peace, longsuffering, kindness, goodness, faithfulness, gentleness, self-control. Against such there is no law.'' - Galatians 5:22–23
\end{quotation}

The \textit{fruit} is a byproduct of our mindset. God has given us the spirit of a sound mind, so that the things we produce and learn are valuable to us and everyone around us. The value that came out of this incident were not just physical in nature. They were entirely because of my perspective about it. My mindset could have produced anguish, but instead it produced peace. I enjoyed the Sabbath as I would have any other week. I had patience to find the best solutions. I was joyful that this happened to me and not someone who would not have been prepared. And lastly, I was at peace about the whole situation. That is a level of value that outshines anything physical.

The takeaway is simple. Just remember the acronym MVP. Our mindset determines how we react to the world around us. Our perspective shapes how we interpret it. And together, those dispositions define the value we will get from the events we live.



\begin{thebibliography}{9}
	{\footnotesize
		
	\bibitem{Perspective} Perspective, L. Jim. Tuck, August 21, 2025. \url{https://www.ucg.org/sites/default/files/public/2025-08/Perspective%20082325.pdf}
	
	}
\end{thebibliography}

\end{multicols}


%%%%%%%%%%%%%%%%%%%%%%%%%%%%%%%%%%%%%%%%%%%%%%%%%%%%%%%%%%%%%%%%%%%%%%%%%%%%%%%%%%%%%%%%%%%
\end{document}





















}{den}