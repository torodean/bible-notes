\chapter{Preface}
\rule{\textwidth}{0.5pt}    % Line 
\thispagestyle{fancy}

This project is a personal undertaking to rewrite and annotate the Holy Bible for study, reflection, and understanding. The primary text used throughout is the \textit{New King James Version (NKJV)}, though other translations may occasionally be referenced where clarity, variation, or historical interest requires. Each book of the Bible is treated as a separate chapter, and each begins with a brief summary outlining its themes or narrative structure.

\tab The scripture is presented verse by verse using a custom format that allows for inline references and annotations. These annotations include (but are not limited to) historical background, translation notes, literary and cultural context, scientific references, theological or reflective questions, and general comments. Each of these various types are also color-coded for ease of reference/reading. They are marked inline with a custom symbol (e.g., \vmark{$\star$}) and collected directly below the verse for easy association.

\tab This is an ongoing and evolving project. At the time of writing this preface, it is highly incomplete and is being developed incrementally. As updates, corrections, and expansions are made, the version number of the document will be incremented accordingly. This allows for tracking progress and identifying updated material across future iterations. This work is intended for personal study and is freely shareable for non-commercial purposes.

\rule{\textwidth}{0.5pt}    % Line 


\subsection*{Various Subtle Formatting Meanings}

There are various subtle formatting techniques which have specific meanings, These are outlined below:
\begin{itemize}
	\item As done in the NKJV, scriptural words which are \textit{italicized} are `added', but are not meant to change any meaning - only to provide better understanding with our modern English language. Within the commentary and notes, \textit{italics} are instead used as they typically are in standard English writings - for emphesis or proper reference formatting.
	\item Scripture numbers are in \textbf{bold} for ease of finding.
	\item Words that Jesus spoke are \jesus{quoted and in red.}
\end{itemize}




\subsection*{Note Types and Symbols}

The following is a list of annotation types used throughout this work. Each note is marked with a specific icon and color to indicate its category and purpose:\medskip

\historicalnote{History}{Provides historical or cultural background relevant to the verse or passage, such as ancient customs, geopolitical context, or historical events.}\medskip

\translationnote{Translation}{Offers insight into the original language, alternative translations, or meanings of words and phrases from the Hebrew, Aramaic, or Greek texts.}\medskip

\contextnote{Context}{Highlights the literary, narrative, or cultural context of a passage, aiding in its interpretation and situational relevance.}\medskip

\sciencenote{Science}{Connects the passage to scientific ideas, natural phenomena, or the relationship between scriptural content and modern science.}\medskip

\questionnote{Question}{Poses theological, philosophical, or reflective questions raised by the verse, encouraging deeper thought and study.}\medskip

\sermonnote{Sermon}{Presents insights, illustrations, or applications drawn from sermons. These notes may include quotes, thematic connections, or interpretations offered by preachers, providing a homiletic perspective that complements the text.}\medskip

\crossrefnote{Cross-reference}{Identifies connections to other passages of Scripture, highlighting thematic, prophetic, or textual parallels that provide interpretive insight or support broader biblical understanding.}\medskip

\geographynote{Geography}{Provides geographical information related to the passage, such as locations, travel routes, terrain features, and regional significance. This can aid in contextualizing the narrative within its physical setting.}\medskip

\doctrinenote{Doctrine}{Highlights theological principles, doctrines, or key insights derived from the passage. These notes may address topics such as the nature of God, salvation, sin, covenant, or other foundational beliefs, helping readers engage with the text theologically.}\medskip

\literarynote{Literary}{Draws attention to the literary qualities of the text, such as poetic structure, metaphor, genre, parallelism, or symbolic language. These notes help identify stylistic features that shape interpretation and highlight the artistry of the biblical text.}\medskip

\generalnote{Note}{A general-purpose note for observations, clarifications, or commentary that doesn't fall into one of the above categories.}\medskip
