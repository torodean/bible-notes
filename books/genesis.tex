\booktitle{Genesis}{}
\thispagestyle{fancy}





% Chapter 1
\bookchapter{The History of Creation}

\bverse{In the beginning God created the heavens and the earth.\vmark{a}}
\sciencenote{a}{``In the beginning'' implies that there was a beginning. ``The heavens'' could refer to the entire cosmos or even space itself. ``The earth'' could refer to the entirety of matter within the universe or simply bring the focus to earth (while not explicitly excluding anything outside of earth).}

\bverse{The earth was\vmark{a} without form, and void; and darkness \textit{was} on the face of the deep. And the Spirit of God was hovering over the face of the waters\vmark{b}.}
\translationnote{a}{The word for ``was'' (Strong's H1961) can be translated as \textit{became} or \textit{come to pass}.}
\sciencenote{a}{If translated as \textit{became}, this could imply that earth was something before. This suggests that there could have been a passage of time before creation and the following events. It could either then mean that there was a possible history of Earth before the following creation events, or it could simply be that after the creation mentioned in \bref{Genesis 1:1}, earth was in this state. This idea supports a pre-existing physical earth before the creation described in the following scriptures. This is consistent with modern scientific understanding of the age of the earth.}
\contextnote{b}{``Hovering'' establishes an Earth-based perspective, with the narrative describing creation from Earth's surface, focusing on transforming chaos to order.}

\bverse{Then God said, ``Let there be light\vmark{a}"; and there was light.}
\sciencenote{a}{Diffuse sunlight through a thick early atmosphere could create a day-night cycle before celestial bodies are visible.}


\bverse{And God saw the light, that \textit{it was} good; and God divided the light from the darkness.}
\bverse{God called the light Day, and the darkness He called Night. So the evening\vmark{a} and the morning\vmark{b} were the first day.}
\translationnote{a}{The word for ``evening'' (Strongs H6153) is also translated \textit{sunset}.}
\translationnote{b}{The word for ``morning'' (Strongs H1242) means sunrise or the end of night (start of the day).}
\generalmnote{b}{The ``evening and the morning'' are established as the time measurement for a \textit{day}. This is a foundational basis for time keeping throughout the Bible that is important to remain consistent on. The time period for a day begins at \textit{sunset}, and ends the following \textit{sunset} (encompassing an evening and a morning).}

\bverse{Then God said, ``Let there be a firmament in the midst of the waters, and let it divide the waters from the waters.''}

\bverse{Thus God made the firmament, and divided the waters which \textit{were} under the firmament from the waters which \textit{were} above the firmament; and it was so.}

\bverse{And God called the firmament Heaven, so the evening and the morning were the second day.}

\bverse{Then God said, ``Let the waters under the heavens be gathered together into one place, and let the dry \textit{land} appear''; and it was so.}

\bverse{And God called the dry \textit{land} Earth, and the gathering together of the waters He called Seas. And God saw that \textit{it was} good.}

\bverse{Then God said, ``Let the earth bring forth grass, that herb \textit{that} yields seed, \textit{and} the fruit tree \textit{that} yields fruit according to its kind, whos seed \textit{is} in itself, on the earth''; and it was so.}

\bverse{And the earth brought forth grass, the herb \textit{that} yields seed according to its kind, and the tree \textit{that} yields fruit, whos seed \textit{is} in itself according to its kind. And God saw that \textit{it was} good.}

\bverse{So the evening and the morning were the third day.}