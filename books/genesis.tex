\booktitle{Genesis}{}
\thispagestyle{fancy}





% Chapter 1
\bookchapter{The History of Creation}

\bverse{In the beginning God created the heavens and the earth.\vmark{a}}
	\sciencenote{a}{``In the beginning'' implies that there was a beginning. ``The heavens'' could refer to the entire cosmos or even space itself. ``The earth'' could refer to the entirety of matter within the universe or simply bring the focus to earth (while not explicitly excluding anything outside of earth).}

\bverse{The earth was\vmark{a} without form, and void; and darkness \textit{was} on the face of the deep. And the Spirit of God was hovering over the face of the waters\vmark{b}.}
	\translationnote{a}{The word for ``was'' (Strong's H1961\cite{Strong's GodRules}) can be translated as \textit{became} or \textit{come to pass}.}
	\sciencenote{a}{If translated as \textit{became}, this could imply that earth was something before. This suggests that there could have been a passage of time before creation and the following events. It could either then mean that there was a possible history of Earth before the following creation events, or it could simply be that after the creation mentioned in \bref{Genesis 1:1}, earth was in this state. This idea supports a pre-existing physical earth before the creation described in the following scriptures. This is consistent with modern scientific understanding of the age of the earth.}
	\contextnote{b}{``Hovering'' establishes an Earth-based perspective, with the narrative describing creation from Earth's surface, focusing on transforming chaos to order.}

\bverse{Then God said, ``Let there be light\vmark{a}"; and there was light.}
\sciencenote{a}{Diffuse sunlight through a thick early atmosphere could create a day-night cycle before celestial bodies are visible.}


\bverse{And God saw the light, that \textit{it was} good; and God divided the light from the darkness.}
\bverse{God called the light Day, and the darkness He called Night. So the evening\vmark{a} and the morning\vmark{b} were the first day.\vmark{c}}
	\translationnote{a}{The word for ``evening'' (Strongs H6153\cite{Strong's GodRules}) is also translated \textit{sunset}.}
	\translationnote{b}{The word for ``morning'' (Strongs H1242\cite{Strong's GodRules}) means sunrise or the end of night (start of the day).}
	\generalmnote{b}{The ``evening and the morning'' are established as the time measurement for a \textit{day}. This is a foundational basis for time keeping throughout the Bible that is important to remain consistent on. The time period for a day begins at \textit{sunset}, and ends the following \textit{sunset} (encompassing an evening and a morning).}
	
\bverse{Then God said, ``Let there be a firmament\vmark{a} in the midst of the waters, and let it divide the waters from the waters.''}
	\translationnote{a}{The word for ``firmament'' (Strongs H7549\cite{Strong's GodRules}) can sometimes be interpreted as a solid (This is one scripture that is used to suggest the earth has a solid dome around it), but more appropriately means an expanse, the heavens or the sky.}
	\sciencenote{a}{This corresponds to a stable atmosphere forming, separating surface waters from vapor and creating an atmosphere for the earth.}
	

\bverse{Thus God made the firmament, and divided the waters which \textit{were} under the firmament\vmark{a} from the waters which \textit{were} above the firmament\vmark{b}; and it was so.}
	\generalmnote{a}{These would be the water on the ground - seas, puddles, ponds, rivers, etc.}
	\generalmnote{b}{This would be the water in the sky - clouds, vapor, etc.}
	
\bverse{And God called the firmament Heaven\vmark{a}, so the evening and the morning were the second day.}
	\translationnote{a}{There are multiple different words commonly translated to ``heaven''. In this context (Strongs H8064\cite{Strong's GodRules}), the word simply means the sky.}

\bverse{Then God said, ``Let the waters under the heavens be gathered together into one place, and let the dry \textit{land} appear''; and it was so.}	
\bverse{And God called the dry \textit{land} Earth, and the gathering together of the waters He called Seas. And God saw that \textit{it was} good.}
\bverse{Then God said, ``Let the earth bring forth grass, the herb \textit{that} yields seed, \textit{and} the fruit tree \textit{that} yields fruit according to its kind, whos seed \textit{is} in itself, on the earth''; and it was so.}
	\sciencenote{}{Early vegetation (e.g., algae) could grow easily with diffuse light from day 1, before the sun’s visibility. Given perpetual cloud coverage, plants could also still flourish if enough diffuse light made it through the cloud coverage. Even a thick hazy atmosphere which would obstruct the view of stars and the sun could still allow enough photons through for photosynthesis to occur (Many plants in rain forests thrive in these conditions while shaded by other trees).}
	\contextnote{h}{Plants before the sun’s visibility (day 4) fits a sequence from earth's perspective with a potential hazy atmosphere not yet revealing celestial bodies.}

\bverse{And the earth brought forth grass, the herb \textit{that} yields seed according to its kind\vmark{a}, and the tree \textit{that} yields fruit, whos seed \textit{is} in itself according to its kind. And God saw that \textit{it was} good.}
	\sciencenote{a}{The phrase ``according to its kind'' is of great importance. This states that fruit will always create seeds \textit{according to its kind}, and not of another kind. This has always been observed to be true, and even in cases of cross-breading - the fruit yields seed according to its kind, and not of other kinds.}

\bverse{So the evening and the morning were the third day.}
\bverse{Then God said, ``Let there be lights in the firmament of the heavens to divide the day from the night; and let them be for signs and seasons, and for days and years;}
\bverse{and let them be for lights in the firmaments of the heavens to give light on the earth''; and it was so.}