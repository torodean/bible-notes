% Chapter 2 of 1 John\index{John}
\bookchapter{The Test of Knowing Christ}

\bverse My little children, these things I write to you, so that you may not sin\vmark{a}. And if anyone sins, we have an Advocate with the Father, Jesus Christ the righteous\vmark{b}.
	\doctrinenote{a}{One of the main purposes of this letter (as stated here) is so that we understand that we should \textit{not} sin. If sinning was okay, this would not really serve a purpose. It further clarifies that the law \textit{has to} be physically followed and our actions matter. If they did not, there would be no reason to not sin.}
	\doctrinenote{b}{Jesus is our advocate with the Father. When we inevitably sin, because we all do, He is the one who we have to go to in order to reconcile what we've done. This is pivotal, as explained in \bref{1 John 5:7}.}

\bverse And He Himself is the propitiation for our sins, and not for ours only but also for the whole world.

\bmarkerdown{The Test of Knowing Him}

\bverse Now by this we know that we know Him, if we keep His commandments.
	\doctrinenote{}{This again reiterates the actions of \textit{doing} the commandments in order to know Him. If we do not do this, we have no ability to understand who Jesus Christ really was.}

\bverse He who says, ``I know Him,'' and does not keep His commandments, is a liar, and the truth is not in him.
	\doctrinenote{}{It is therefore impossible to know Jesus Christ without keeping his commandments.}

\bverse But whoever keeps His word, truly the love of God is perfected in him. By this we know that we are in Him.
\bverse He who says he abides in Him ought himself also walk just as He walked.
	\contextnote{}{Jesus kept all of the commandments. This is how he \textit{walked}. This is directly referring to His actions and how He lived His life.}

\bverse Brethren, I write no new commandment to you, but an old commandment which you have had from the beginning\vmark{a}. The old commandment is the word which you heard from the beginning.
	\contextnote{a}{These things that are being said are \textit{not} new. These are the same commandments that have always existed and no different. Anything that seems different is simply a misrepresentation of the original commandments which have always existed (from the beginning).}

\bverse Again, a new commandment I write to you\vmark{a}, which thing is true in Him and you, because the darkness is passing away, and the true light is already shining.
	\contextnote{a}{This is a little confusing. At first he says he is not giving any new commandment, but an old commandment which they have had from the beginning. Now he says that he is giving a new commandment. The commandment here is not new in the sense that it has always existed. The \textit{new} part comes from the new realization, depth, and meaning that this commandment is presented in through Jesus Christ's teaching. In a sense, it's saying that we've had this commandment all along, but we didn't really understand if fully - so it seems new even though we've had it all along.}

\bverse 
\bverse 
\bverse 
\bverse 
\bverse 
\bverse 
\bverse 
\bverse 
\bverse 
\bverse 
\bverse 
\bverse 
\bverse 
\bverse 
\bverse 
\bverse 
\bverse 
\bverse 
\bverse 
\bverse 
\bverse 
