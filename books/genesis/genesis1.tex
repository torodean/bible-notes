
% Chapter 1
\bookchapter{The History of Creation}

\bverse{In the beginning God created the heavens\index{Heaven} and the earth\index{Earth}.\vmark{a}}
	\sciencenote{a}{``In the beginning'' implies that there was a beginning. ``The heavens\index{Heaven}'' could refer to the entire cosmos or even space itself. ``The earth\index{Earth}'' could refer to the entirety of matter within the universe or simply bring the focus to earth\index{Earth} (while not explicitly excluding anything outside of earth\index{Earth}).}




\bmarkerdown{Beginning of day one}

\bverse{The earth\index{Earth} was\vmark{a} without form, and void; and darkness \textit{was} on the face of the deep. And the Spirit of God was hovering over the face of the waters\index{Water}\vmark{b}.}
	\translationnote{a}{The word for ``was'' (Strong's 1961\cite{Strong's GodRules}) can be translated as \textit{became} or \textit{come to pass}.}
	\sciencenote{a}{If translated as \textit{became}, this could imply that earth\index{Earth} was something before. This suggests that there could have been a passage of time before creation and the following events. It could either then mean that there was a possible history of Earth\index{Earth} before the following creation events, or it could simply be that after the creation mentioned in \bref{Genesis 1:1}, earth\index{Earth} was in this state. This idea supports a pre-existing physical earth\index{Earth} before the creation described in the following scriptures. This is consistent with modern scientific understanding of the age of the earth\index{Earth}.}
	\contextnote{b}{``Hovering'' establishes an Earth\index{Earth}-based perspective, with the narrative describing creation from Earth\index{Earth}'s surface, focusing on transforming chaos to order.}

\bverse{Then God said, ``Let there be light\index{Light}\vmark{a}"; and there was light\index{Light}.}
\sciencenote{a}{Diffuse sunlight through a thick early atmosphere could create a day-night cycle before celestial bodies are visible.}

\bverse{And God saw the light\index{Light}, that \textit{it was} good; and God divided the light\index{Light} from the darkness.}
\bverse{God called the light\index{Light} Day, and the darkness He called Night. So the evening\vmark{a} and the morning\vmark{b} were the first day.\vmark{c}}
	\translationnote{a}{The word for ``evening'' (Strong's 6153\cite{Strong's GodRules}) is also translated \textit{sunset}.}
	\translationnote{b}{The word for ``morning'' (Strong's 1242\cite{Strong's GodRules}) means \textit{sunrise} or the end of night (start of the day).}
	\generalnote{b}{The ``evening and the morning'' are established as the time measurement for a \textit{day}. This is a foundational basis for time keeping throughout the Bible that is important to remain consistent on. The time period for a day begins at \textit{sunset}, and ends the following \textit{sunset} (encompassing an evening and a morning).}



\bmarkerdown{Beginning of day two}

\bverse{Then God said, ``Let there be a firmament\vmark{a} in the midst of the waters\index{Water}, and let it divide the waters\index{Water} from the waters\index{Water}.''}
	\translationnote{a}{The word for ``firmament'' (Strong's 7549\cite{Strong's GodRules}) can sometimes be interpreted as a solid (This is one scripture that is used to suggest the earth\index{Earth} has a solid dome around it), but more appropriately means an expanse, the heavens\index{Heaven} or the sky.}
	\sciencenote{a}{This corresponds to a stable atmosphere forming, separating surface waters\index{Water} from vapor and creating an atmosphere for the earth\index{Earth}.}
	
\bverse{Thus God made the firmament, and divided the waters\index{Water} which \textit{were} under the firmament\vmark{a} from the waters\index{Water} which \textit{were} above the firmament\vmark{b}; and it was so.}
	\generalnote{a}{These would be the water\index{Water} on the ground - seas, puddles, ponds, rivers, etc.}
	\generalnote{b}{This would be the water\index{Water} in the sky - clouds, vapor, etc.}
	
\bverse{And God called the firmament Heaven\index{Heaven}\vmark{a}, so the evening and the morning were the second day.}
	\translationnote{a}{There are multiple different words commonly translated to ``heaven\index{Heaven}''. In this context (Strong's 8064\cite{Strong's GodRules}), the word simply means the sky.}



\bmarkerdown{Beginning of day three}

\bverse{Then God said, ``Let the waters\index{Water} under the heavens\index{Heaven} be gathered together into one place, and let the dry \textit{land} appear''; and it was so.}	
\bverse{And God called the dry \textit{land} Earth\index{Earth}, and the gathering together of the waters\index{Water} He called Seas. And God saw that \textit{it was} good.}
\bverse{Then God said, ``Let the earth\index{Earth} bring forth grass, the herb \textit{that} yields seed, \textit{and} the fruit tree \textit{that} yields fruit according to its kind, whos seed \textit{is} in itself, on the earth\index{Earth}''; and it was so.}
	\sciencenote{}{Early vegetation (e.g., algae) could grow easily with diffuse light\index{Light} from day 1, before the sun’s visibility. Given perpetual cloud coverage, plants could also still flourish if enough diffuse light\index{Light} made it through the cloud coverage. Even a thick hazy atmosphere which would obstruct the view of stars and the sun could still allow enough photons through for photosynthesis to occur (Many plants in rain forests thrive in these conditions while shaded by other trees).}
	\contextnote{h}{Plants before the sun’s visibility (day 4) fits a sequence from earth\index{Earth}'s perspective with a potential hazy atmosphere not yet revealing celestial bodies.}

\bverse{And the earth\index{Earth} brought forth grass\vmark{a}, the herb \textit{that} yields seed according to its kind\vmark{b}, and the tree \textit{that} yields fruit, whos seed \textit{is} in itself according to its kind. And God saw that \textit{it was} good.}
	\sciencenote{a}{If the atmosphere was thick hazy (at this point) such that light\index{Light} could not easily get through, the plants would naturally clear this, by filtering the air, allowing a natural progression from this to the next day's events of the atmosphere clearing.}
	\sciencenote{b}{The phrase ``according to its kind'' is of great importance. This states that fruit will always create seeds \textit{according to its kind}, and not of another kind. This has always been observed to be true, and even in cases of cross\index{Cross}-breading - the fruit yields seed according to its kind, and not of other kinds.}

\bverse{So the evening and the morning were the third day.}



\bmarkerdown{Beginning of day four}

\bverse{Then God said, ``Let there be lights\index{Light} in the firmament\vmark{a} of the heavens\index{Heaven} to divide the day from the night; and let them be for signs and seasons\vmark{b}, and for days and years;}
	\sciencenote{a}{As the atmosphere clears, the stars, sunlight, and moonlight become visible through the atmosphere, which creates the lights\index{Light} in the sky.}
	\generalnote{b}{The ``signs and seasons'' establishes a time-keeping system for the Biblical Holy Days.}
	
\bverse{and let them be for lights\index{Light} in the firmaments of the heavens\index{Heaven} to give light\index{Light} on the earth\index{Earth}''; and it was so.}
\bverse{Then God made two great lights\index{Light}\vmark{a}: the greater light\index{Light} to rule the day, and the lesser light\index{Light} to rule the night. He \textit{made} the stars also\vmark{b}.}
	\sciencenote{a}{This is the same thought/day as \bref{Genesis 1:14}, where the stars become visible through the atmosphere. From the perspective of earth\index{Earth}, as the atmosphere clear, the sun and moon become visible and appears as though they are new creations.}
	\sciencenote{b}{If we consider the sequence of events here in correlation to modern day understanding (which would of course be very limited compared to God), this sequence of earth\index{Earth} being created in six days suggests that either the entire universe was created in this sequence to demonstrate Gods power and capabilities, or that the rest of the creation (outside of earth\index{Earth}) was already created (which perfectly correlates to the \textit{gap theory} mentioned in \bref{Genesis 1:2}). This would also align with modern theories of the age of the universe and other timelines.}
	\questionnote{b}{If the \textit{gap theory} is not close to accurate, why would God have taken six days to create earth\index{Earth}, when he was able to create the rest of the universe (``the stars also'') in less than a day? This could either be explained by the above note, or by God outlining the importance of earth\index{Earth} by timing these events in this way. But as mentioned in previous creation events, things would be more consistent if this latter thought was not the case.}
	
\bverse{God set them in the firmament of the heavens\index{Heaven} to give light\index{Light} on the earth\index{Earth},}
\bverse{and to rule over the day and over the night, and to divide the light\index{Light} from the darkness. And God saw that \textit{it was} good.}
\bverse{So the evening and the morning were the fourth day.}



\bmarkerdown{Beginning of day five}

\bverse{Then God said, ``Let the waters\index{Water} abound with an abundance of living creatures, and let the birds fly above the earth\index{Earth} across the face of the firmament of the heavens\index{Heaven}.''}
\bverse{So God created great sea creatures and every living thing that moves, with which the waters\index{Water} abounded, according to their kind, and every winged bird according to its kind. And God saw that \textit{it was} good.}
\bverse{And God blessed them, saying, ``Be fruitful and multiply, and fill the waters\index{Water} in the seas, and let birds multiply on  the earth\index{Earth}.''}
\bverse{So the evening and the morning were the fifth day.}



\bmarkerdown{Beginning of day six}

\bverse{Then God said, ``Let the earth\index{Earth} bring forth the living creature according to its kind: cattle and creeping thing and beast of the earth\index{Earth}, \textit{each} according to its kind\vmark{a}''; and it was so.}
	\sciencenote{a}{The ``according to its kind,'' distinction is of great importance. What is seen in nature is that each animal only ever gives birth or creates offspring of something which is according to its kind. This is similar to the idea of species, except not quite as constricted.}
	
	
\bverse{And God made the beast of the earth\index{Earth} according to its kind, cattle according to its kind, and everything that creeps on the earth\index{Earth} according to its kind. And God saw that \textit{it was} good.}
	\sciencenote{}{It's possible to interpret this in some interesting ways based on the current day understanding of the ecosystem and interdependence of all the plants, animals, bugs, etc. Since many things today are shown to be interdependent (such as the plants requiring insects for pollination or animals requiring plants for food), the literal six-day period of earths\index{Earth} creation would make the most sense because of both the order that things appear, and the dependence of the various things that are coming into existence. The short time period would be required by the interdependence of the various life forms, whereas the order they appear in would make sense from an \textit{evolution} perspective. There are many fields of study that attempt to show that various organisms are similar to each-other in a way that connects them. However, It would make sense that God would create new things using elements from the previous things he made - or have his creations build off of others (as any computer programmer would do for example).}

\bverse{Then God said, ``Let Us\vmark{a} make man in Our\vmark{b} image, according to Our likeness; let them have dominion over the fish of the sea, over the birds of the air, and over the cattle, over all the earth\index{Earth} and over every creeping thing that creeps on the earth\index{Earth}.''}
	\translationnote{a}{The word ``Us'' here is translated into a \textit{plural} English word. this supports there being at least two beings in the God family as seen in \bref{John 1:1}.}
	\translationnote{b}{The word ``Our'' here is translated into a \textit{plural} English word.}

\bverse{So God created man in His \textit{own} image; in the image of God He created him; male and female He created them.}
\bverse{Then God blessed them, and God said to them, ``Be fruitful and multiply; fill the earth\index{Earth} and subdue it; have dominion over the fish of the sea, over the birds of the air, and over every living thing that moves on the earth\index{Earth}.''}
\bverse{And God said, ``See, I have given you every herb \textit{that} yields seed which \textit{is} on the face of all the earth\index{Earth}, and every tree whos fruit yields seed; to you it shall be for food.}
	\sermonnote{}{There's an apparent contradictory claim here against some modern (mis)understandings. Many seeds contain cyanide or cyanide producing compounds, which suggest they are \textit{not} made for food. I wrote a sermonette message about this titled \textit{Apple Seeds}\cite{Antonius' Sermonettes}, where I demonstrate that these compounds are actually perfectly safe and the human body is designed explicitly to break them down in normal quantities.}

\bverse{Also, to every beast of the earth\index{Earth}, to every bird of the air, and to everything that creeps on the earth\index{Earth}, in which \textit{there is} life, \textit{I have given} every green herb for food; and it was so.}
\bverse{Then God saw everything that He had made, and indeed \textit{it was} very good. So the evening and the morning were the sixth day.}





















