% Chapter 2 of Genesis
\bookchapter{The Garden of Eden\index{Eden}}


\bverse{Thus the heavens\index{Heaven} and the earth\index{Earth}, and all the host of them, were finished.}
	\generalnote{}{This scripture, which directly follows the concluding creation of earth\index{Earth} and man, could mean that these described creation events were the last event needed for God to have fulfilled what he was making. If a \textit{gap-theory} is correct, this implies that the earth\index{Earth} was the final stone to place. If not, then it could have just been that this simply concludes the creation.}

\bverse{And on the seventh day God ended His work which He had done, and He rested on the seventh day from all His work which He had done.}
\bverse{Then God blessed the seventh day and sanctified it, because in it He rested from all His work which God had created and made.}
		\contextnote{}{This establishes the seventh day as a sanctified day of rest for the first time - as that is what God himself did.}
		
\bverse{This \textit{is} the history of the heavens\index{Heaven} and the earth\index{Earth} when they were created, in the day that the \lordvmark{a} God made the earth\index{Earth} and the heavens\index{Heaven},}
	\translationnote{a}{This is the \textit{Tetragrammaton}\index{Tetragrammaton}, which is the four-letter Hebrew\index{Hebrew} name of God, \textit{YHWH}. This is the first time it appears.}
	
\bverse{before any plant of the field was in the earth\index{Earth} and before any herb of the field had grown. For the Lord God had not caused it to rain on the earth\index{Earth}, and \textit{there was} no man to till the ground;}
\bverse{but a mist went up from the earth\index{Earth} and watered the whole face of the ground.}
\bverse{And the \lord God formed man \textit{of} the dust of the ground, and breathed into his nostrils the breath of life; and man became a living being.}
\bverse{The \lord God planted a garden eastward in Eden\index{Eden}, and there He put the man whom He had formed.}
\bverse{And out of the ground the \lord God made every tree grow that is pleasant to the sight and good for food. The tree of life \textit{was} also in the midst of the garden, and the tree of the knowledge of good and evil.}
\bverse{Now a river went out of Eden\index{Eden} to water\index{Water} the garden, and from there it parted and became four riverheads.}
\bverse{The name of the first \textit{is} Pishon\index{Pishon}\vmark{a}; it \textit{is} the one which skirts the whole land of Havilah\index{Havilah}, where \textit{there is} gold\index{Gold}.}
	\translationnote{a}{The river ``Pishon\index{Pishon}'' (Strong's 6376\cite{Hebrew Concordance}) is a Hebrew\index{Hebrew} proper noun. This term only occurs in the Bible once.}

\bverse{And the gold\index{Gold} of that land \textit{is} good. Bdellium\index{Bdellium}\vmark{a} and the onyx\index{Onyx}\vmark{b} stone \textit{are} there.}
	\translationnote{a}{The word ``Bdellium\index{Bdellium}'' (Strong's 916\cite{Hebrew Concordance}) is a fragrant resin similar to myrrh\index{Myrrh}.}
	\translationnote{b}{The word ``onyx\index{Onyx}'' (Strong's 7718\cite{Hebrew Concordance}) is commonly translated as \textit{onyx}\index{Onyx}, but more accurately represents a precious gem or stone (\textit{perhaps onyx}\index{Onyx}). It is often associated with beauty, value, and it was used in the high priest's breastplate and other sacred objects.}

\bverse{The name of the second river \textit{is} Gihon\index{Gihon}\vmark{a}; it \textit{is} the one which goes around the whole land of Cush.}
	\translationnote{a}{The name ``Gihon\index{Gihon}'' (Strong's 1521\cite{Hebrew Concordance}) suggests a river that bursts forth or flows abundantly.}
	
\bverse{The name of the third river \textit{is} Hiddekel\index{Hiddekel}\vmark{a}; it \textit{is} the one which goes toward the east of Assyria. The fourth river \textit{is} the Euphrates\index{Euphrates}.}
	\translationnote{a}{The name ``Hiddekel\index{Hiddekel}'' (Strong's 2313\cite{Hebrew Concordance}) is often identified as the modern day Tigris\index{Tigris} River, which flows through present-day Turkey and Iraq. \textit{Hiddekel}\index{Hiddekel} is an ancient Mesopotamian river name.}
	\historicalnote{a}{``The Tigris\index{Tigris} River, along with the Euphrates\index{Euphrates}, is one of the two major rivers of Mesopotamia, a region often referred to as the "Cradle of Civilization." This area is historically significant as it is believed to be one of the earliest centers of human civilization, with ancient cities such as Nineveh\index{Nineveh} and Babylon\index{Babylon} located along its banks. The Tigris\index{Tigris} has been a vital water\index{Water} source for agriculture and trade throughout history.\cite{Hebrew Concordance}''}

\bverse{}
\bverse{}
\bverse{}
\bverse{}
\bverse{}
\bverse{}
\bverse{}
\bverse{}
\bverse{}
\bverse{}
\bverse{}
