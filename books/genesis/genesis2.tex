% Chapter 2 of Genesis
\bookchapter{Life in The Garden of Eden\index{Eden}}


\bverse Thus the heavens\index{Heaven} and the earth\index{Earth}, and all the host of them, were finished.
	\contextnote{}{This scripture, which directly follows the concluding creation of earth\index{Earth} and man, could mean that these described creation events were the last event needed for God to have fulfilled what he was making. If a \textit{gap-theory} is correct, this implies that the earth\index{Earth} was the final stone to place. If not, then it could have just been that this simply concludes the creation.}

\bverse And on the seventh day God ended His work which He had done, and He rested on the seventh day from all His work which He had done.
\bverse Then God blessed the seventh day and sanctified it, because in it He rested from all His work which God had created and made.
		\contextnote{}{This establishes the seventh day as a sanctified day of rest for the first time - as that is what God himself did.}
		
\bverse This \textit{is} the history of the heavens\index{Heaven} and the earth\index{Earth} when they were created, in the day that the \lordvmark{a} God made the earth\index{Earth} and the heavens\index{Heaven},
	\translationnote{a}{This is the \textit{Tetragrammaton}\index{Tetragrammaton}, which is the four-letter Hebrew\index{Hebrew} name of God, \textit{YHWH}. This is the first time it appears.}
	
\bverse before any plant of the field was in the earth\index{Earth} and before any herb of the field had grown. For the Lord God had not caused it to rain on the earth\index{Earth}, and \textit{there was} no man to till the ground;
\bverse but a mist went up from the earth\index{Earth} and watered the whole face of the ground.
\bverse And the \lord God formed\vmark{a} man \textit{of} the dust of the ground, and breathed into his nostrils the breath\index{Breath} of life; and man became a living being.
	\contextnote{a}{The word for ``formed'' here implies personal involvement.}

\bverse The \lord God planted a garden eastward in Eden\index{Eden}, and there He put the man whom He had formed.
\bverse And out of the ground the \lord God made every tree\index{Tree} grow that is pleasant to the sight and good for food. The tree\index{Tree} of life\vmark{a} \was also in the midst of the garden, and the tree\index{Tree} of the knowledge of good and evil.
	\crossrefnote{a}{Here we see that the \textit{tree of life} is \textit{separate} from the \textit{tree of the knowledge of good and evil}. This is referenced again in \bref{Genesis 3:22}.}

\bverse Now a river went out of Eden\index{Eden} to water\index{Water} the garden, and from there it parted and became four riverheads.
\bverse The name of the first \textit{is} Pishon\index{Pishon}\vmark{a}; it \textit{is} the one which skirts the whole land of Havilah\index{Havilah}, where \textit{there is} gold\index{Gold}.
	\translationnote{a}{The river ``Pishon\index{Pishon}'' (Strong's 6376\cite{Hebrew Concordance}) is a Hebrew\index{Hebrew} proper noun. This term only occurs in the Bible once.}

\bverse And the gold\index{Gold} of that land \textit{is} good. Bdellium\index{Bdellium}\vmark{a} and the onyx\index{Onyx}\vmark{b} stone \are there.
	\translationnote{a}{The word ``Bdellium\index{Bdellium}'' (Strong's 916\cite{Hebrew Concordance}) is a fragrant resin similar to myrrh\index{Myrrh}.}
	\translationnote{b}{The word ``onyx\index{Onyx}'' (Strong's 7718\cite{Hebrew Concordance}) is commonly translated as \textit{onyx}\index{Onyx}, but more accurately represents a precious gem or stone (\textit{perhaps onyx\index{Onyx}}\index{Onyx}). It is often associated with beauty, value, and it was used in the high priest's breastplate and other sacred objects.}

\bverse The name of the second river \textit{is} Gihon\index{Gihon}\vmark{a}; it \textit{is} the one which goes around the whole land of Cush.
	\translationnote{a}{The name ``Gihon\index{Gihon}'' (Strong's 1521\cite{Hebrew Concordance}) suggests a river that bursts forth or flows abundantly.}
	
\bverse The name of the third river \textit{is} Hiddekel\index{Hiddekel}\vmark{a}; it \textit{is} the one which goes toward the east of Assyria\index{Assyria}. The fourth river \textit{is} the Euphrates\index{Euphrates}\vmark{b}.
	\translationnote{a}{The name ``Hiddekel\index{Hiddekel}'' (Strong's 2313\cite{Hebrew Concordance}) is often identified as the modern day Tigris\index{Tigris} River, which flows through present-day Turkey and Iraq. \textit{Hiddekel}\index{Hiddekel} is an ancient Mesopotamian river name.}
	\historicalnote{a}{``The Tigris\index{Tigris} River, along with the Euphrates\index{Euphrates}, is one of the two major rivers of Mesopotamia, a region often referred to as the "Cradle of Civilization." This area is historically significant as it is believed to be one of the earliest centers of human civilization, with ancient cities such as Nineveh\index{Nineveh} and Babylon\index{Babylon} located along its banks. The Tigris\index{Tigris} has been a vital water\index{Water} source for agriculture and trade throughout history.\cite{Hebrew Concordance}''}
	\translationnote{b}{The name ``Euphrates\index{Euphrates}'' (Strong's 6578\cite{Hebrew Concordance}) is translated from the Hebrew\index{Hebrew} term \textit{Perath} which is frequently mentioned as a geographical landmark and boundary marker. It is one of the most significant rivers in the ancient Near East.}

\bverse Then\vmark{a} the \lord God took the man and put him in the garden of Eden\index{Eden} to tend and keep it.
	\literarynote{a}{This verse seems somewhat redundant when considered alongside \bref{Genesis 2:8}. The structure of the text reflects a Hebrew\index{Hebrew} narrative style and this verse serves to emphasize that God is assigning Adam\index{Adam} responsibilities and not just placing him in the garden.}

\bverse And the \lord God commanded the man\vmark{a}, saying, ``Of every tree\index{Tree} of the garden you may freely eat;
	\doctrinenote{a}{This is the first example of a commandment, where the following verse indicates an associated consequence.}

\bverse but of the tree\index{Tree} of the knowledge of good and evil you shall not eat, for in the day that you eat\vmark{a} of it you shall surely die\vmark{b}.''
	\translationnote{a}{The phrase ``for in the day that you eat'' sounds, in English, like it could be a known statement of the future - almost like God is saying, when this inevitably happens, here are the consequences. In Hebrew\index{Hebrew}, a verb can be in \textit{perfect tense}, which implies that it has been completed. This phrase is in the perfect tense, but is referring to a future event - which could imply that it is certain to happen. In Hebrew\index{Hebrew}, the perfect tense can sometimes be used prophetically or futuristically to indicate certainty about something that will happen. This is often determined contextually\cite{Parsons Hebrew for Christians}.}
	\translationnote{b}{As written in English, the phrase ``you shall surely die'' seems to suggest that Adam\index{Adam} would die within the day or nearly immediately. However, some would say this can be better translated as ``dying, you shall die,''\cite{Mortenson You Shall Surely Die} or ``In dying you will die,'' which would suggest this would only be the start of the process of dying.}


\bmarkerdown{Creation of Woman}

\bverse And the \lord God said, ``It is not good that a man should be alone; I will make him a helper comparable to him.''
\bverse Out of the ground the \lord God formed every beast of the field and every bird of the air, and brought \textit{them} to Adam\index{Adam} to see what he would call them. And whatever Adam\index{Adam} called each living creature, that \was its name.
\bverse So Adam\index{Adam} gave names to all cattle, to the birds of the air, and to every beast of the field. But for Adam\index{Adam} there was not found a helper comparable to him\vmark{a}.
	\contextnote{a}{This initially seems redundant, as this was mentioned in \bref{Genesis 2:18}. However, it is a Hebrew\index{Hebrew} style of writing where a thing is introduced and then further outlined and expanded upon. This is the exact same literary device mentioned in \bref{Genesis 2:15}.}

\bverse And the \lord God caused a deep sleep to fall on Adam\index{Adam}, and he slept; and He took one of his ribs, and closed up the flesh in its place.
\bverse Then the rib which the \lord God had taken from man He made into a woman, and brought her to the man.
\bverse And Adam\index{Adam} said: 
\begin{bquotation}
``This is now bone of my bones And flesh of my flesh; She shall be called Woman, Because she was taken out of Man.''
\end{bquotation}
\bverse Therefore a man shall leave his father and mother and be joined to his wife, and they shall become one flesh.
\bverse And they were both naked, the man and his wife, and were not ashamed.









