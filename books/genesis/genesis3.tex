% Chapter 3 of Genesis
\bookchapter{The Temptation and Fall of Man}


\bverse{Now the serpent\vmark{a} was more cunning than any beast of the field which the \lord God had made. And he said to the woman, ``Has God indeed said, `You shall not eat of every tree of the garden'?''}
	\crossrefnote{a}{In \bref{Revelation 12:9}, the ``serpent of old'' is referred to as ``the Devil and Satan.'' This is well known to be this serpent here in Genesis.}

\bverse{And the woman said to the serpent, ``We may eat of the fruit of the trees of the garden;}
\bverse{but of the fruit of the tree which \textit{is} in the midst of the garden God has said, `You shall not eat it, nor shall you touch it\vmark{a}, lest you die.' ''}
	\questionnote{a}{Did God say ``not shall you touch it'' or was this added by the woman? Perhaps they added this in their minds as a \textit{safeguard} to keep them from even being tempted by the tree. For if they never touch it, they surely would fulfill never eating it.}

\bverse{Then the serpent said to the woman, ``You will not surely die\vmark{a}.}
	\generalnote{a}{This is the first example of a \textit{lie} within the Bible. Satan uses a single word to change a truth to a lie. He then follows it up with what appears to be truth (though they may not know that) to both entice and persuade the woman that he knows things they do not - that God was not telling them the whole truth.}

\bverse{For God knows\vmark{a} that in the day you eat of it, your eyes will be opened, and you will be like God, knowing good and evil\vmark{b}.''}
	\generalnote{a}{This is a tactic of deception. Satan is making it seem like God knows something that he may be intentionally keeping from them.}
	\generalnote{b}{To go against God is the definition of sin, which is an evil that they will immediately know.}

\bverse{So when the woman saw that the tree \textit{was} good for food, that it \textit{was} pleasant to the eyes\vmark{a}, and a tree desirable to make \textit{one} wise\vmark{b}, she took of its fruit\vmark{c}, and ate. She also gave to her husband with her\vmark{d}, and he ate.}
	\contextnote{a}{When questioning the commandment to not eat the tree, the first thing the woman noticed was that this tree fit other characteristics of the trees which they were allowed to eat of.}
	\generalnote{b}{The serpents words were interpreted as a good thing for the woman. She took the information she was given and saw it as good.}
	\contextnote{c}{When she took of the fruit, she would have noticed that she did not immediately die. In \bref{Genesis 3:2}, she said she could not \textit{touch it} lest they die - which they likely added to Gods commandment. When seeing that she did not die from touching it, she could have deceived herself into thinking that they were not told what is true, which then made the decision to eat of it easier to come to.}
	\translationnote{d}{The phrase ``her husband with her'' perhaps makes it seem (at least how it appears in English) like Adam was with her the entire time as this was happening or at least with her when she took of the tree.}
	

\bverse{Then the eyes of both of them were opened, and they knew that they \textit{were} naked; and they sewed fig leaves together and made themselves coverings.}
\bverse{And they heard the sound of the \lord God walking in the garden in the cool of the day\vmark{a}, and Adam and his wife hid themselves from the presence of the \lord God. among the trees of the garden.}
	\contextnote{a}{This makes it seem like a regular and completely normal occurrence that God is walking around the garden with them.}

\bverse{}
\bverse{}
\bverse{}
\bverse{}
\bverse{}
\bverse{}
\bverse{}
\bverse{}
\bverse{}
\bverse{}
\bverse{}
\bverse{}
\bverse{}
\bverse{}
\bverse{}
\bverse{}
\bverse{}
