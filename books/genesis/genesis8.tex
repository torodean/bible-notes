% Chapter 8 of Genesis
\bookchapter{Noah’s Deliverance}

\bverse Then God remembered Noah\index{Noah}, and every living thing, and all the animals that \textit{were} with him in the ark\index{Ark}. And God made a wind to pass over the earth\index{Earth}, and the waters\index{Water} subsided.
\bverse The fountains of the deep\vmark{a} and the windows of heaven\index{Heaven} were also stopped, and the rain from heaven\index{Heaven} was restrained.
	\contextnote{a}{The `fountains of the deep' are clearly referred to here as something separate from the `rain from heaven\index{Heaven},' which supports the idea that water\index{Water} was coming from within earth\index{Earth} as well and not just from the rain. This is also referenced in \bref{Genesis 7:11}.}

\bverse And the waters\index{Water} receded continually from the earth\index{Earth}\vmark{a}. At the end of the hundred and fifty days the waters\index{Water} decreased.
	\literarynote{a}{This is a literary technique that is used throughout these books (see \bref{Genesis 2:8}, \bref{Genesis 2:15}, and others). In \bref{Genesis 8:1}, it clearly states the waters\index{Water} subsided (implying completion). However, it follows up by saying the waters\index{Water} were receding continually - which may seem out of order and thus contradictory. This literary format is seen repeatedly and clearly expands on the statement made prior to this rather than giving specific accounts in order.}

\bverse Then the ark\index{Ark} rested in the seventh month, the seventeenth day of the month, on the mountains of Ararat.
\bverse And the waters\index{Water} decreased continually until the tenth month. In the tenth \textit{month}, on the first \textit{day} of the month, the tops of the mountains were seen.
	\sciencenote{}{A question arises - does the earth\index{Earth} have enough water\index{Water} to fit this flood account? A simple calculation based on a few key details suggests the answer is yes. First, the ocean water\index{Water} alone would not provide enough water\index{Water} to flood the earth\index{Earth} in this manner. Thus, the ``fountains of the deep'' outlined in \bref{Genesis 8:2} and \bref{Genesis 7:11} are key to understanding how this is possible - which suggests water\index{Water} came from deep within earth\index{Earth}, and not just from the oceans. To determine approximately how much water\index{Water} would be needed to flood the earth\index{Earth}, the formula for the volume of a sphere can be used: $\frac{4}{3}\pi r^3$. Assuming earth\index{Earth} is a perfect sphere and there are about as many hills and mountains as there are valleys and ravines (assuming terrain averages out to sea level, excluding oceans and filled basins), we can estimate the amount of water\index{Water} by finding the volume of that sphere from the highest mountain (Mount Everest at 6386.8 km from the center of the earth\index{Earth}) to sea level (radius of earth\index{Earth} which is about 6378.0 km from the center of the earth\index{Earth}). This yields a total volume of water\index{Water} approximately three times that of Earth\index{Earth}’s oceans\footnote{\begin{minipage}[t]{\linewidth}The results for these calculations were obtained from a Google featured snippet displaying the value directly in the search results after prompting Google with these questions. The calculations and conversions were done using Wolfram Alpha's computational engine \cite{WolframAlpha}. This calculation is ``$4*pi*(6384.8^3 - 6378.0^3)/3$ km$^3$'', which gave ``$\approx 3\times$ volume of Earth\index{Earth}'s oceans ($1.332\times10^9$ km$^3$)'' as an automatically generated comparison value. When entering in this calculation as miles instead of kilometers, the results was 3.4 times the volume of earths\index{Earth} oceans (slightly more precise).\end{minipage}}. According to the Brookhaven National Laboratory, there is evidence for oceans of water\index{Water} deep in the earth\index{Earth}. One fascinating statement on the subject (which should be more than enough to suggest this may be possible) is ``If just one percent of the weight of mantle rock located in the transition zone is \ce{H2O}, that would be equivalent to nearly three times the amount of water\index{Water} in our oceans, the researchers said'' \cite{deep earth water}. }
	\sermonnote{}{I wrote a sermonette on this topic  titled \textit{Fountains of The Deep}, which covers it a bit more concisely \cite{Antonius' Sermonettes}.}
	
\bverse So it came to pass, at the end of the forty days, that Noah\index{Noah} opened the window of the ark\index{Ark} which he had made.
\bverse Then he sent out a raven, which kept going to and fro until the waters\index{Water} had dried from the earth\index{Earth}.
\bverse He also sent out from himself a dove, to see if the waters\index{Water} had receded from the face of the ground.
\bverse But the dove found no resting place for the sole of her foot, and she returned into the ark\index{Ark} to him, for the waters\index{Water} \textit{were} on the face of the whole earth\index{Earth}. So he put out his hand and took her, and drew her into the ark\index{Ark} himself.
\bverse And he waited yet another seven days, and again he sent the dove out from the ark\index{Ark}.
\bverse Then the dove came to him in the evening, and behold, a freshly plucked olive leaf\vmark{a} \textit{was} in her mouth; and Noah\index{Noah} knew that the waters\index{Water} had receded from the earth\index{Earth}.
	\questionnote{a}{This is interesting. Plants are obviously in a different category than the animals and people of the earth\index{Earth} as they did not need saved by the ark\index{Ark}. Perhaps this is part of what is meant by \textit{the breath\index{Breath} of life} in \bref{Genesis 7:15} and \bref{Genesis 7:22}. However, how did an olive tree\index{Tree} grow enough to bring forth olive leaves for the dove to pluck? It's possible it was just a freshly sprouted plant (small/tiny), and just a small sproutling (this can easily happen in a few days) and would likely be the case.}

\bverse So he waited yet another seven days and sent out a dove, which did not return again to him anymore. 
\bverse And it came to pass in the six hundred and first year, in the first \textit{month}, the first \textit{day} of the month\vmark{a}, that the waters\index{Water} were dried up from the earth\index{Earth}; and Noah\index{Noah} removed the covering of the ark\index{Ark} and looked, and indeed the surface of the ground was dry.
	\contextnote{a}{This time is a reference to Noah\index{Noah}'s years on the earth\index{Earth}. When the flood happened, he was 600 years old (see figure \ref{fig:noahs_genealogy} and \bref{Genesis 7:6}).}
	
\bverse And in the second month, on the twenty-seventh day of the month, the earth\index{Earth} was dried.
\bverse Then God spoke to Noah\index{Noah}, saying,
\bverse ``Go out of the ark\index{Ark}, you and your wife, and your sons and your sons' wives with you.
\bverse Bring out with you every living thing of all flesh that \textit{is} with you: birds and cattle and every creeping thing that creeps on the earth\index{Earth}, so that they may abound on the earth\index{Earth}, and be fruitful and multiply on the earth\index{Earth}.''
\bverse So Noah\index{Noah} went out, and his sons and his wife and his sons' wives with him.
\bverse Every animal, every creeping thing, every bird, \textit{and} whatever creeps on the earth\index{Earth}, according to their families, went out from the ark\index{Ark}.

\bmarkerdown{God’s Covenant\index{Covenant} with Creation}

\bverse Then Noah\index{Noah} built an altar\index{Altar} to the \lord, and took of every clean animal and of every clean bird, and offered burnt offerings on the altar\index{Altar}.
	\contextnote{}{This is the first time a \textit{burnt offering} is explicitly mentioned in scripture. It is possible that Abel\index{Abel}'s offering in \bref{Genesis 4:4}, though it does not state explicitly that it is a \text{burnt} offering. The laws\index{Law} around offerings have not yet been elaborated on or explained.}	

\bverse And the \lord smelled a smooth aroma. Then the \lord said in His heart, ``I will never again curse the ground for man's sake, although the imagination\vmark{a} of man's heart \textit{is} evil from his youth; nor will I again destroy every living thing as I have done.
	\translationnote{a}{The word `imagination' (Strong's 3336 \cite{Strong's GodRules}) means framing, or figuratively form. It can be thought of as \textit{intent} or \textit{thoughts} in this context.}

\begin{bquotation}
\bverse ``While the earth\index{Earth} remains, Seedtime and harvest, Cold and heat, Winter and summer, And day and night, Shall not cease.''
\end{bquotation}
