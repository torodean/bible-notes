% Chapter 8 of Genesis
\bookchapter{Noah’s Deliverance}

\bverse Then God remembered Noah, and every living thing, and all the animals that \textit{were} with him in the ark. And God made a wind to pass over the earth, and the waters subsided.
\bverse The fountains of the deep\vmark{a} and the windows of heaven were also stopped, and the rain from heaven was restrained.
	\contextnote{a}{The `fountains of the deep' are clearly referred to here as something separate from the `rain from heaven,' which supports the idea that water was coming from within earth as well and not just from the rain. This is also referenced in \bref{Genesis 7:11}.}

\bverse And the waters receded continually from the earth\vmark{a}. At the end of the hundred and fifty days the waters decreased.
	\literarynote{a}{This is a literary technique that is used throughout these books (see \bref{Genesis 2:8}, \bref{Genesis 2:15}, and others). In \bref{Genesis 8:1}, it clearly states the waters subsided (implying completion). However, it follows up by saying the waters were receding continually - which may seem out of order and thus contradictory. This literary format is seen repeatedly and clearly expands on the statement made prior to this rather than giving specific accounts in order.}

\bverse Then the ark rested in the seventh month, the seventeenth day of the month, on the mountains of Ararat.
\bverse And the waters decreased continually until the tenth month. In the tenth \textit{month}, on the first \textit{day} of the month, the tops of the mountains were seen.
	\sciencenote{}{A question arises - does the earth have enough water to fit this flood account? A simple calculation based on a few key details suggests the answer is yes. First, the ocean water alone would not provide enough water to flood the earth in this manner. Thus, the ``fountains of the deep'' outlined in \bref{Genesis 8:2} and \bref{Genesis 7:11} are key to understanding how this is possible - which suggests water came from deep within earth, and not just from the oceans. To determine approximately how much water would be needed to flood the earth, the formula for the volume of a sphere can be used: $\frac{4}{3}\pi r^3$. Assuming earth is a perfect sphere and there are about as many hills and mountains as there are valleys and ravines (assuming terrain averages out to sea level, excluding oceans and filled basins), we can estimate the amount of water by finding the volume of that sphere from the highest mountain (Mount Everest at 6386.8 km from the center of the earth) to sea level (radius of earth which is about 6378.0 km from the center of the earth). This yields a total volume of water approximately three times that of Earth’s oceans\footnote{\begin{minipage}[t]{\linewidth}The results for these calculations were obtained from a Google featured snippet displaying the value directly in the search results after prompting Google with these questions. The calculations and conversions were done using Wolfram Alpha's computational engine \cite{WolframAlpha}. This calculation is ``$4*pi*(6384.8^3 - 6378.0^3)/3$ km$^3$'', which gave ``$\approx 3\times$ volume of Earth's oceans ($1.332\times10^9$ km$^3$)'' as an automatically generated comparison value. When entering in this calculation as miles instead of kilometers, the results was 3.4 times the volume of earths oceans (slightly more precise).\end{minipage}}. According to the Brookhaven National Laboratory, there is evidence for oceans of water deep in the earth. One fascinating statement on the subject (which should be more than enough to suggest this may be possible) is ``If just one percent of the weight of mantle rock located in the transition zone is \ce{H2O}, that would be equivalent to nearly three times the amount of water in our oceans, the researchers said'' \cite{deep earth water}. }
	
\bverse 
\bverse 
\bverse 
\bverse 
\bverse 
\bverse 
\bverse 
\bverse 
\bverse 
\bverse 
\bverse 
\bverse 
\bverse 
\bverse 

\bmarkerdown{God’s Covenant\index{Covenant} with Creation}

\bverse 
\bverse 
\bverse 
