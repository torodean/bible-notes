% Chapter 11 of Genesis
\bookchapter{The Tower of Babel}

\bverse Now the whole earth had one language\vmark{a} and one speech.
	\translationnote{a}{The word for `language' here (Strong's 8193 \cite{Strong's GodRules}) is distinct from the word `tongues' used in \bref{Genesis 10}. This word more literally means \textit{language} or \textit{speech}.}
	\contextnote{a}{In \bref{Genesis 10}, it makes it seem like the different lineages were separated based on their \textit{tongues}, which could mean languages. However, to be under one language here means either that \textit{tongues} specifically refers to different dialects \textit{or} that this languages is referring to something else such as the language of mathematics. We see in modern society that even though many countries have different languages, we all can write in common terms using something like mathematics. Though the `one speech' makes it seem like a more literal form of spoken language.}

\bverse And it came to pass, as they journeyed from the east, that they found a plain in the land of Shinar, and they dwelt there.
\bverse Then they said to one another, ``Come, let us make bricks and bake \textit{them} thoroughly.'' They had brick for stone, and they had asphalt for mortar.
\bverse And they said, ``Come, let us build ourselves a city, and a tower whos top \is in the heavens; let us make a name for ourselves, lest we be scattered abroad over the face of the whole earth.''
	\questionnote{}{We see here that they (the people of the time) did \textit{not} want to be scattered abroad over the face of the earth. Perhaps they knew something about sticking together that would give them an advantage? I wonder why they did not want to spread out. Perhaps they were advancing too quickly?}

\bverse But the \lord came down to see the city and the tower which the sons of men had build.
\bverse And the \lord said, ``Indeed the people \are one and they all have one language, and this is what they begin to do; now nothing that they propose to do will be withheld from them.
\bverse Come, let Us go down and there confuse their language, that they may not understand one another's speech.''
\bverse So the \lord scattered them abroad from there over the face of all the earth, and they ceased building the city.
\bverse Therefore its name is called Babel, because there the \lord confused the language of all the earth; and from there the \lord scattered them abroad over the face of all the earth.
	\questionnote{}{For some reason, the \lord decided that he had to scatter them abroad the face of the earth - which is directly opposed to what the people wanted (see \bref{Genesis 11:4}). Why did he find the need or desire to do this?}

\bmarkerdown{Shem's Descendants}

\bverse This \is the genealogy of Shem: Shem was one hundred years old, and begot Arphaxad two years after the flood.
\bverse After he begot Arphaxad, Shem lived five hundred years, and begot sons and daughters.
\bverse Arphaxad lived thirty-five years, and begot Salah.
\bverse After he begot Salah, Arphaxad lived four hundred and three years, and begot sons and daughters.
\bverse Salah lived thirty years, and begot Eber.
\bverse After he begot Eber, Salah lived four hundred and three years, and begot sons and daughters.
\bverse Eber lived thirty-four years, and begot Peleg.
\bverse After he begot Peleg, Eber lived four hundred and thirty years, and begot sons and daughters.
\bverse Peleg lived thirty years, and begot Reu. After he begot Reu, Peleg lived two hundred and nine years, and begot sons and daughters.
\bverse 
\bverse 
\bverse 
\bverse 
\bverse 
\bverse 
\bverse 
\bverse 
\bverse 
\bverse 
\bverse 
\bverse 
\bverse 
\bverse 
