% Chapter 4 of Genesis
\bookchapter{\name{Cain} Murders \name{Abel}}


\bverse Now \name{Adam} knew\vmark{a} \name{Eve} his wife, and she conceived and bore \name{Cain}\vmark{b}, and said, ``I have acquired a man from the \lord.''
	\contextnote{a}{To \textit{know} someone in the Bible (as it is used like this) means to have sexual relations with them.}
	\translationnote{b}{The word/name `\name{Cain}' (Strong's 7014\cite{Strong's GodRules}) means `possession.'}

\bverse Then she bore again, this time his brother \name{Abel}\vmark{a}. Now \name{Abel} was a keeper of sheep, but \name{Cain} was a tiller of the ground.
	\translationnote{a}{The word/name `\name{Abel}' (Strong's 1893\cite{Strong's GodRules}) means `breath\index{Breath}.'}
	\contextnote{}{There is an implied passage of time in this story which is definitely not explicitly stated. First, it does not state that these are the only children \name{Adam} and \name{Eve} bore. It also states the professions of \name{Cain} and \name{Abel}, which implies they had to have grown up and began working (babies cannot work). The directly implies that some unknown (potentially very long) amount of tie has passed during these events.}
	
\bverse And in the process of time it came to pass that \name{Cain} brought an offering of fruit of the ground to the \lord.
\bverse \name{Abel} also brought of the firstborn of his flock and of their fat. And the \lord respected \name{Abel} and his offering,
\bverse but He did not respect \name{Cain} and his offering\vmark{a}. And \name{Cain} was very angry, and his countenance fell.
	\contextnote{a}{In \bref{1 John 3:12}, we see that \name{Cain}'s works were evil - referring to his works outside of the act of killing \name{Abel}. It also says that \name{Abel}'s works were righteous. This is likely why his offering here was not respected.}

\bverse So the \lord said to \name{Cain}, ``Why are you angry? And why has your countenance fallen?
\bverse If you do well, will you not be accepted?\vmark{a} And if you do not do well, sin\index{Sin} lies at the door\vmark{b}. And its desire \textit{is} for you, but you should rule over it.''
	\questionnote{a}{The context here is a little unclear. Is this perhaps Gods way of saying ``what did you expect to happen? You were not doing what you are suppose to and you knew better'' (aka: typical stubborn human behavior).}
	\contextnote{b}{This could refer to a \textit{sin offering}. If \name{Cain} needed to give a sin\index{Sin} offering, but did not, God would have been displeased. Perhaps \name{Cain} did something that he was not suppose to but acted as if everything was fine. This is supported in \bref{1 John 3:12}, where it specifically says \name{Cain}'s works were evil, and that is why he eventually murdered \name{Abel}.}

\bverse Now \name{Cain} talked with \name{Abel} his brother; and it came to pass, when they were in the field, that \name{Cain} rose up against \name{Abel} his brother and killed him.
\bverse Then the \lord said to \name{Cain}, ``Where is \name{Abel} your brother?'' He said, ``I do not know. \textit{Am} I my brother's keeper?''
\bverse And He said, ``What have you done? The voice of your brother's blood\index{Blood} cries out to Me from the ground.
	\generalnote{}{The blood\index{Blood} crying out from the ground has an implication of God having a connection to our \textit{blood}\index{Blood} (or soul) after we are physically dead. This is obviously important, as it shows that some part of us remains in death.}

\bverse So now you \are cursed from the earth\index{Earth}, which has opened its mouth to receive your brother's blood\index{Blood} from your hand.
\bverse When you till the ground, it shall no longer yield its strength to you. A fugitive and a vagabond\vmark{a} you shall be on the earth\index{Earth}.''
	\geographynote{a}{A vagabond (in English) is a person who wanders from place to place without a home or job - having no settled home\cite{Dictionary-Merriam-Webster}. This is a fitting term since it says later in \bref{Genesis 4:16} that he dwelt in the land of \textit{Nod}, which means wandering.}
	
\bverse And \name{Cain} said to the \lord. ``My punishment \textit{is} greater than I can bear!
\bverse Surely You have driven me out this day from the face of the ground: I shall be hidden from Your face: I shall be a fugitive and a vagabond on the earth\index{Earth}, and it will happen \that anyone who finds me will kill me.''
\bverse And the \lord said to him, ``Therefore, whoever kills \name{Cain}, vengeance shall be taken on him sevenfold.'' And the \lord set a mark on \name{Cain}, lest anyone finding him should kill him.

\bmarkerdown{The Family Of \name{Cain}}

\bverse Then \name{Cain} went out from the presence of the \lord and dwelt in the land of Nod\vmark{a} on the east of Eden\index{Eden}.
	\translationnote{a}{The word/name `Nod' (Strong's 5113\cite{Strong's GodRules}) means `wandering.'}

\bverse And \name{Cain} knew his wife, and she conceived and bore \name{Enoch}\vmark{a}. And he built a great city, and called the name of the city after the name of his son--\name{Enoch}.
	\translationnote{a}{The word/name `\name{Enoch}' (Strong's 2585\cite{Strong's GodRules}) means `dedicated.'}
	
\bverse To \name{Enoch} was born Irad\vmark{a}; and Irad begot \name{Mehujael}\vmark{b}, and \name{Mehujael} begot \name{Methusael}\vmark{c}, and \name{Methusael} begot \name{Lamech}\vmark{d}.
	\translationnote{a}{The word/name `Irad' (Strong's 5897\cite{Strong's GodRules}) means `fleet.'}
	\translationnote{b}{The word/name `\name{Mehujael}' (Strong's 4232\cite{Strong's GodRules}) means `smitten by God.'}
	\translationnote{c}{The word/name `\name{Methusael}' (Strong's 4967\cite{Strong's GodRules}) means `who is of God.'}
	\translationnote{c}{The word/name `\name{Lamech}' (Strong's 4232\cite{Strong's GodRules}) means `powerful.'}

\bverse Then \name{Lamech} took for himself two wives: the name of one \was \name{Adah}, and the name of the second \was \name{Zillah}.
\bverse And \name{Adah} bore \name{Jabal}. He was the father of those who dwell in tents and have livestock.
\bverse His brother's name \was Jubal. He was the father of all those who play the harp and flute.
\bverse And as for \name{Zillah}, she also bore Tubal-\name{Cain}, an instructor of every craftsman in bronze and iron. And the sister of Tubal-\name{Cain} \was \name{Naamah}.
\bverse Then \name{Lamech} said to his wives:
\begin{bquotation}
``\name{Adah} and \name{Zillah}, hear my voice; Wives of \name{Lamech}, listen to my speech! For I have killed a man for wounding me, Even a young man for hurting me. \bverse If \name{Cain} shall be avenged sevenfold, Then \name{Lamech} seventy-sevenfold.''
\end{bquotation}

\bmarkerdown{\name{Adam} Has A New Son: \name{Seth}}

\bverse And \name{Adam} knew his wife again, and she bore a son and named him \name{Seth}, ``For God has appointed another seed for me instead of \name{Abel}, whom \name{Cain} killed.''
\bverse And as for \name{Seth}, to him also a son was born; and he named him \name{Enosh}. Then \textit{men} began to call on the name of the \lord.
