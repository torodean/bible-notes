% Chapter 4 of Genesis
\bookchapter{Cain Murders Abel}


\bverse Now Adam knew\vmark{a} Eve his wife, and she conceived and bore Cain\vmark{b}, and said, ``I have acquired a man from the \lord.''
	\contextnote{a}{To \textit{know} someone in the Bible (as it is used like this) means to have sexual relations with them.}
	\translationnote{b}{The word/name `Cain' (Strong's 7014\cite{Strong's GodRules}) means `possession.'}

\bverse Then she bore again, this time his brother Abel\vmark{a}. Now Abel was a keeper of sheep, but Cain was a tiller of the ground.
	\translationnote{a}{The word/name `Abel' (Strong's 1893\cite{Strong's GodRules}) means `breath.'}
	\contextnote{}{There is an implied passage of time in this story which is definitely not explicitly stated. First, it does not state that these are the only children Adam and Eve bore. It also states the professions of Cain and Abel, which implies they had to have grown up and began working (babies cannot work). The directly implies that some unknown (potentially very long) amount of tie has passed during these events.}
	
\bverse And in the process of time it came to pass that Cain brought an offering of fruit of the ground to the \lord.
\bverse Abel also brought of the firstborn of his flock and of their fat. And the \lord respected Abel and his offering,
\bverse but He did not respect Cain and his offering. And Cain was very angry, and his countenance fell.
\bverse So the \lord said to Cain, ``Why are you angry? And why has your countenance fallen?
\bverse If you do well, will you not be accepted?\vmark{a} And if you do not do well, sin lies at the door\vmark{b}. And its desire \textit{is} for you, but you should rule over it.''
	\questionnote{a}{The context here is a little unclear. Is this perhaps Gods way of saying ``what did you expect to happen? You were not doing what you are suppose to and you knew better'' (aka: typical stubborn human behavior).}
	\questionnote{b}{This could refer to a \textit{sin offering}. If Cain needed to give a sin offering, but did not, God would have been displeased. Perhaps Cain did something that he was not suppose to but acted as if everything was fine.}

\bverse Now Cain talked with Abel his brother; and it came to pass, when they were in the field, that Cain rose up against Abel his brother and killed him.
\bverse Then the \lord said to Cain, ``Where is Abel your brother?'' He said, ``I do not know. \textit{Am} I my brother's keeper?''
\bverse And He said, ``What have you done? The voice of your brother's blood cries out to Me from the ground.
	\generalnote{}{The blood crying out from the ground has an implication of God having a connection to our \textit{blood} (or soul) after we are physically dead. This is obviously important, as it shows that some part of us remains in death.}

\bverse So now you \textit{are} cursed from the earth, which has opened its mouth to receive your brother's blood from your hand.
\bverse When you till the ground, it shall no longer yield its strength to you. A fugitive and a vagabond you shall be on the earth.''
\bverse And Cain said to the \lord. ``My punishment \textit{is} greater than I can bear!
\bverse Surely You have driven me out this day from the face of the ground: I shall be hidden from Your face: I shall be a fugitive and a vagabond on the earth, and it will happen \textit{that} anyone who finds me will kill me.''
\bverse And the \lord said to him, ``Therefore, whoever kills Cain, vengeance shall be taken on him sevenfold.'' And the \lord set a mark on Cain, lest anyone finding him should kill him.

\bmarkerdown{The Family Of Cain}

\bverse Then Cain went out from the presence of the \lord and dwelt in the land of Nod\vmark{a} on the east of Eden.
	\translationnote{a}{The word/name `Nod' (Strong's 5113\cite{Strong's GodRules}) means `wandering.'}

\bverse And Cain knew his wife, and she conceived and bore Enoch\vmark{a}. And he built a great city, and called the name of the city after the name of his son--Enoch.
	\translationnote{a}{The word/name `Enoch' (Strong's 2585\cite{Strong's GodRules}) means `dedicated.'}
	
\bverse To Enoch was born Irad\vmark{a}; and Irad begot Mehujael\vmark{b}, and Mehujael begot Methusael\vmark{c}, and Methusael begot Lamech\vmark{d}.
	\translationnote{a}{The word/name `Irad' (Strong's 5897\cite{Strong's GodRules}) means `fleet.'}
	\translationnote{b}{The word/name `Mehujael' (Strong's 4232\cite{Strong's GodRules}) means `smitten by God.'}
	\translationnote{c}{The word/name `Methusael' (Strong's 4967\cite{Strong's GodRules}) means `who is of God.'}
	\translationnote{c}{The word/name `Lamech' (Strong's 4232\cite{Strong's GodRules}) means `powerful.'}

\bverse 
\bverse 
\bverse 
\bverse 
\bverse 
\bverse 
\bverse 
\bverse 
