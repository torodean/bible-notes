\booktitle{Genesis}{\lipsum[1]}
\thispagestyle{fancy}

\bookchapter{The Creation}
\bverse{In the beginning God\vmark{a} created\vmark{b} the heaven\vmark{c} and the earth.}

\historicalnote{a}{The Hebrew *Elohim* is a plural form, but used with singular verbs.}
\translationnote{a}{"Created" translates *bara*, a verb uniquely used for divine acts of creation.}
\contextnote{b}{"Heaven" and "earth" together form a merism—a figure of speech meaning "everything."}
\sciencenote{b}{The idea of a beginning of time aligns with modern cosmological models like the Big Bang.}
\questionnote{c}{Does this verse imply a creation from nothing (*ex nihilo*) or an ordering of chaos?}
\generalmnote{c}{This verse sets the foundation for the biblical narrative and worldview.}



\bverse{And the earth was without form, and void; and darkness was upon the face of the deep.}


\bookchapter{The Creation}
\bverse{In the beginning God created the heaven and the earth. test test, here is some text.}
\bverse{And the earth was without form, and void; and darkness was upon the face of the deep.}
\bverse{And the earth was without form, and void; and darkness was upon the face of the deep.}
\bverse{And the earth was without form, and void; and darkness was upon the face of the deep.}